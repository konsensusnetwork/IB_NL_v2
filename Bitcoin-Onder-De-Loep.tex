% Options for packages loaded elsewhere
\PassOptionsToPackage{unicode}{hyperref}
\PassOptionsToPackage{hyphens}{url}
\PassOptionsToPackage{dvipsnames,svgnames,x11names}{xcolor}
%
\documentclass[
  letterpaper,
]{scrbook}

\usepackage{amsmath,amssymb}
\usepackage[]{Alegreya}
\usepackage{iftex}
\ifPDFTeX
  \usepackage[T1]{fontenc}
  \usepackage[utf8]{inputenc}
  \usepackage{textcomp} % provide euro and other symbols
\else % if luatex or xetex
  \usepackage{unicode-math}
  \defaultfontfeatures{Scale=MatchLowercase}
  \defaultfontfeatures[\rmfamily]{Ligatures=TeX,Scale=1}
\fi
% Use upquote if available, for straight quotes in verbatim environments
\IfFileExists{upquote.sty}{\usepackage{upquote}}{}
\IfFileExists{microtype.sty}{% use microtype if available
  \usepackage[]{microtype}
  \UseMicrotypeSet[protrusion]{basicmath} % disable protrusion for tt fonts
}{}
\makeatletter
\@ifundefined{KOMAClassName}{% if non-KOMA class
  \IfFileExists{parskip.sty}{%
    \usepackage{parskip}
  }{% else
    \setlength{\parindent}{0pt}
    \setlength{\parskip}{6pt plus 2pt minus 1pt}}
}{% if KOMA class
  \KOMAoptions{parskip=half}}
\makeatother
\usepackage{xcolor}
\usepackage[top=30mm,left=20mm,heightrounded]{geometry}
\setlength{\emergencystretch}{3em} % prevent overfull lines
\setcounter{secnumdepth}{5}
% Make \paragraph and \subparagraph free-standing
\ifx\paragraph\undefined\else
  \let\oldparagraph\paragraph
  \renewcommand{\paragraph}[1]{\oldparagraph{#1}\mbox{}}
\fi
\ifx\subparagraph\undefined\else
  \let\oldsubparagraph\subparagraph
  \renewcommand{\subparagraph}[1]{\oldsubparagraph{#1}\mbox{}}
\fi


\providecommand{\tightlist}{%
  \setlength{\itemsep}{0pt}\setlength{\parskip}{0pt}}\usepackage{longtable,booktabs,array}
\usepackage{calc} % for calculating minipage widths
% Correct order of tables after \paragraph or \subparagraph
\usepackage{etoolbox}
\makeatletter
\patchcmd\longtable{\par}{\if@noskipsec\mbox{}\fi\par}{}{}
\makeatother
% Allow footnotes in longtable head/foot
\IfFileExists{footnotehyper.sty}{\usepackage{footnotehyper}}{\usepackage{footnote}}
\makesavenoteenv{longtable}
\usepackage{graphicx}
\makeatletter
\def\maxwidth{\ifdim\Gin@nat@width>\linewidth\linewidth\else\Gin@nat@width\fi}
\def\maxheight{\ifdim\Gin@nat@height>\textheight\textheight\else\Gin@nat@height\fi}
\makeatother
% Scale images if necessary, so that they will not overflow the page
% margins by default, and it is still possible to overwrite the defaults
% using explicit options in \includegraphics[width, height, ...]{}
\setkeys{Gin}{width=\maxwidth,height=\maxheight,keepaspectratio}
% Set default figure placement to htbp
\makeatletter
\def\fps@figure{htbp}
\makeatother
\newlength{\cslhangindent}
\setlength{\cslhangindent}{1.5em}
\newlength{\csllabelwidth}
\setlength{\csllabelwidth}{3em}
\newlength{\cslentryspacingunit} % times entry-spacing
\setlength{\cslentryspacingunit}{\parskip}
\newenvironment{CSLReferences}[2] % #1 hanging-ident, #2 entry spacing
 {% don't indent paragraphs
  \setlength{\parindent}{0pt}
  % turn on hanging indent if param 1 is 1
  \ifodd #1
  \let\oldpar\par
  \def\par{\hangindent=\cslhangindent\oldpar}
  \fi
  % set entry spacing
  \setlength{\parskip}{#2\cslentryspacingunit}
 }%
 {}
\usepackage{calc}
\newcommand{\CSLBlock}[1]{#1\hfill\break}
\newcommand{\CSLLeftMargin}[1]{\parbox[t]{\csllabelwidth}{#1}}
\newcommand{\CSLRightInline}[1]{\parbox[t]{\linewidth - \csllabelwidth}{#1}\break}
\newcommand{\CSLIndent}[1]{\hspace{\cslhangindent}#1}

\makeatletter
\makeatother
\makeatletter
\@ifpackageloaded{bookmark}{}{\usepackage{bookmark}}
\makeatother
\makeatletter
\@ifpackageloaded{caption}{}{\usepackage{caption}}
\AtBeginDocument{%
\ifdefined\contentsname
  \renewcommand*\contentsname{Table of contents}
\else
  \newcommand\contentsname{Table of contents}
\fi
\ifdefined\listfigurename
  \renewcommand*\listfigurename{List of Figures}
\else
  \newcommand\listfigurename{List of Figures}
\fi
\ifdefined\listtablename
  \renewcommand*\listtablename{List of Tables}
\else
  \newcommand\listtablename{List of Tables}
\fi
\ifdefined\figurename
  \renewcommand*\figurename{Figure}
\else
  \newcommand\figurename{Figure}
\fi
\ifdefined\tablename
  \renewcommand*\tablename{Table}
\else
  \newcommand\tablename{Table}
\fi
}
\@ifpackageloaded{float}{}{\usepackage{float}}
\floatstyle{ruled}
\@ifundefined{c@chapter}{\newfloat{codelisting}{h}{lop}}{\newfloat{codelisting}{h}{lop}[chapter]}
\floatname{codelisting}{Listing}
\newcommand*\listoflistings{\listof{codelisting}{List of Listings}}
\makeatother
\makeatletter
\@ifpackageloaded{caption}{}{\usepackage{caption}}
\@ifpackageloaded{subcaption}{}{\usepackage{subcaption}}
\makeatother
\makeatletter
\@ifpackageloaded{tcolorbox}{}{\usepackage[many]{tcolorbox}}
\makeatother
\makeatletter
\@ifundefined{shadecolor}{\definecolor{shadecolor}{rgb}{.97, .97, .97}}
\makeatother
\makeatletter
\makeatother
\ifLuaTeX
  \usepackage{selnolig}  % disable illegal ligatures
\fi
\IfFileExists{bookmark.sty}{\usepackage{bookmark}}{\usepackage{hyperref}}
\IfFileExists{xurl.sty}{\usepackage{xurl}}{} % add URL line breaks if available
\urlstyle{same} % disable monospaced font for URLs
\hypersetup{
  pdftitle={Bitcoin Onder De Loep},
  pdfauthor={Yann Pritzker},
  colorlinks=true,
  linkcolor={Maroon},
  filecolor={Maroon},
  citecolor={Blue},
  urlcolor={Blue},
  pdfcreator={LaTeX via pandoc}}

\title{Bitcoin Onder De Loep}
\usepackage{etoolbox}
\makeatletter
\providecommand{\subtitle}[1]{% add subtitle to \maketitle
  \apptocmd{\@title}{\par {\large #1 \par}}{}{}
}
\makeatother
\subtitle{De technologie achter het eerste werkelijk schaarse en
gedecentraliseerde geld}
\author{Yann Pritzker}
\date{3/3/23}

\begin{document}
\frontmatter
\maketitle
\ifdefined\Shaded\renewenvironment{Shaded}{\begin{tcolorbox}[sharp corners, boxrule=0pt, enhanced, interior hidden, breakable, borderline west={3pt}{0pt}{shadecolor}, frame hidden]}{\end{tcolorbox}}\fi

\renewcommand*\contentsname{Inhoudsopgave}
{
\hypersetup{linkcolor=}
\setcounter{tocdepth}{2}
\tableofcontents
}
\mainmatter
\bookmarksetup{startatroot}

\hypertarget{preface}{%
\chapter*{Preface}\label{preface}}
\addcontentsline{toc}{chapter}{Preface}

\markboth{Preface}{Preface}

This is a Quarto book.

To learn more about Quarto books visit
\url{https://quarto.org/docs/books}.

\begin{center}\rule{0.5\linewidth}{0.5pt}\end{center}

Gepubliceerd door

Konsensus Network

\bookmarksetup{startatroot}

\hypertarget{inleiding}{%
\chapter{Inleiding}\label{inleiding}}

Veel mensen die voor het eerst over bitcoin horen zijn al snel geneigd
om een mening te vormen voor ze een poging doen om het te begrijpen. Dat
wordt bemoeilijkt doordat je door een flinke laag (mis)informatie heen
moet om te begrijpen wat bitcoin is en hoe het werkt. Tot drie jaar
geleden, behoorde ik ook tot een van de mensen met een mening op basis
van onvoldoende kennis.

Waarom schrijf ik dit boek? De laatste twintig jaar heb ik mij gericht
op het opstarten van technische start-ups. Elke dag stort ik mij in een
nieuwe technologie en ik ben vrij goed geworden in het uitdokteren hoe
iets werkt. Desondanks duurde het vijf jaar sinds ik voor het eerst over
bitcoin hoorde, voordat ik besloot om er goed voor te gaan zitten om het
te begrijpen. Ik heb het gevoel dat ik niet de enige ben die een beetje
hulp kan gebruiken om deze mogelijk wereldveranderende innovatie beter
te begrijpen.

Ik hoorde voor het eerst over bitcoin in 2011 van slashdot.org, een
nieuwssite voor nerds. In die tijd was de prijs gestegen tot de enorme
piek van \$30 dollar per bitcoin. Alles wat ik erover wist was dat
sommige mensen op het internet probeerden om een peer-to-peer
betaalsysteem op te starten. Niet wetende wat het precies was, hoe het
werkte en terwijl ik niks wist over investeren en marktcycli, besloot ik
toch wat geld in te leggen voor het geval het iets belangrijks zou
worden. Ik moest de verschrikkelijk uitziende website van Mt. Gox
gebruiken om dat te doen. Dit dollar-naar-bitcoinhandelsplatform bleek
later onbekwaam.

Langzaam zag ik mijn investering krimpen tot nagenoeg niets, terwijl de
prijs daalde van \$30 naar \$2. Op een gegeven moment vergat ik het
volledig en ging ik verder met mijn leven, de start-ups. Ik weet niet
eens wat er met die bitcoins gebeurd is. Ik denk dat ik de sleutels
ergens heb opgeslagen op een oude laptop die inmiddels op de vuilnisbelt
ligt.

In 2013 hoorde ik weer over bitcoin. Dit keer was het geluid in de media
luider en de aanschaf ging een stuk soepeler. Er waren apps zoals
Coinbase, die er legitiem uitzagen. Dit was een duidelijke vooruitgang
op het tijdperk van Mt. Gox. Ik kreeg sterk het gevoel dat bitcoin
weleens zou kunnen slagen.

In het geval dit zo was en weer zonder enige kennis, kocht ik weer op de
piek (rond \$1000 per bitcoin) en zag mijn investering weer instorten,
maar nu naar een koers van \$200 per bitcoin. Dit keer besloot ik dat
het de moeite niet waard was om de bitcoins te verkopen en dus besloot
ik het zo te laten. Ik ging verder en keek er niet meer naar om, omdat
ik me op mijn volgende start-up richtte; Reverb.com.

In de daaropvolgende vier jaren groeide Reverb sterk. Het werd dé
website voor muzikanten. Ik betekende iets voor de wereld doordat ik
muziek en mensen verbond. Ik was hoofd Technologie (CTO) van een snel
groeiend en opwindend techbedrijf. Ik deed iets waar ik gepassioneerd
over was en ik had geen tijd voor vreemd internetgeld.

Ik voel me beschaamd om te vertellen dat het pas in de zomer van 2016
was, dat ik voor het eerst een video van
\href{https://www.youtube.com/channel/UCJWCJCWOxBYSi5DhCieLOLQ}{\textbf{Andreas
Antonopoulos}} bekeek. Daardoor begon ik mezelf zaken af te vragen als;
waar komen bitcoins vandaan? Wie beheert het? Hoe werkt het? Wat is
mining en welke impact zal het hebben op de wereld? Ik begon me te
verdiepen. Anderhalf jaar lang las ik alles waar mijn oog op viel,
luisterde ik uren podcasts en keek ik elke video over bitcoin die ik
tegenkwam.

En toen eindelijk, begin 2018, net nadat de koers van bitcoin een nieuwe
recordhoogte had bereikt rond \$20.000 per bitcoin, besloot ik om Reverb
achter me te laten en me volledig op bitcoin te richten. Waarom ik mijn
succesvolle start-up verliet voor bitcoin? Omdat ik geloof dat een
uitvinding van iets als bitcoin, maar een keer in een mensenleven zich
voordoet. En misschien zelfs nog minder vaak.

Als bitcoin slaagt, kan het net zo belangrijk blijken als de uitvinding
van de drukpers (decentralisatie van de productie van informatie), het
internet (decentralisatie van data en communicatie) en trias politicas
(decentralisatie van de overheid). Ik hoop dat door te begrijpen hoe
bitcoin werkt, je zal begrijpen hoe het de wereld ten goede kan komen.
Bitcoin zal de productie en consumptie van geld decentraliseren; dat is
het middel waarmee de mensheid tot nieuwe ontwikkelingen kan komen op
een schaal die voorheen ondenkbaar was.

In de media gaat het vaak alleen over de koers van bitcoin. Het ene
moment wordt de suggestie gewekt dat hij naar een miljoen dollar gaat,
het andere moment zit hij in een neerwaartse spiraal die pas zal stoppen
als bitcoin waardeloos is geworden. En anders zijn er wel verhalen dat
bitcoin zoveel energie gebruikt dat het de aarde binnen tien jaar zal
vernietigen. Dit is natuurlijk fout en ik hoop dat je dat zal begrijpen
zodra je leert hoe het werkt. Je zal ook begrijpen waarom koers-bubbels
de minst interessante dingen zijn wat betreft bitcoin.

Met dit boek probeer ik niet de economie van bitcoin en gedegen geld uit
te leggen, hoewel we die concepten wel kort zullen aanraken. Ik ga
bitcoin ook niet beschrijven vanuit een investeringsstandpunt of je
overtuigen dat iedereen een beetje bitcoin zou moeten hebben. Ik raad
iedereen aan om \emph{De Bitcoin Standaard} van Saifedean Ammous te
lezen als je dat nog niet hebt gedaan.\footnote{Nvdr: ook dit boek is
  inmiddels in het Nederlands vertaald en verkrijgbaar via Konsensus
  Netwerk.}

Ook zal dit boek niet uitwijden over de computercode, en computerkennis
is niet noodzakelijk om dit boek te begrijpen. Als je naar bitcoin wilt
kijken vanuit dat perspectief, raad ik je Antonopoulos' \emph{Mastering
Bitcoin} en Jimmy Songs \emph{Programming Bitcoin} aan.

Voor mij gaat het om het begrijpen hoe alle dingen samenkomen waardoor
bitcoin werkt. Met dit boek hoop ik die kennis met je te kunnen delen.
Mijn doel is om je hersenen een beetje te kietelen en om je met een
vleugje computerkennis, economisch en speltheoretisch inzicht te geven
hoe bitcoin een van de meest interessante en belangrijkste uitvindingen
is van deze tijd. Als je begrijpt hoe bitcoin werkt hoop ik dat jij, net
als ik, zal inzien dat bitcoin veel meer is dan het op het eerste
gezicht lijkt en dat het een geweldige impact zal hebben op de volgende
generaties van deze wereld.

In dit boek nemen we bitcoin onder de loep en laat ik je stap voor stap
zien hoe alles samenkomt. Ik hoop dat je genoeg kennis opdoet om daarna
je reis \emph{down the rabbit hole} te vervolgen. Laten we beginnen!

\part{Hoofdstukken}

\hypertarget{wat-is-bitcoin}{%
\chapter{Wat is bitcoin?}\label{wat-is-bitcoin}}

Bitcoin is een vorm van \emph{peer-to-peer elektronisch geld}, een
nieuwe vorm van digitaal geld dat kan worden overgedragen tussen mensen
of computers zonder enige vertrouwde tussenpersoon (zoals een bank), en
waarvan de uitgifte niet gecontroleerd wordt door één enkele partij.

Denk aan een papieren dollar of fysieke, metalen munt. Wanneer je die
aan anderen geeft, hoeven ze niet te weten wie je bent. Zij moeten er
alleen op te vertrouwen dat het geld dat ze van je krijgen geen
vervalsing is. Normaal gesproken vertrouwen mensen bij fysiek geld op
hun ogen en vingers, of bij grotere hoeveelheden op het gebruik van
testapparatuur om vals van echt te onderscheiden.

Maar in de digitale samenleving worden de meeste van onze betalingen
gedaan via een tussenpersoon: via een bank met behulp van IDeal, een
kredietkaartmaatschappij zoals Visa, een digitale betalingsprovider
zoals PayPal, of WeChat in China.

Met de overgang naar de digitale wereld is geld veranderd van een fysiek
iets dat je zelf kunt bijhouden, overdragen en verifiëren, naar digitale
bits die door een derde partij worden opgeslagen, geverifieerd en
verstuurd. Daardoor ben je voor elke digitale financiële handeling nu
afhankelijk van een derde partij.

Omdat we ons contant geld opgeven voor het gemak van digitale
betalingen, creëren we ook een systeem waarin anderen buitengewone macht
hebben om ons te onderdrukken. Digitale betalingsplatformen zijn de
basis geworden van dystopische, autoritaire controlesystemen die
bijvoorbeeld worden gebruikt door de Chinese overheid om afvalligen te
controleren en om te voorkomen dat specifieke burgers goederen of
diensten gebruiken.

Bitcoin biedt een alternatief voor centraal gestuurd, digitaal geld. Het
biedt os een systeem dat ons de vrijheid van contant geld teruggeeft,
maar op digitale wijze:

\begin{enumerate}
\def\labelenumi{\arabic{enumi}.}
\item
  Een digitaal bezit waarvan het aanbod beperkt, vooraf bekend en
  onveranderlijk is. Dit staat in schril contrast met bankbiljetten en
  hun digitale versies die worden uitgegeven door overheden en centrale
  banken, waarvan het aantal zich in een onvoorspelbaar tempo blijft
  uitbreiden.
\item
  Een stel onderling verbonden computers (\emph{het bitcoinnetwerk}),
  waar iedereen kan aan deelnemen door een stukje software te draaien op
  hun computer. Dit netwerk dient om bitcoins uit te geven, hun
  eigendomstitel te volgen en om overdrachten tussen deelnemers uit te
  voeren, zonder te vertrouwen op tussenpersonen zoals banken,
  betalingsbedrijven en overheidsinstanties. Zie
  Figure~\ref{fig-bitcoinnetwerk}.
\item
  De bitcoinclientsoftware, een stukje computercode die iedereen kan
  uitvoeren om deelnemer te worden van het netwerk. Deze software is
  \emph{open source}, wat betekent dat iedereen kan zien hoe het werkt
  en bij kan dragen aan nieuwe functies en verbeteringen.
\end{enumerate}

\begin{figure}

{\centering \includegraphics{./images/fig1.png}

}

\caption{\label{fig-bitcoinnetwerk}Bitcoin is een netwerk van computers
die de bitcoinclientsoftware draaien.}

\end{figure}

Later in het boek komen we terug op de motivaties achter bitcoin.

\hypertarget{waar-komt-bitcoin-vandaan}{%
\section{Waar komt bitcoin vandaan?}\label{waar-komt-bitcoin-vandaan}}

Bitcoin is rond 2008 uitgevonden door een of meer personen die bekend
staan onder het pseudoniem
\href{https://nl.wikipedia.org/wiki/Satoshi_Nakamoto}{\textbf{Satoshi
Nakamoto}}\footnote{Door de gemeenschap wordt gerefereerd aan Satoshi
  als man. In dit boek zullen we ook als dusdanig naar hem refereren
  ondanks dat het onbekend is of het gaat om een man, vrouw of groep
  personen}. Niemand kent de identiteit van Satoshi, en voor zover we
weten is hij verdwenen en heeft hij al jaren niets meer van zich laten
horen.

Op 11 februari 2009 berichtte Satoshi over een vroege versie van bitcoin
op een online forum voor mensen die werken aan cryptografie en veel
waarde hechten aan individuele privacy en vrijheid ---
\emph{cypherpunks}. Hoewel dit niet de eerste officiële aankondiging van
bitcoin was, bevatte het een goede samenvatting van Satoshi's
beweegredenen. Daarom gebruik ik het om de basis te leggen voor onze
discussie.

Ik zal een aantal citaten weergeven die verduidelijken welke problemen
van het huidige financiële systeem Satoshi probeerde op te lossen:

\begin{quote}
Ik heb een nieuw open source P2P e-cash systeem ontwikkeld genaamd
bitcoin. Het is volledig gedecentraliseerd, zonder centrale server of
vertrouwde partijen. Alles is gebaseerd op cryptografisch bewijs in
plaats van vertrouwen. {[}...{]}

Het kernprobleem met conventionele valuta is het vertrouwen dat nodig is
om het te laten werken. De centrale bank moet worden vertrouwd, maar de
geschiedenis van fiatvaluta zit vol met schendingen van dat vertrouwen.
We moeten banken vertrouwen om ons geld te bewaren en elektronisch over
te maken, maar ze lenen het uit in golven van kredietbubbels met
nauwelijks een fractie van die waarde in reserve. We moeten hen onze
privacy toevertrouwen, en we moeten erop vertrouwen dat ze
identiteitsdieven weerhouden om onze rekeningen te plunderen. Hun enorme
overheadkosten maken microbetalingen onmogelijk.

Vroeger hadden multi-user time-sharing computersystemen een
vergelijkbaar probleem. Voor sterke encryptie moesten gebruikers nog
vertrouwen op wachtwoorden om hun bestanden te beveiligen {[}...{]}

Toen werd sterke encryptie beschikbaar voor de massa en was vertrouwen
niet langer nodig. Gegevens konden worden beveiligd waardoor het voor
anderen onmogelijk was om toegang te krijgen, ongeacht om welke reden,
ongeacht hoe goed het excuus, wat er ook gebeurt.

Het wordt tijd dat we dit ook hebben voor geld. Met e-valuta op basis
van cryptografisch bewijs, zonder de noodzaak om een externe
tussenpersoon te vertrouwen. Geld kan veilig zijn en transacties
moeiteloos. {[}...{]}

De oplossing van bitcoin is om een peer-to-peer-netwerk te gebruiken om
te controleren op dubbele uitgaven. In een notendop werkt het netwerk
als een gedistribueerde tijdstempel, waarbij de eerste uitgave van een
munt van een tijdstempel wordt voorzien. Het maakt gebruik van het feit
dat informatie gemakkelijk te verspreiden is, maar moeilijk is om tegen
te houden.

Voor meer informatie over hoe het werkt, zie het ontwerpdocument op
\url{http://www.bitcoin.org/bitcoin.pdf}.\footnote{Kijk voor de
  Nederlandse versie van de whitepaper op:
  \url{https://btcdirect.eu/nl-nl/bitcoin-whitepaper}}

--- {SATOSHI NAKAMOTO}
\end{quote}

\hypertarget{welke-problemen-lost-het-op}{%
\section{Welke problemen lost het
op?}\label{welke-problemen-lost-het-op}}

Laten we een aantal van Satoshi's beweringen nader onderzoeken. Doorheen
het boek zullen we bespreken hoe deze concepten daadwerkelijk worden
geïmplementeerd. Maak je geen zorgen als iets onbekend aanvoelt in deze
sectie want we gaan er later dieper op in. Het idee is om de bedoeling
van Satoshi te begrijpen, zodat we ze later in dit boek kunnen
onderzoeken.

\begin{quote}
\emph{Ik heb een nieuw open source P2P e-cash systeem ontwikkeld}
\end{quote}

P2P staat voor \emph{peer-to-peer} en geeft een systeem aan waarbij één
persoon een interactie heeft met een ander, zonder tussenpartij. De twee
partijen (ontvanger en verzender) zijn gelijkwaardig. Je herinnert je
misschien P2P-technologieën zoals Napster, Kazaa en BitTorrent, waarmee
mensen voor het eerst met elkaar bestanden, muziek en films konden delen
zonder tussenpersoon. Satoshi ontwierp bitcoin om mensen de mogelijkheid
te geven om op vrijwel dezelfde manier elektronisch contant geld
(\emph{e-cash}) uit te wisselen zonder tussenpersoon.

De software is \emph{open source}, wat betekent dat iedereen kan zien
hoe het werkt en iedereen kan bijdragen. Dit is belangrijk omdat het
zorgt voor transparantie. Er is geen vertrouwen nodig. We hoeven niets
te geloven van wat Satoshi schreef in zijn berichten over hoe de
software werkt. We kunnen de code bekijken en controleren hoe het werkt.
Bovendien kunnen we de functionaliteit van het systeem verbeteren door
de code te wijzigen.

\begin{quote}
\emph{Het is volledig gedecentraliseerd, zonder centrale server of
vertrouwde partijen...}
\end{quote}

Satoshi vermeldt dat het systeem \emph{gedecentraliseerd} is om het te
onderscheiden van systemen die wel centrale aansturing hebben. Eerdere
pogingen om digitaal contant geld te creëren, zoals DigiCash van David
Chaum, werden ondersteund door een \emph{centrale server}; een computer
of set van computers die verantwoordelijk waren voor uitgifte en
verificatie van betalingen, onder controle van één bedrijf.

Dergelijke centraal aangestuurde particuliere vormen van digitaal geld
zijn gedoemd te mislukken; we kunnen niet vertrouwen op geld dat kan
verdwijnen wanneer een bedrijf failliet gaat, wordt gehackt, last heeft
van IT-problemen of wordt gestopt door de overheid.

Bitcoin, aan de andere kant, wordt niet gerund en gecontroleerd door een
enkel bedrijf, maar door een netwerk van individuen en bedrijven van
over de hele wereld. Het stoppen van bitcoin vereist het stoppen van
tienduizenden tot honderdduizenden computers over de hele wereld,
waarvan velen lastig te traceren zijn. Het is een hopeloos
kat-en-muisspel aangezien elke aanval van deze aard eenvoudigweg de
creatie van nieuwe \emph{bitcoin-nodes} of computers op het netwerk
aanmoedigt.

\begin{quote}
\emph{... alles is gebaseerd op cryptografisch bewijs in plaats van
vertrouwen}
\end{quote}

Het internet en de meeste moderne computersystemen zijn gebouwd op
cryptografie; een methode om informatie te versleutelen zodat alleen de
ontvanger van de informatie deze kan ontcijferen. Hoe ontsnapt bitcoin
aan de noodzaak van \emph{vertrouwen}? We zullen hier later in het boek
op ingaan, maar het basisidee is dat in plaats van iemand te vertrouwen
die zegt Ik ben Alice of Ik heb \$10 in mijn account, we cryptografie
kunnen gebruiken om dezelfde feiten dusdanig te presenteren zodat de
ontvanger van het bericht dit eenvoudig zelf kan verifiëren en het
onmogelijk is om te vervalsen. Bitcoin maakt gebruik van cryptografie om
deelnemers in staat te stellen het gedrag van alle anderen te
controleren, zonder dat hierbij een centrale partij vertrouwt hoeft te
worden.

\begin{quote}
\emph{We moeten hen {[}de banken{]} onze privacy toevertrouwen, en we
moeten erop vertrouwen dat ze identiteitsdieven ervan weerhouden om onze
rekeningen te plunderen}
\end{quote}

In tegenstelling tot het gebruik van een bankrekening, het digitale
betalingssysteem of kredietkaarten, stelt bitcoin twee partijen in staat
om transacties uit te voeren zonder persoonlijke identificatie op te
geven. Banken, kredietkaartmaatschappijen, betalingsverwerkers en
overheden beschikken over gecentraliseerde databanken van
consumentengegevens. Deze gegevens zijn een gigantische buit voor
hackers. Zo werden bij de hack van Equifax in 2017 de identiteits- en
financiële gegevens van meer dan 140 miljoen mensen buitgemaakt. Dit
soort databanken en de bijbehorende hacks kunnen resulteren in
identiteitsfraude op grote schaal.

Bitcoin ontkoppelt financiële transacties van identiteiten uit de echte
wereld. Immers, wanneer we contant geld aan iemand geven, hoeft de
ontvanger niet te weten wie we zijn en hoeft de betaler zich geen zorgen
te maken dat zijn informatie wordt gebruikt om hem op een later moment
te bestelen. Waarom zouden we niet hetzelfde, of meer, verwachten van
digitaal geld?

\begin{quote}
\emph{De centrale bank moet worden vertrouwd om de valuta niet te
devalueren, maar de geschiedenis van fiatvaluta zit vol met schendingen
van dat vertrouwen}
\end{quote}

\emph{Fiat}, Latijn voor laat het gebeuren, verwijst naar valuta
uitgegeven door de overheid en centrale bank en dat door de overheid als
wettig betaalmiddel is aangenomen. Historisch gezien werd geld gemaakt
uit dingen die moeilijk zijn om te produceren, gemakkelijk zijn om te
verifiëren en gemakkelijk zijn om te vervoeren, zoals schelpen,
glaskralen, zilver en goud. Telkens wanneer iets als geld werd gebruikt,
bestond de verleiding om er meer van te maken. Als iemand langskwam met
superieure technologie om snel veel van iets te creëren, verloor het
goed zijn waarde. Zo konden Europese kolonisten het Afrikaans continent
ontdoen van haar rijkdom, door te handelen in glazen kralen die voor de
Europeanen gemakkelijk, maar voor de Afrikanen moeilijk, te produceren
waren. Dit is waarom goud al zo lang wordt beschouwd als betrouwbare
vorm van geld - het is moeilijk om snel meer goud te
produceren.\footnote{Voor een goed overzicht van de monetaire
  geschiedenis raad ik het essay \emph{Shelling Out} van Nick Szabo aan:
  \url{https://nakamotoinstitute.org/shelling-out}}

We zijn langzaam overgestapt van een wereldeconomie met goud als geld
naar een wereld waarbij papieren certificaten werden uitgegeven als
claim op datzelfde goud. Uiteindelijk werden door president Nixon de
papieren claims volledig losgekoppeld van goud. Hij maakte in 1971 een
einde aan de internationale inwisselbaarheid van de Amerikaanse dollar
voor goud.

Het einde van de goudstandaard stelde overheden en centrale banken in
staat om de geldhoeveelheid naar believen te vergroten, waardoor ieder
biljet in omloop minder waard werd. Dit staat bekend als
\emph{geldontwaarding}. Hoewel fiatvaluta wordt uitgegeven door een
regering, het inwisselbaar is voor niets, en we het allemaal dagelijks
gebruiken, is het eigenlijk een relatief nieuw experiment in de
wereldgeschiedenis.

We moeten erop vertrouwen dat onze overheden de drukpers niet
misbruiken, maar ver hoeven we niet te zoeken om voorbeelden van
schendingen van dat vertrouwen te vinden. Dit zien we voornamelijk terug
in autocratische regimes, waar de overheid directe invloed op de
geldpers heeft. Een bekend voorbeeld is Venezuela, waar de munt nagenoeg
waardeloos is geworden. De Venezolaanse bolivar ging van een koers van 2
bolivar per Amerikaanse dollar in 2009 naar 250.000 bolivar per
Amerikaanse dollar in 2019. Terwijl ik dit boek schrijf, is de
ineenstorting van Venezuela volop aan de gang als gevolg van het
vreselijke economische wanbeleid van de regering.

Satoshi wilde een alternatief bieden voor \emph{fiatvaluta} waarvan het
aantal te allen tijde onvoorspelbaar kan worden verhoogd. Om
\emph{ontwaarding} te voorkomen, ontwierp Satoshi een geldsysteem
waarbij het totale aanbod vooraf werd vastgelegd en het uitgeven van
nieuwe munten een voorspelbaar en onveranderlijk patroon volgt. Er
zullen slechts 21 miljoen bitcoins bestaan en elke bitcoin kan worden
verdeeld in 100 miljoen eenheden die nu satoshis worden genoemd. In de
code is vastgelegd dat rond het jaar 2140 het eindtotaal wordt bereikt
van 2,1 biljard satoshis.

Tot bitcoin was het onmogelijk om te verkomen dat een digitaal goed
oneindig werd gekopieerd. Het is goedkoop en gemakkelijk om een digitaal
boek, audio- of video-bestand digitaal te kopiëren en door te sturen. De
enige uitzonderingen hierop waren digitale goederen die beheerd werden
door tussenpersonen. Bijvoorbeeld wanneer je een film kijkt via Netflix,
dan kun je deze alleen op jouw apparaat bekijken omdat Netflix de film
levert. Je kan deze film zelf niet verspreiden of kopiëren. Op dezelfde
manier wordt je digitale geld beheerd door de bank. Het is de taak van
de bank om bij te houden hoeveel geld je hebt, en als je het aan iemand
anders overdraagt, regelt de bank de overdracht.\footnote{Ze zijn dus
  ook in staat deze te weigeren}

Bitcoin is het eerste digitale systeem dat schaarste afdwingt zonder
tussenpersonen en het is het eerste goed waarvan het totale aanbod en
het uitgifteschema vooraf bekend is. Zelfs edelmetalen zoals goud hebben
deze eigenschap niet, omdat we meer goud kunnen delven als het rendabel
is om dat te doen. Stel je voor dat je een asteroïde ontdekt die tien
keer zoveel goud bevat als wij op aarde hebben. Wat zou er gebeuren met
de prijs van goud? Bitcoin is immuun voor dergelijke ontdekkingen. Het
is onmogelijk om er meer van te produceren, en we zullen in latere
hoofdstukken uitleggen waarom.

De aard van geld en de werking van het bestaande monetaire systeem zijn
ingewikkeld. Dit boek gaat hier niet dieper op in. Als je hier meer over
wilt weten in de context van bitcoin, dan raad ik \emph{De Bitcoin
Standaard} van Saifedean Ammous aan.

\begin{quote}
\emph{Gegevens konden worden beveiligd op een manier waardoor het voor
anderen onmogelijk was om toegang te krijgen, ongeacht om welke reden,
ongeacht hoe goed het excuus, wat er ook gebeurt. {[}...{]} Het wordt
tijd dat we dit ook hebben voor geld.}
\end{quote}

Onze huidige systemen om geld veilig te stellen, zoals het op de bank
zetten, vertrouwen erop dat iemand anders zijn werk goed doet.
Vertrouwen op zo'n tussenpersoon vereist niet alleen vertrouwen dat ze
niets kwaadaardigs of dwaas zullen doen, maar ook vertrouwen dat de
overheid niet via druk op de tussenpersoon dergelijke dingen doet. Denk
hierbij aan zaken als iemand de toegang tot hun geld ontzeggen of het
geld onteigenen. Helaas is keer op keer aangetoond dat overheden wanneer
zij bedreiging verwachten of zien, dit soort dingen kunnen doen en tot
uitvoering brengen.

Het klinkt misschien gek voor iemand die in de Verenigde Staten woont
(of in een andere sterk gereguleerde economie) om te denken dat je op
een dag wakker kan worden en dat dan al je geld weg is, maar het gebeurt
de hele tijd. Ik heb zelf eens voorgehad dat mijn tegoed op Paypal
bevroren was omdat ik mijn account al maanden niet had gebruikt. Het
kostte me meer dan een week om weer toegang te krijgen tot mijn geld. Ik
heb het geluk dat ik in de Verenigde Staten woon, waar ik tenminste
juridische hulp kan zoeken als PayPal mijn tegoed bevriest, en waar ik
er in principe op vertrouw dat mijn regering en de bank mijn geld niet
zullen stelen.

Er zijn veel ergere dingen gebeurd en nog steeds aan de orde in landen
met minder vrijheid, zoals banken die sluiten tijdens de crisis in
Griekenland (2015)\footnote{Zie
  \href{https://www.nbcnews.com/business/business-news/greece-crisis-banks-shut-week-restrictions-imposed-atms-n383606/}{deze
  link}}, banken in Cyprus die via bail-ins het geld van hun klanten in
beslag nemen (2013), of de overheid die bepaalde bankbiljetten
waardeloos verklaart in India (2016)\footnote{Zie
  \href{https://www.washingtonpost.com/world/asia_pacific/india-invalidates-large-bank-notes-in-crackdown-on-crime/2016/11/08/cc705ee2-a5c6-11e6-ba46-53db57f0e351_story.html}{deze
  link}}.

Ik ben opgegroeid in de voormalige Sovjet-Unie. Daar reguleerde de
overheid de economie, wat leidde tot enorme tekorten aan goederen. Het
was illegaal om vreemde valuta zoals de Amerikaanse dollar te bezitten.
Toen mijn ouders weg wilden, konden we slechts een beperkt bedrag per
persoon naar dollars omwisselen. De wisselkoers was door de overheid
bepaald en ver verwijderd van de wisselkoers op de vrije markt. In feite
heeft de regering ons het kleine beetje vermogen dat we hadden ontnomen,
door een ijzeren greep te houden op de economie en het betalingsverkeer.

Autocratische landen hebben de neiging om strenge economische controles
in te voeren om te voorkomen dat mensen hun geld uit banken opnemen, het
land uitvoeren of inruilen voor nog-niet-waardeloze valuta zoals de
Amerikaanse dollar. Hierdoor heeft de regering vrij spel om krankzinnige
economische experimenten, zoals het socialistische systeem van de
Sovjet-Unie, uit te voeren.

Bitcoin werkt niet op basis van vertrouwen in een derde partij voor het
veilig stellen van je geld. In plaats daarvan maakt bitcoin het
\emph{onmogelijk} voor anderen om toegang te krijgen zonder een speciale
sleutel die alleen de eigenaar bezit, \emph{ongeacht om welke reden,
ongeacht hoe goed het excuus, wat er ook gebeurt}. Door bitcoin te
bezitten, bezit je de sleutels van je eigen financiële vrijheid. Bitcoin
scheidt geld en staat.

\begin{quote}
\emph{De oplossing van bitcoin is om een peer-to-peer-netwerk te
gebruiken om te controleren op dubbele uitgaven {[}...{]} als een
gedistribueerde tijdstempel, waarbij de eerste uitgave van een munt van
een tijdstempel wordt voorzien.}
\end{quote}

Een \emph{netwerk} verwijst naar het idee dat een stel computers zijn
verbonden en informatie naar elkaar kunnen sturen. Het woord
\emph{gedistribueerd} betekent dat er geen centrale partij aan de macht
is, maar dat alle deelnemers samenwerken om het netwerk succesvol te
maken.

In een systeem zonder centrale controle is het belangrijk dat niemand
vals kan spelen. Het idee van \emph{dubbele uitgaven} verwijst naar de
mogelijkheid om twee keer hetzelfde geld uit te geven. Dit is geen
probleem met fysiek geld, want het geld wisselt van hand zodra je het
uitgeeft. Digitale transacties kunnen echter worden gekopieerd, net als
muziek of films. Wanneer je geld overmaakt via een bank, zorgt de bank
ervoor dat je niet twee keer hetzelfde geld kunt verplaatsen. In een
systeem zonder centrale controle, hebben we een manier nodig om dit
soort \emph{dubbele uitgaven} te voorkomen (Dubbele uitgaven zijn in
feite hetzelfde als geld vervalsen).

Satoshi beschrijft dat de deelnemers van het bitcoinnetwerk samenwerken
om transacties van \emph{tijdstempels} te voorzien. Door deze
tijdstempel weten we welke transactie eerst kwam zodat we toekomstige
pogingen om datzelfde geld uit te geven afwijzen. In de volgende
hoofdstukken zullen we dit systeem vanaf de basis bespreken. Het zal ons
in staat stellen om vals spel te detecteren zonder te vertrouwen op een
centrale uitgever of validator.

Bitcoin was geen uitvinding die op zichzelf stond. In de paper noemde
Satoshi verschillende belangrijke pogingen om soortgelijke systemen te
implementeren, waaronder Wei Dai's b-money en Adam Back's Hashcash.
Ondanks de technologische inzichten van voorgangers, had nog niemand de
juiste oplossing gevonden. Maar de uitvinding van bitcoin bracht daar
verandering in, waardoor het eerste systeem voor de uitgifte en het
overmaken van een echt schaars, digitaal geld zonder centrale controle
mogelijk werd.

Satoshi loste een aantal interessante technische problemen op om de
kwesties van privacy, ontwaarding en centrale controle in de huidige
monetaire stelsels aan te pakken:

\begin{enumerate}
\def\labelenumi{\arabic{enumi}.}
\item
  Hoe zet je een peer-to-peer-netwerk op, waar iedereen vrijwillig aan
  kan deelnemen.
\item
  Hoe koppel je een groep mensen zodat zij gezamenlijk een grootboek bij
  kunnen houden, zonder hun identiteit te onthullen en zonder dat zij
  elkaar hoeven te vertrouwen, zelfs als sommigen van hen oneerlijk
  zijn.
\item
  Hoe stel je mensen in staat om hun eigen, onvervalsbare valuta uit te
  geven, zonder op een centrale uitgever te steunen om de schaarste te
  verzekeren.
\end{enumerate}

Toen bitcoin van start ging, gebruikte slechts een handvol mensen het en
draaiden zij de bitcoin-software op hun \emph{nodes} (computers, we gaan
hier later verder op in). De meeste mensen dachten destijds dat het een
grap was, of dat het systeem na verloop van tijd ernstige ontwerpfouten
zou bevatten waardoor het zou falen.

Maar steeds meer mensen sloten zich aan bij het netwerk. Ze beveiligden
het netwerk met hun computers, ruilden hun valuta in voor bitcoin, of
accepteerden bitcoin in ruil voor goederen of diensten. Dit alles
verstevigde de gedachte dat bitcoin waarde had. Vandaag, tien jaar
later, wordt bitcoin gebruikt door miljoenen mensen en draaien
tienduizenden tot honderdduizenden nodes de gratis bitcoin software,
ontwikkelt door honderden vrijwilligers en bedrijven wereldwijd.

Laten we onderzoeken hoe we zo'n systeem kunnen bouwen!

\hypertarget{de-tussenpersoon-verwijderen}{%
\chapter{De tussenpersoon
verwijderen}\label{de-tussenpersoon-verwijderen}}

In het vorige hoofdstuk hebben we besproken dat bitcoin een peer-to-peer
systeem is voor de overdracht van waarde. Voordat we ingaan op hoe dat
werkt, kijken we eerst hoe een traditionele bank of betalingsverwerker
omgaat met het beheer van eigendom en overdrachten van activa.

\hypertarget{banken-zijn-slechts-grootboeken}{%
\section{Banken zijn slechts
grootboeken}\label{banken-zijn-slechts-grootboeken}}

Hoe werkt een digitale betaling via de bank, PayPal of ApplePay? Heel
eenvoudig, fungeren deze tussenpersonen als veredelde grootboeken van
rekeningen en overschrijvingen.

Het doel van een bank is om tegoeden op te slaan en te bewaken. Maar
tegoeden zijn tegenwoordig voornamelijk elektronisch, in plaats van
munten of papier. Als zodanig is het nu de taak van een bank om data te
beheren en te bewaken. Omdat de gegevens elektronisch zijn, is de
beveiliging ook voornamelijk elektronisch. Banken maken gebruik van
softwarematige inbraakdetectiesystemen, back-ups om te beschermen tegen
gegevensverlies, audits door derden om er zeker van te zijn dat interne
processen niet in het gedrang komen en ze verzekeren zich uit voorzorg
voor het geval er iets misgaat.

Hieronder zie je hoe banken werken. In dit voorbeeld hebben we het over
een bank, maar je kan dit lezen als elke partij die betalingen verwerkt.
We beginnen met een grootboek van rekeningen waaruit blijkt dat Alice en
Bob geld hebben gestort bij de bank.

TABEL

Wanneer Alice €2 naar Bob wil sturen, belt ze haar bank of gebruikt ze
een website of mobiele app van haar bank, logt ze in bij de bank met
behulp van een gebruikersnaam en wachtwoord of pincode, en doet
vervolgens het verzoek tot overdracht. De bank registreert dit in hun
grootboek.

TABEL

De bank heeft de credits en debets en bijbehorende tegoeden
geregistreerd, en nu is het geld overgedragen.

\hypertarget{het-dubbele-uitgavenprobleem}{%
\section{Het
dubbele-uitgavenprobleem}\label{het-dubbele-uitgavenprobleem}}

Wat gebeurt er als Alice diezelfde twee dollar nu weer probeert uit te
geven? Dit wordt het \emph{dubbele-uitgavenprobleem} genoemd. Zij dient
het verzoek in bij de bank, maar de bank zegt: "Sorry, we zien dat je
die €2 al hebt uitgegeven aan Bob. Je kan dat geld niet meer uitgeven."

Wanneer je een centrale autoriteit hebt zoals een bank, is het heel
gemakkelijk voor de bank om te zien dat je probeert geld uit te geven
dat je al hebt uitgegeven. Dat komt omdat zij de enigen zijn die het
grootboek kunnen wijzigen. Zij beschikken over verschillende interne
processen zoals back-ups en audits, om ervoor te zorgen dat het
grootboek correct is en om te zorgen dat er niet mee wordt geknoeid.

We noemen dit een \emph{gecentraliseerd} systeem omdat het een enkel
controlepunt heeft.

\begin{figure}

{\centering \includegraphics{./images/fig2.png}

}

\caption{De bank houdt een grootboek bij waar iedereen toegang toe
heeft, maar alleen via de bank bij kan.}

\end{figure}

\hypertarget{het-grootboek-delen-tussen-verschillende-partijen}{%
\section{Het grootboek delen tussen verschillende
partijen}\label{het-grootboek-delen-tussen-verschillende-partijen}}

Het eerste probleem dat bitcoin wil oplossen, is het verwijderen van een
tussenpersoon door het creëren van een \emph{peer-to-peer} systeem. Stel
je voor dat banken verdwenen zijn en we ons financiële systeem opnieuw
moeten creëren. Hoe kunnen we een grootboek bijhouden zonder centrale
partij?

Als we niet één centraal grootboek hebben, moet het zo zijn dat het
grootboek van het volk is. \emph{Vive la révolution}. Dit is hoe we dat
doen.

Eerst komen een aantal mensen samen en creëren zij een \emph{netwerk}.
Dit betekent simpelweg dat een manier bestaat om informatie met elkaar
te delen. Laten we zeggen dat we telefoonnummers of Snapchat-accounts
uitwisselen. Wanneer Alice geld wil overmaken naar Bob, belt ze niet
naar de bank, maar zegt ze tegen al haar vrienden: Ik stuur €2 naar Bob.
Iedereen bevestigt met: Top, we noteren het, en schrijft het in hun
eigen kopie van het grootboek. Het ziet er uit zoals in
Figure~\ref{fig-figure3}.

\begin{figure}

{\centering \includegraphics{./images/fig3.png}

}

\caption{\label{fig-figure3}Iedereen heeft een eigen kopie van het
grootboek, waar ze onafhankelijk van een ander bij kunnen.}

\end{figure}

Dus nu heeft iedereen (in plaats van enkel de bank) een kopie van het
grootboek in handen. Telkens wanneer iemand geld wil uitgeven, vertellen
ze het aan al hun vrienden. Iedereen houdt de transacties bij. Het
grootboek staat niet meer op één plek. Dit heet \emph{gedistribueerd}.
We noemen we het ook \emph{decentraal}, omdat geen enkele centrale
partij het voor het zeggen heeft. Er hoeft geen tussenpersoon meer
vertrouwd te worden.

Nu we geen tussenpersoon hebben, hoe gaan we dan om met dubbele
uitgaven? Wie of wat kunnen we (in plaats van de bank) raadplegen om na
te gaan of het geld dat wordt uitgegeven nog niet is uitgegeven? Omdat
iedereen een kopie van het grootboek heeft, moeten we iedereen
raadplegen. Het systeem zoals we nu bespreken, is \emph{gebaseerd op
consensus} omdat er in het netwerk consensus heerst. Iedereen is het
eens over een bepaalde versie van de waarheid.

Als Alice probeert om de €2 die ze al naar Bob heeft gestuurd opnieuw
uit te geven, zal haar transactie worden afgewezen door iedereen op het
netwerk. De leden op het netwerk zullen hun grootboeken raadplegen en
aan Alice vertellen dat het geld al is uitgegeven. Ze zouden haar tweede
transactie van diezelfde €2 dus niet opnemen. We hebben nu een
peer-to-peer consensusnetwerk voor het registreren van eigendom en
overdrachten van tegoeden.

Zolang partijen \emph{toestemming} nodig hebben om deel te mogen nemen
aan ons gedistribueerde grootboek, en we erop kunnen \emph{vertrouwen}
dat elke partij eerlijk is, werkt het systeem. Maar dit soort ontwerpen
kunnen niet geschaald worden om door miljoenen mensen van over de hele
wereld te worden gebruikt. Gedistribueerde systemen bestaande uit
willekeurige deelnemers zijn inherent onbetrouwbaar. Sommige mensen gaan
af en toe offline. Dat betekent dat ze mogelijk niet op de hoogte zijn
van onze transacties op het moment dat we die uitsturen. Anderen
proberen ons misschien actief te bedriegen door te zeggen dat bepaalde
transacties juist wel of niet hebben plaatsgevonden. Nieuwe mensen
kunnen zich aansluiten bij het netwerk en zo ontstaan conflicterende
kopieën van het grootboek.

In het volgende deel onderzoeken we hoe iemand zou kunnen proberen om
vals te spelen.

\hypertarget{de-dubbele-uitgaven-aanval}{%
\section{De dubbele-uitgaven aanval}\label{de-dubbele-uitgaven-aanval}}

Als ik Alice ben, kan ik \emph{samenspannen} met een aantal van de
andere mensen en hen vertellen: Als ik geld uitgeef, schrijf het dan
niet in jullie grootboek. Doe alsof het nooit gebeurd is. Laten we eens
kijken hoe Alice zo'n dubbele-uitgaven aanval kan uitvoeren.

Beginnend met een saldo van €2, doet Alice het volgende:

\begin{enumerate}
\def\labelenumi{\arabic{enumi}.}
\item
  Ze stuurt haar €2 naar Bob, om een reep chocolade te kopen. Nu heeft
  Alice €0 over.
\item
  David, Eva en Femke spannen samen met Alice en schrijven de transactie
  van Alice naar Bob niet in hun grootboeken. In hun exemplaar heeft
  Alice haar geld nooit uitgegeven en heeft ze nog steeds een saldo van
  €2.
\item
  Charlotte is een eerlijke grootboekhouder. Ze registreert correct de
  transactie van Alice naar Bob. In haar grootboek heeft Alice €0.
\item
  Henri was een week op vakantie en heeft nog nooit van de transactie
  gehoord. Hij sluit zich aan bij het netwerk en vraagt om een kopie van
  het grootboek.
\item
  Henri krijgt 4 valse kopieën (David, Eva, Femke en Alice) en één
  eerlijke kopie (Charlotte). Hoe bepaalt hij welke echt is? Zonder
  beter systeem vertrouwt hij de meerderheid van deelnemers en wordt dus
  misleidt. Hij neemt aan dat het nep-grootboek klopt.
\item
  Alice koopt een chocoladereep van Henri met de €2 die ze eigenlijk
  niet heeft. Henri accepteert het omdat voor zo ver hij weet, Alice nog
  steeds €2 op haar rekening heeft.
\item
  Alice heeft nu 2 chocoladerepen en er is €4 aan nepgeld gemaakt in het
  systeem. Ze betaalt haar vrienden met chocoladerepen, en ze herhalen
  de aanval 100 keer op elke nieuwe persoon die lid wordt van het
  netwerk.
\item
  Alice heeft nu alle chocoladerepen en alle anderen hebben grote zakken
  nepgeld gekregen.
\item
  Wanneer de verkopers van chocoladerepen het geld willen uitgeven dat
  Alice ze heeft gestuurd zullen Alice, David, Eva en Femke (die de
  meerderheid van de netwerk uitmaken) deze uitgaven afwijzen omdat ze
  weten dat het nepgeld is.
\end{enumerate}

Dit wordt een \emph{consensusfout} genoemd. De mensen in het netwerk
kwamen niet tot consensus over de stand van zaken. Doordat er geen beter
systeem voor handen was, werd de meerderheidsregel gehanteerd, wat er
voor zorgde dat oneerlijke mensen het netwerk konden bedotten en geld
konden uitgeven dat ze eigenlijk niet hadden.

Als we een systeem willen maken dat \emph{zonder toestemming} werkt,
waar iedereen aan kan deelnemen zonder iemand permissie te moeten
vragen, dan moet het weerbaar zijn tegen oneerlijke deelnemers.

\hypertarget{het-oplossen-van-het-gedistribueerd-consensus-probleem}{%
\section{Het oplossen van het gedistribueerd consensus
probleem}\label{het-oplossen-van-het-gedistribueerd-consensus-probleem}}

Nu komen we bij een van de moeilijkste problemen in de technologie:
consensus delen tussen partijen waar sommige oneerlijk of onbetrouwbaar
zijn. Dit probleem staat bekend als \emph{het probleem van de
Byzantijnse Generalen} en de oplossing bleek de sleutel tot succes van
Satoshi Nakamoto's ontdekking. Een heleboel mensen moeten het eens
worden over de transacties in het grootboek zonder te weten welke
grootboekhouders alle transacties correct en eerlijk hebben
opgeschreven.

Een naïeve oplossing is simpelweg het aanstellen van eerlijke
grootboekhouders. In plaats van iedereen het grootboek laten bijhouden,
kiezen we een handvol vrienden zoals Charlotte, Geert, Frank en Zoe om
het te doen, omdat ze geen leugens vertellen en iedereen weet dat ze
nooit feesten in het weekend.

Dus telkens we een transactie willen verwerken, bellen we Charlotte en
de rest op. Ze zijn blij om het grootboek voor ons bij te houden en
vragen slechts een kleine vergoeding. Nadat zij de transactie in het
grootboek hebben geschreven, bellen ze de anderen om ze op de hoogte te
brengen van de wijziging, waarna ook zij die aan het grootboek toevoegen
als back-up.

Dit systeem werkt heel goed, tot op een dag agenten willen weten wie dit
schaduw financiële systeem draaiende houdt. Ze arresteren Charlotte,
Geert, Frank en Zoe en nemen hen mee, waardoor een einde komt aan ons
gedistribueerde grootboek. We hebben allemaal verschillende back-ups,
kunnen elkaar niet vertrouwen en kunnen niet achterhalen wiens back-up
moet worden gebruikt om een nieuw systeem te starten.

In plaats van een volledige sluiting, zou de overheid onze
grootboekhouders ook stilletjes met gevangenisstraffen kunnen bedreigen
als ze transacties naar Alice accepteren (die verdacht wordt van het
verkopen van drugs). Het systeem is nu effectief onder centrale controle
en we kunnen het niet langer noemen.

Wat als we democratie proberen? Laten we een groep van 50 eerlijke
mensen zoeken en we houden dagelijks verkiezingen om te bepalen wie die
dag het grootboek bij mag houden. Iedereen in het netwerk krijgt een
stem.

Dit systeem werkt geweldig totdat mensen geweld of financiële dwang
gebruiken om dezelfde doelen te bereiken als voorheen:

\begin{enumerate}
\def\labelenumi{\arabic{enumi}.}
\item
  Dwing het electoraat om te stemmen op de grootboekhouders van hun
  keuze.
\item
  Dwing de gekozen grootboekhouders om valse vermeldingen in het
  grootboek te verwerken of juist bepaalde transacties tegen te houden.
\end{enumerate}

We hebben een probleem. Telkens wanneer we specifieke mensen aanstellen
om het grootboek bij te houden, moeten we erop vertrouwen dat ze eerlijk
handelen. Er is geen enkele manier om hen te verdedigen tegen dwang om
oneerlijk te handelen en het grootboek te corrumperen.

\hypertarget{valse-identiteit-en-sybil-aanvallen}{%
\section{Valse identiteit en
Sybil-aanvallen}\label{valse-identiteit-en-sybil-aanvallen}}

Tot nu toe hebben we twee mislukte methoden gezien om eerlijkheid te
garanderen: de een maakte gebruik van specifieke grootboekhouders en de
ander van democratisch gekozen en roterende grootboekhouders. De zwakte
van beide systemen is dat het vertrouwen gekoppeld is aan identiteiten
uit de reële wereld: er moeten personen worden aangewezen die
verantwoordelijk zijn voor het grootboek. Wanneer we een systeem
hanteren waar vertrouwen afhankelijk is van identiteit, lopen we kans op
een zogeheten \emph{Sybil-aanval}. Dit is eigenlijk een mooie naam voor
imitatie. Het houdt in dat iemand zich voordoet als een ander en is
vernoemd naar een vrouw met een meervoudige persoonlijkheidsstoornis.

Heb je ooit een rare SMS ontvangen van een van je vrienden, waarna bleek
dat zijn telefoon(nummer) was \emph{gehackt}? Als het om miljarden of
zelfs biljoenen dollars gaat, zullen mensen allerlei vormen van geweld
rechtvaardigen om die telefoon te stelen en die SMS te versturen. Het is
absoluut noodzakelijk dat we voorkomen dat de mensen die het grootboek
bijhouden, op welke manier dan ook, tot zaken gedwongen kunnen worden.
Maar hoe doen we dat?

\hypertarget{we-starten-een-loterij}{%
\section{We starten een loterij}\label{we-starten-een-loterij}}

Als we niet willen dat mensen worden beïnvloed door dreigementen met
geweld of omkoping, hebben we een systeem nodig met zoveel
grootboekhouders dat het onhaalbaar is om ze te dwingen. Of nog beter;
we willen dat hun identiteit geheim blijft. Het moet zo zijn dat
iedereen aan ons systeem kan deelnemen en dat we niet afhankelijk zijn
van een vorm van stemmen, omdat zo'n systeem gevoelig is voor omkoping
of afpersing.

Wat als we een loterij gebruiken waarbij telkens een willekeurige
persoon wordt gekozen om in het grootboek te schrijven? Hier is het
eerste ontwerp:

\begin{enumerate}
\def\labelenumi{\arabic{enumi}.}
\item
  Iedereen ter wereld kan meedoen. Tienduizenden mensen kunnen lid
  worden van de grootboekhouderloterij.
\item
  Wanneer we geld willen overmaken vertellen we het gehele netwerk over
  de transacties die we willen uitvoeren, net zoals we hiervoor deden.
\item
  In plaats van iedereen de transacties te laten opschrijven, houden we
  een loterij om te zien wie het recht wint om de transacties in het
  grootboek te schrijven.
\item
  Wanneer we een winnaar selecteren, mag die persoon alle transacties in
  het grootboek schrijven die bij hem bekend zijn.
\item
  Als de persoon \emph{geldige} transacties in het grootboek schrijft,
  die voldoen aan de regels zoals deze door alle deelnemers zijn
  opgelegd, krijgen ze een vergoeding.
\item
  Iedereen houdt een kopie van het grootboek bij en voegt de transacties
  toe die door de laatste loterijwinnaar zijn opgeschreven.
\item
  We wachten een tijdje zodat de meeste mensen tijd hebben om hun
  grootboek bij te werken, om vervolgens de loterij opnieuw uit te
  voeren.
\end{enumerate}

Dit systeem is een verbetering. Het is onmogelijk om te weten wie de
deelnemers zijn en wie de volgende winnaar zal zijn, waardoor deze
methode ongevoelig is voor afpersing.

We hebben echter nog geen duidelijk antwoord op de vraag hoe we deze
loterij kunnen uitvoeren zonder dat iemand de leiding heeft, of waarom
we erop zouden vertrouwen dat de winnaar eerlijk gaat handelen bij het
schrijven in het grootboek. In het volgende hoofdstuk zien we hoe we dat
kunnen oplossen.

\hypertarget{proof-of-work}{%
\chapter{Proof-of-work}\label{proof-of-work}}

Het loterij-systeem uit het vorige hoofdstuk, heeft twee grote
problemen:

\begin{enumerate}
\def\labelenumi{\arabic{enumi}.}
\item
  Als we er van uitgaan dat dat we geen enkele centrale partij kunnen
  vertrouwen, wie verkoopt dan de loten van de loterij en wie kiest de
  winnende loten?
\item
  Hoe zorgen we ervoor dat de winnaar van de loterij de rest niet
  bedriegt en alleen geldige transacties in het grootboek opneemt?
\end{enumerate}

Als we een systeem willen waar iedereen \emph{zonder toestemming} lid
van kan worden, dan moeten we de eis dat iets betrouwbaar moet zijn uit
het systeem halen; het systeem moet \emph{trustless} zijn.\footnote{Nvdr.
  vrij vertaald: \emph{zonder vertrouwen}. \emph{Trustless} houdt in dit
  geval in dat de gebruiker het systeem niet hoeft te vertrouwen, maar
  alles zelf objectief kan verifiëren.}

We moeten tijdens het bedenken van ons systeem rekening houden met de
volgende punten:

\begin{enumerate}
\def\labelenumi{\arabic{enumi}.}
\item
  Bij gecentraliseerde loterijen zoals de Staatsloterij is één partij
  verantwoordelijk voor het genereren van alle loten. In ons systeem
  kunnen we centraal gezag niet vertrouwen, en moet iedereen dus zijn
  eigen loten kunnen genereren.
\item
  We moeten voorkomen dat iemand de loterij volledig in handen krijgt
  door een enorm aantal loten te genereren. De loten kunnen dus niet
  gratis zijn. Hoe zorgen we ervoor dat je daadwerkelijk geld moet
  uitgeven om kaartjes te kopen als er niemand is bij wie je ze kunt
  kopen? De loten moeten van het universum gekocht worden: je moet
  elektriciteit verbruiken om ze te genereren.
\item
  Het moet voor alle andere deelnemers gemakkelijk zijn om te verifiëren
  dat je de loterij gewonnen hebt door alleen je lotnummer te
  controleren. Bij de Staatsloterij bepaalt de trekkingsmachine van de
  Nederlandse Loterij wat het winnende lot is. In een decentraal systeem
  kan dat niet. In plaats daarvan laten we iedereen van tevoren
  overeenstemmen over een getallenreeks. Valt je lotnummer binnen het
  vooraf bepaalde bereik, dan win je. We gebruiken een cryptografische
  truc om dit te doen met behulp van een \emph{hash-functie}.
\end{enumerate}

\hypertarget{een-energie-intensieve-asymmetrische-puzzel}{%
\section{Een energie-intensieve asymmetrische
puzzel}\label{een-energie-intensieve-asymmetrische-puzzel}}

De elegante oplossing voor alle drie deze problemen heet
\emph{proof-of-work}.\footnote{Proof-of-work betekent letterlijk
  \emph{bewijs van uitgevoerd werk/gedane arbeid}.} Dit onderdeel van
ons systeem werd al in 1993, lang voor bitcoin, uitgevonden. Dit is
waarschijnlijk het moeilijkst te begrijpen onderdeel van onze loterij,
dus we zullen hier in de komende hoofdstukken uitgebreid op
ingaan.\footnote{\url{https://en.wikipedia.org/wiki/Proof-of-work_system}}

Zoals we hierboven (in punt 2) concludeerden, moet het duur zijn om de
loten te genereren. Anders kan iedereen zomaar een onbeperkt aantal
loten in handen krijgen. Wat is gegarandeerd duur en is niet afkomstig
van een centrale autoriteit?

Dit is waar de natuurkundige kant van bitcoin de hoek om komt kijken: de
eerste wet van de thermodynamica stelt dat energie niet kan worden
gecreëerd of vernietigd. Een \emph{gratis lunch} bestaat niet als het op
energie aankomt. Elektriciteit is altijd duur omdat je het moet kopen
van de stroomproducenten, of je eigen energiecentrale moet bouwen. In
beide gevallen is het verkrijgen van elektriciteit kostbaar.

Het concept achter proof-of-work is dat je deelneemt aan een willekeurig
proces, vergelijkbaar met het rollen van een dobbelsteen. Maar in plaats
van de gekende zes zijden, heeft onze dobbelsteen ongeveer evenveel
zijden als atomen in het universum. Om deze dobbelsteen te rollen, en
dus lotnummers te genereren, moet je computer berekeningen uitvoeren die
elektriciteit kosten.

Om de loterij te winnen, moet je een getal vinden dat wiskundig is
afgeleid van de transacties die je in het grootboek wilt schrijven, plus
het getal dat je op de dobbelsteen hebt gerold. Om dit winnende getal te
vinden, moet je misschien wel miljarden, triljoenen of quadriljoenen
keren met de dobbelsteen rollen, waarbij je duizenden dollars aan
energie verbruikt. Omdat dit proces op basis van willekeur geschied, is
het voor iedereen mogelijk om zijn eigen loten te genereren, zonder
centrale autoriteit. Hiervoor heb je slechts de lijst met transacties
nodig die je naar het grootboek wilt schrijven en een computer die een
willekeurig getal genereert.

Ook al heeft het vinden van een winnend getal misschien duizenden
dollars aan verbrande energie gekost, toch hoeven andere mensen op het
netwerk slechts een paar simpele checks uit te voeren om jouw werk te
controleren:

\begin{enumerate}
\def\labelenumi{\arabic{enumi}.}
\item
  Is het getal dat je hebt opgegeven minder dan het bereik dat van
  tevoren is afgestemd?
\item
  Is het getal inderdaad wiskundig afgeleid van een geldige verzameling
  van transacties die je naar het grootboek wilt schrijven?
\item
  Voldoen de transacties die gepresenteerd worden aan de regels van
  bitcoin (zijn er geen dubbele uitgaven, worden er geen nieuwe bitcoin
  gegenereerd buiten het toegestane schema, etc.)?
\end{enumerate}

Het proces van proof-of-work berust op toeval en vereist vele
computerhandelingen om een winnend lot te vinden. Het heeft echter maar
één handeling nodig om het te verifiëren. Je kunt het zien als een
kruiswoordpuzzel of een sudoku. Het kost je misschien uren om op te
lossen, maar als je de opgeloste puzzel aan iemand die de regels kent
geeft, kan hij in een oogopslag zien of jouw oplossing correct is. Dit
maakt het systeem \emph{asymmetrisch}: het is moeilijk voor de mensen
die meespelen, maar heel makkelijk voor de mensen die de uitkomst
controleren.

Omdat je een aanzienlijke hoeveelheid energie (en dus geld) verbrandt
bij het spelen van deze loterij, wil je dat iedereen jouw winnende lot
accepteert zodat jij de prijs krijgt. Je wordt dus gestimuleerd om
alleen transacties die voldoen aan de regels toe te voegen aan het
grootboek.

Als je bijvoorbeeld geld probeert uit te geven dat al eerder is
uitgegeven, dan wordt je winnende lot door iedereen afgewezen en is alle
energie die je hebt verbrand om je lot te kopen voor niets geweest. Aan
de andere kant, wie zich aan de regels houdt en alleen geldige
transacties aan het grootboek toevoegt, wordt beloond met bitcoin om de
energierekening te betalen en hopelijk ook nog een beetje winst over te
houden.

Het proof-of-work systeem heeft de belangrijke eigenschap dat het kosten
heeft die in de echte wereld verankerd zijn. Als je het netwerk aan zou
willen vallen door sommige deelnemers te bedreigen, zou je niet alleen
hun computers moeten hacken of overnemen, maar ook hun
elektriciteitsrekening moeten betalen.

Maar hoe kunnen deelnemers bewijzen dat ze deze energie daadwerkelijk
hebben verbrand? Hiervoor hebben we een spoedcursus
computerwetenschappen nodig over twee belangrijke concepten:
\emph{hashing} en \emph{bits}.

\hypertarget{hashing}{%
\section{Hashing}\label{hashing}}

De asymmetrische proof-of-work-puzzel van bitcoin omvat het gebruik van
een \emph{hash-functie}.\footnote{\url{https://en.wikipedia.org/wiki/Hash_function}}
Een functie is een wiskundige bewerking waarin je een invoer (x) hebt en
je hiervoor een \emph{uitvoerwaarde} \(f(x)\) krijgt. De functie
\(f(x)=2x\) neemt bijvoorbeeld de waarde x en vermenigvuldigt deze met
twee. Dus de invoer \(x=2\) geeft ons de uitvoerwaarde \(f(x)=4\).

Een hash-functie is een speciale functie. Een invoer van deze functie
kan iedere willekeurige reeks gegevens zijn, en de uitvoer is een getal
dat er willekeurig uitziet:

\begin{verbatim}
    66ef3d9a8035fa324e813fdc368ac175
    2e329a1cb663cd1559c747d549983bf8
\end{verbatim}

Bovenstaande uitvoer is het resultaat van een specifieke hash-functie
met de invoer Hallo Wereld. De specifieke hash-functie die hiervoor
gebruikt werd, is sha256. Niet geheel toevallig dezelfde hash-functie
die bitcoin ook gebruikt.\footnote{\url{https://en.wikipedia.org/wiki/SHA-2}}

\includegraphics{./images/fig4.png}\{\#fig-fig4)\}

De sha256 hash-functie heeft de volgende eigenschappen die nuttig zijn
voor ons:

\begin{enumerate}
\def\labelenumi{\arabic{enumi}.}
\item
  De output is deterministisch. Dat wil zeggen dat je bij dezelfde
  invoer altijd dezelfde uitvoer krijgt.
\item
  De uitvoer is onvoorspelbaar. Indien slechts één letter van de invoer
  veranderd wordt, dan is de output \emph{volledig} anders, zonder enige
  correlatie met de oude invoer.
\item
  De uitvoer-hash is snel te berekenen, onafhankelijk van de grootte van
  de invoer.
\item
  Het is praktisch onmogelijk twee verschillende invoerwaardes te vinden
  die dezelfde uitvoer hebben.
\item
  De sha256 functie is een éénrichtingsfunctie. Het is onmogelijk om de
  invoer te herleiden uit de uitvoer.
\item
  De uitvoer is altijd een specifieke grootte (256 \emph{bits} voor
  sha256).
\end{enumerate}

\hypertarget{een-korte-uitleg-over-bits}{%
\section{Een korte uitleg over bits}\label{een-korte-uitleg-over-bits}}

Het getallensysteem waar je bekend mee bent, bestaande uit de getallen 0
tot en met 9 wordt \emph{decimaal} genoemd omdat het tien cijfers heeft.
Computers geven de voorkeur aan een ander getallenstelsel: een systeem
gemaakt van enen en nullen, die respectievelijk de aan- of afwezigheid
van een elektrisch signaal aangeven. Dit getallensysteem wordt
\emph{binair} genoemd.

In het decimale stelsel gebruik je slechts de cijfers \(0\) tot en met
\(9\). Als je slechts één cijfer gebruikt, kun je tien verschillende
getallen vertegenwoordigen, 0 tot en met 9. Als je twee cijfers
gebruikt, kun je \(10 \times 10 = 100\) verschillende getallen
voorstellen: \(00, 01,...\) tot en met \(99\). Voor drie cijfers kun je
\(10 \times 10 \times 10 = 1000\) getallen hebben: \(000, 001,...\) tot
en met \(999\).

Hopelijk begin je hier een patroon in te zien. Om erachter te komen hoe
groot het getal is dat we kunnen voorstellen met N cijfers,
vermenigvuldigen we tien, \(N\) keer met zichzelf, oftewel \(10^N\)
(\(10\) tot de macht van \(N\)).

Het binaire stelsel werkt op dezelfde manier. Het enige dat verandert is
het aantal cijfers die beschikbaar zijn. Terwijl we gewend zijn aan het
decimaal stelsel met tien cijfers, kan een \emph{binair cijfer} of
\emph{bit} slechts twee waarden hebben: nul en één.

Als een \emph{bit} 2 waarden kan vertegenwoordigen, dan kunnen twee
\emph{bits} 4 waarden vertegenwoordigen: \(00, 01, 10, 11\). Je kunt dit
berekenen door \(2 \times 2\) te vermenigvuldigen, aangezien elk cijfer
twee waarden kan hebben. Drie bits kunnen
\(2 \times 2 \times 2 = 2^3 = 8\) waarden vertegenwoordigen:
\(000, 001, 010, 011, 100, 101, 110, 111\).

Een \emph{binair} getal dat N \emph{bits} lang is, kan dus \(2^N\)
verschillende waarden vertegenwoordigen.

Daarom is het aantal unieke waarden die je kunt vertegenwoordigen met
256 bits, de grootte van de sha256 hashing functie, \(2^{256}\). Dat is
een gigantisch, bijna onvoorstelbaar groot aantal. Weergegeven in
decimaal, is getal dit 78 cijfers lang. Ter vergelijking is dit ongeveer
dezelfde ordegrootte als het geschatte aantal atomen in het bekende
universum.

\(2^{256}\) = 115 792 089 237 316 195 423 570 985 008 687 907 853 269
984 665 640 564 039 457 584 007 913 129 639 936

Bovenstaande getal is het aantal mogelijke resultaten van een sha256
hash-functie. Het is dus zo goed als onmogelijk om te voorspellen wat
het getal zal zijn dat door deze functie wordt geproduceerd. Het zou
hetzelfde zijn als het perfect voorspellen van de uitkomst van 256
achtereenvolgende muntworpen, of het raden van de locatie van één
willekeurig uitgekozen atoom, ergens in het universum.

Dit getal is uiteraard te lang om te blijven uitschrijven, dus we houden
het vanaf nu gewoon op \(2^{256}\), maar ik hoop dat dit het
ongelofelijke aantal mogelijkheden duidelijk heeft gemaakt.

\hypertarget{laten-we-een-aantal-teksten-hashen}{%
\section{Laten we een aantal teksten
hashen}\label{laten-we-een-aantal-teksten-hashen}}

Hier zijn enkele voorbeeldteksten en hun sha256 hashes. De uitvoer wordt
weergegeven in decimale notatie, maar binnenin een computer zou dit de
bekende reeks van enen en nullen zijn.

Het punt is om te tonen hoe drastisch de uitvoer verandert op basis van
één kleine wijziging in de invoer, en om te laten zien dat je niet kunt
voorspellen welke uitvoer geproduceerd wordt door de hash-functie op
basis van wat je erin stopt:

\begin{verbatim}
    “Hello World!”
    869913660443924676617831651669733090238
    07181648024718778313526389892860994842
   
    “Hello World!!”
    849402277206958989554476271088404243643
    90283616735576803008868844073193772558
\end{verbatim}

Op geen enkele manier kan iemand op basis van deze getallen zien of
berekenen wat de invoer is geweest, zelfs geen computer. Als je zelf met
sha256 wilt spelen, kun je het uitproberen op
\url{https://passwordsgenerator.net/sha256-hash-generator}.

\hypertarget{hashen-om-de-proof-of-work-loterij-te-winnen}{%
\section{Hashen om de proof-of-work loterij te
winnen}\label{hashen-om-de-proof-of-work-loterij-te-winnen}}

Nu zijn we klaar om te praten over het belangrijkste stukje magie. We
zeiden dat er \(2^{256}\) totale mogelijke sha256-uitvoerwaarden zijn.
Laten we in dit voorbeeld voor het gemak even doen alsof er slechts 1000
verschillende uitkomsten zijn.

Het loterijsysteem werkt als volgt:

\begin{enumerate}
\def\labelenumi{\arabic{enumi}.}
\item
  Alice kondigt aan dat ze 2 dollar naar Bob wil sturen.
\item
  Iedereen die meespeelt, neemt de transactie Alice geeft \$2 aan Bob,
  en voegt hier een willekeurig getal aan toe, wat we een \emph{nonce}
  noemen.\footnote{Staat voor number used only once} Hierdoor zal de
  input van hun sha256 hash-functie anders zijn dan de input van die van
  anderen, wat helpt om een winnend getal te vinden.
\item
  Als dat getal kleiner is dan het \emph{doelnummer} (dit bespreken we
  verder in het volgende hoofdstuk), winnen ze de loterij.
\item
  Als het getal dat ze krijgen groter is dan het doelnummer, dan hashen
  ze opnieuw, maar voegen dit keer een andere nonce toe: \emph{Alice
  geeft \$2 aan Bob nonce=12345}, dan \emph{Alice geeft \$2 aan Bob
  nonce=92435}, dan \emph{Alice geeft \$2 aan Bob
  nonce=132849012348092134}, enzovoort. Ze doen dit net zolang tot
  iemand een hash heeft gevonden die kleiner is dan het doelnummer.
\end{enumerate}

Het kan vele, vele pogingen kosten om een hash te vinden die kleiner is
dan het doelnummer. We kunnen in feite bepalen hoe vaak iemand de
loterij kan winnen door de kans dat ze een winnend getal vinden te
manipuleren. Als er 1000 mogelijke hashes zijn, en we stellen het
doelnummer in op 100, welk percentage van hashes zit er dan onder de
doelnummer?

Dat is natuurlijk vrij basale wiskunde; 100 van de 1000 mogelijkheden,
of 100/1000 = 10\% van de hashes zullen minder zijn dan het doelnummer.
Dus als je een stuk tekst hasht en je hash-functie heeft 1000
verschillende uitkomsten, verwacht je dat 10\% van de hashes onder het
doelnummer van 100 uitkomt.

En dit is dus precies hoe onze loterij werkt: we spreken een doelnummer
af, dan nemen we alle transacties die mensen toe willen voegen aan het
grootboek, en hashen ze met een willekeurig getal erbij (de
\emph{nonce}). Zodra iemand een hash vindt die onder het doelnummer
valt, deelt hij dit mee aan het hele netwerk:

Hoi iedereen!

\begin{itemize}
\item
  Ik heb de transacties Alice stuurt \$2 naar Bob en Charlotte stuurt
  \$5 naar Alice genomen.
\item
  Ik heb hier de nonce 32895 aan toegevoegd.
\item
  De hash hiervan kwam uit op 42, wat minder is dan het afgesproken
  doelnummer van 100.
\item
  Hier is mijn proof-of-work: de transactiegegevens, de nonce die ik heb
  gebruikt, en de hash die werd geproduceerd op basis van die inputs.
\end{itemize}

Het heeft mij misschien miljarden hash-pogingen en duizenden dollars aan
energie gekost om deze hash te vinden, maar iedereen kan onmiddellijk
valideren dat ik dit werk daadwerkelijk heb gedaan; omdat ik zowel de
invoergegevens (transacties en nonce) als de verwachte uitvoer (het
hash-nummer) aan iedereen heb laten zien, kunnen ze dezelfde hash
uitvoeren om in één poging te valideren of ik ze de juiste gegevens heb
gegeven.

Hoe verhoudt dit zich tot het verbruiken van energie? We zeiden al
eerder dat de set van alle mogelijke hashes eigenlijk een gigantisch
getal is, dat ongeveer net zo groot is als het aantal atomen in het
universum. We kunnen het doel instellen op een laag genoeg getal zodat
slechts een heel klein deel van de hashes geldig is. Dit betekent dat
iedereen die een geldige hash wil vinden, een enorme hoeveelheid
rekentijd, en dus elektriciteit, zal moeten verbruiken om een hash te
vinden die kleiner is dan ons doelnummer.

Hoe lager het doelnummer, hoe meer pogingen het kost om een geldig getal
te vinden, en hoe hoger het doelnummer, hoe sneller we een winnende hash
kunnen vinden. Als onze kans om het juiste getal te vinden één op een
miljoen is, dan bewijzen we door dit getal te vinden dat we ongeveer een
miljoen berekeningen hebben uitgevoerd.

\begin{figure}

{\centering \includegraphics{./images/fig5.png}

}

\caption{\label{fig-fig5}We kunnen hashing zien als het rollen van een
gigantische dobbelsteen op basis van specifieke invoergegevens, met een
aantal zijden gelijk aan het aantal atomen in het universum. Alleen die
invoergegevens die ervoor zorgen dat je onder het doelnummer rolt,
winnen de loterij. Om de loterij te winnen, moet je aan de rest van het
netwerk laten zien welke gegevens je hebt gebruikt om tot de uitkomst te
komen.}

\end{figure}

\hypertarget{mining}{%
\chapter{Mining}\label{mining}}

Meedoen aan de proof-of-work loterij om de mogelijkheid tot het
aanpassen van het bitcoin-grootboek te winnen, is beter bekend als
\emph{minen}. Dit is hoe het werkt:

\begin{enumerate}
\def\labelenumi{\arabic{enumi}.}
\item
  Iedereen die wil deelnemen, sluit zich aan bij het bitcoinnetwerk door
  hun computer aan te zetten, de juiste bitcoin-software te draaien en
  te luisteren naar anderen die hun transacties aankondigen.
\item
  Alice kondigt haar voornemen aan om wat munten naar Bob te sturen. De
  computers op het netwerk roddelen met elkaar om deze transactie naar
  iedereen op het netwerk te verspreiden.
\item
  Alle computers die willen deelnemen aan de loterij starten met het
  \emph{hashen} van de transacties waar ze over hebben gehoord door
  willekeurige \emph{nonces} toe te voegen aan de transactielijst en de
  sha256 \emph{hash}-functie uit te voeren.
\item
  Gemiddeld elke tien minuten vindt een computer een hash afgeleid van
  de transacties die lager is dan het huidige doelnummer en wint daarmee
  de loterij.
\item
  Deze computer maakt zijn winnende getal bekend, evenals de invoer
  (transacties en nonce) die ze hebben gebruikt om het winnende lot te
  produceren. Het kan uren gekost hebben om dat te vinden, of een paar
  minuten. Deze informatie bij elkaar (transacties, nonce en de hash van
  het proof-of-work) wordt een \emph{blok} genoemd.
\item
  Alle andere deelnemers valideren het blok door te controleren of de
  transacties in het blok samen met de nonce inderdaad hashen tot wat er
  gedeeld wordt, dat de hash inderdaad lager is dan het doelnummer, dat
  het blok geen ongeldige transacties bevat én dat de geschiedenis in
  het blok niet in strijd is met voorgaande blokken.
\item
  Iedereen schrijft het blok in zijn kopie van het grootboek en voegt
  het toe aan de bestaande ketting van blokken, bekend als een
  \emph{blockchain}.
\end{enumerate}

Dat is het in een notendop. We hebben ons eerste blok geproduceerd en
mogen onze eerste transacties aan onze grootboeken toevoegen.

Misschien heb je in de media de vaak herhaalde bewering gelezen dat
bitcoin mining het oplossen van ingewikkelde vergelijkingen inhoudt. Nu
zul je begrijpen dat dit volstrekt onjuist is; in plaats van
vergelijkingen op te lossen, moet je bij bitcoin mining herhaaldelijk
gooien met een grote virtuele dobbelsteen om een hash te produceren die
onder een bepaald doelnummer valt. Het is gewoon een kansspel dat
deelnemers dwingt om een bepaalde hoeveelheid elektriciteit te
verbruiken.

\hypertarget{hoe-worden-nieuwe-bitcoins-gemaakt}{%
\section{Hoe worden nieuwe bitcoins
gemaakt?}\label{hoe-worden-nieuwe-bitcoins-gemaakt}}

Tot nu toe hebben we besproken hoe Alice \$2 naar Bob kan sturen. We
gaan vanaf nu stoppen met praten over dollars, want bitcoin kent geen
dollars. Wat we wel hebben, zijn bitcoins zelf: digitale eenheden die
waarde vertegenwoordigen op het bitcoinnetwerk.

Om bij ons voorbeeld te blijven, kunnen we zeggen dat Alice 2 bitcoins
naar Bob stuurt door aan te kondigen dat ze haar munten, die op haar
rekening staan, naar die van Bob overzet. Vervolgens wint iemand de
proof-of-work loterij, en wordt haar aangekondigde transactie toegevoegd
aan het grootboek.

Maar waar kwamen die 2 bitcoins van Alice oorspronkelijk vandaan? Hoe
ging bitcoin van start en hoe verwierf iemand ooit munten voordat er
marktplaatsen waren om ze te kopen met traditionele fiatvaluta zoals
dollars of euro's?

Toen Satoshi bitcoin ontwierp, had hij ervoor kunnen kiezen om een
database te maken die alle 21 miljoen munten aan hem toewees en
vervolgens iedereen te vragen ze van hem te kopen. Echter zou er weinig
reden zijn voor anderen om waarde toe te kennen aan een systeem waarin
één enkeling alle waarde bezat. Hij kon een register gemaakt hebben waar
mensen konden intekenen met een e-mailadres om kans te maken op enkele
munten. Maar een registratiesysteem zou vatbaar zijn aan een zogenaamde
\emph{Sybil-aanval}, aangezien het quasi gratis is om miljoenen
e-mailadressen aan te maken.\footnote{Een Sybil-aanval is een aanval
  waarin een deelnemer aan een bepaald netwerk een groot aantal
  pseudonieme addressen aanmaakt om een grote hoeveelheid invloed in het
  netwerk te vergaren; vernoemd naar Sybil, een boek over een vrouw met
  een dissociatieve identiteitsstoornis}

Satoshi koos uiteindelijk om de nieuwe munten te laten genereren door
het proces van minen, oftewel de deelname aan de proof-of-work loterij
om het recht te verkrijgen in het grootboek te schrijven. Wanneer je een
grote hoeveelheid energie gebruikt en het winnende getal van het
volgende geldige blok vindt, krijg je het recht om de transacties die
bij jou bekend zijn toe te voegen aan het grootboek. Daarbovenop mag je
ook een speciale transactie toevoegen aan het blok die we een
\emph{coinbase}-transactie noemen. Zo'n transactie zegt: 12,5 bitcoin
werden gemaakt en toegekend aan Marie de Miner om haar te compenseren
voor de energie die gebruikt werd om het blok te vinden.

Dit is hoe nieuwe bitcoins gemaakt worden. Met dit proces is iedereen
ter wereld in staat om hun eigen munten te vergaren zonder tussenkomst
van een centrale autoriteit én zonder hun identiteit bekend te moeten
maken. De enige vereiste is dat ze genoeg willen betalen voor de
elektriciteit die nodig is om mee te doen met de loterij. Dit maakt
bitcoin's uitgifte resistent tegen \emph{sybil}-aanvallen. Wie munten
wil, zal energie moeten verbruiken en iets moeten betalen om ze te
maken.

\hypertarget{de-blokbeloning}{%
\section{De blokbeloning}\label{de-blokbeloning}}

De persoon die de loterij wint, mag zichzelf enkele vers gemaakte munten
toekennen. Maar waarom is de beloning 12,5 bitcoins en niet 1000? Waarom
kan niemand valsspelen en zichzelf een willekeurig bedrag toekennen?

Bitcoin is een systeem van gedistribueerde consensus. Dit betekent dat
iedereen akkoord moet gaan met wat geldig is en wat niet. Dat kan door
software op je computer te draaien die een bekende set van regels
afdwingt. Deze regels staan bekend als de consensusregels van bitcoin.
Wanneer een miner een nieuw blok vindt, wordt alles tegen de
consensusregels afgewogen. Wanneer het blok geldig blijkt te zijn,
schrijft iedereen het in hun grootboek en wordt het geaccepteerd als
waarheid. Indien niet, dan wordt het blok verworpen.

Hoewel de volledige lijst van consensusregels vrij complex is, kunnen we
de belangrijkste hieronder opsommen:

\begin{itemize}
\item
  Een geldig blok mag een specifieke hoeveelheid nieuwe munten aan het
  netwerk toevoegen, zolang het voldoet aan de regels van het
  uitgifteschema wat vastgelegd is in de software.
\item
  Transacties moeten geldige handtekeningen hebben die aangeven dat de
  eigenaar het verzenden daadwerkelijk goedgekeurd heeft.
\item
  Transacties die munten proberen uitgeven die al eerder uitgegeven
  werden, kunnen onder geen geval plaatsvinden.
\item
  De hoeveelheid data in het blok mag een specifieke limiet niet
  overschrijden.
\item
  De proof-of-work hash van het blok moet kleiner zijn dan het huidige
  doelnummer, waardoor onbetwistbaar aangetoond wordt dat dit blok tot
  stand kwam door een bepaalde hoeveelheid elektriciteit te gebruiken.
\end{itemize}

Indien Marie een blok vindt en beslist om zichzelf een beetje extra te
geven, dan zullen de computers van de overige deelnemers dit blok
afwijzen en als ongeldig beschouwen. Dit komt omdat de bitcoin-software
die iedereen draait een stukje code bevat dat zegt: ``de huidige
blokbeloning is precies 12,5 bitcoins. Als je een blok ziet waarin
iemand meer dat dit bedrag toekent, verwerp het dan.''

Als Marie probeert vals te spelen en een ongeldig blok produceert, dan
komt het blok in niemands grootboek terecht en zal ze simpelweg veel
elektriciteit verspild hebben om een vervalsing te produceren die
niemand wil. Dit geeft bitcoin een \emph{onvervalsbare kostbaarheid},
een term die werd geïntroduceerd door Nick Szabo in zijn esssay Shelling
Out. Het is vrij makkelijk te begrijpen dat geld wat makkelijk te
vervalsen is, nooit daadwerkelijk nuttig kan zijn als geld. Het is
onmogelijk om bitcoin te vervalsen omdat iedere munt op echtheid te
controleren valt met een simpele, wiskundige test.

Satoshi ontgon het eerste genesis blok, wat de eerste bitcoins ooit
produceerde. De code is \emph{open source} en dus kan iedereen een
kijkje onder de motorkap nemen om te zien of er niets verkeerd loopt.
Maar zelfs Satoshi moest miljarden berekeningen doen en meespelen met de
proof-of-work loterij om die eerste blokken te vinden. Op geen enkele
manier kon hij een vervalsing produceren door te liegen over de energie
die hij nodig had om een winnende oplossing te vinden. En dat terwijl
hij de ontwerper van het systeem was.

Iedereen die zich daarna bij het netwerk aansloot kon controleren dat
zijn gevonden oplossingen voldeden aan de vereisten. Door ze naast het
initiële doelnummer en transactiedata te leggen, blijkt overduidelijk
dat een bepaalde hoeveelheid energie gebruikt werd om een statistisch
zeldzaam doelnummer te vinden. Stel je voor dat het mogelijk is om met
diezelfde precisie en regelmaat te controleren hoe fiatvaluta in het
traditionele bankensysteem tot stand komt.

\hypertarget{de-halvering}{%
\section{De halvering}\label{de-halvering}}

Het proces van minen produceert nieuwe bitcoins. Maar Satoshi wou een
systeem waarin het onmogelijk was om de waarde van de munt te
devalueren. Hij wou niet dat het monetaire aanbod tot in het oneindige
zou blijven stijgen. In plaats daarvan, ontwierp hij een uitgifteschema
dat snel begon en geleidelijk afneemt tot uiteindelijk geen nieuwe
munten meer zullen worden geproduceerd.

In het begin was de beloning voor een blok 50 bitcoin en dat is hoeveel
Satoshi ontving voor zijn allereerste ontgonnen blok. Net zozeer kregen
alle andere deelnemers die zich aansloten bij het netwerk in de
begindagen voor elke gevonden blok 50 bitcoin.

De bitcoin-software dwingt iedere vier jaar een halvering van de
blokbeloning af. De periode is gebaseerd op het aantal blokken in plaats
van een bepaalde periode van tijd, maar omdat blokken ongeveer elke tien
minuten worden geproduceerd, maakt dit in werkelijkheid weinig verschil.
In 2008 was de blokbeloning 50 BTC, in 2012 was het 25 BTC, in 2016
12,5. Vandaag, op 8 juni 2019, zijn al 579.856 blokken ontgonnen sinds
het begin van bitcoin en is de beloning 12,5 BTC per blok.

Binnen 50.144 blokken, ongeveer in mei 2020, zal de beloning verlaagd
worden tot 6,25 BTC per blok. De jaarlijkse toename van het aanbod wordt
dan ongeveer 1,8\%. Na nog eens 12 jaar, 3 halveringen later, zullen
99\% van alle bitcoins in circulatie zijn en krijgen miners nog minder
dan 1 bitcoin per gevonden blok. Je kan de voortgang van de halveringen
volgen op \url{https://bitcoinblockhalf.com}.

{Het gecontroleerde monetaire tijdschema van bitcoin kan door iedereen
worden bevestigd.}

Uiteindelijk, rond het jaar 2140, zal de blokbeloning volledig wegvallen
en moeten miners genoegen nemen met transactievergoedingen van de
gebruikers als compensatie voor hun werk.

De blokbeloningen, en het uitgifteschema in het algemeen, worden
afgedwongen in de bitcoin-software die, zoals eerder gesteld, volledig
\emph{open source} is en dus door iedereen volledig gevalideerd kan
worden. Een blok produceren dat niet voldoet aan de regels zal door
niemand met dezelfde software geaccepteerd worden.

\hypertarget{beheersen-van-de-uitgifte-en-het-blokinterval}{%
\section{Beheersen van de uitgifte en het
blokinterval}\label{beheersen-van-de-uitgifte-en-het-blokinterval}}

Het minen van bitcoin vereist computers en elektriciteit. Dus hoe meer
computers en elektriciteit je ter beschikking hebt, hoe groter de kans
is dat jij een winnend lot weet te vinden. Stel, het netwerk bestaat uit
100 computers met gelijke rekenkracht en jij bezit daar 10 van. Dit
betekent dat je in ongeveer 10\% van de gevallen een winnende oplossing
zal vinden. Maar mining is een proces waar geluk en willekeur mee
gemoeid gaat. Het is dus theoretisch gezien mogelijk dat er uren of
dagen voorbijgaan zonder dat je een winnend lot vind.

In het vorige deel hebben we uitgelegd dat miners zichzelf niet
simpelweg een arbitraire blokbeloning kunnen toekennen. Die zouden door
andere \emph{nodes} verworpen worden. Maar wat als ze een heleboel
energie gebruikten om het miningproces sneller te laten verlopen en zo
meer bitcoins te verwerven? Dit zou in strijd zijn met het opzet in het
ontwerp van bitcoin, namelijk dat het uitgifteschema van tevoren bekend
zou moeten zijn.

Laten we opnieuw het eerder gegeven voorbeeld nemen: er zijn 1000
mogelijke hashes en ons doelnummer is 100. In 10\% van de gevallen
zullen we een getal vinden dat kleiner is dan 100 en een geldig blok
vinden.

Stel dat het 1 seconde duurt om elke hash te berekenen. Als we iedere
seconde ``onze dobbelsteen rollen'' door de huidige transacties en onze
lukrake \emph{nonce} te hashen, vinden we in 10\% van de keren een getal
dat kleiner is dan het doelnummer. We verwachten dus dat het ons
gemiddeld 10 seconden vergt om een geldige hash te vinden.

Wat gebeurt er wanneer 2 computers meedoen aan de loterij? Ze hashen dan
dubbel zo snel en verwachten een geldige hash iedere 5 seconden. Wat als
10 computers meedoen? Elk van hen kan ongeveer iedere seconde een
winnende oplossing vinden.

Het probleem is dus het volgende: als meer mensen meedoen, worden
blokken te snel geproduceerd. Dit leidt tot twee uitkomsten die we niet
willen:

\begin{enumerate}
\def\labelenumi{\arabic{enumi}.}
\item
  Het is niet meer mogelijk om een vooraf vastgesteld uitgiftescheme aan
  te houden. We willen hiervoor ieder uur een relatief consistent aantal
  bitcoins in circulatie zien komen om te zorgen dat het proces tot in
  het jaar 2140 duurt, en niet eerder ten einde komt.
\item
  Het creëert problemen op het netwerk: wanneer blokken te snel gevonden
  worden en geen tijd hebben om de rest van het netwerk te bereiken
  voordat een volgend blok gevonden wordt, kunnen we niet tot consensus
  komen over de lineaire geschiedenis van transacties. Verschillende
  miners zouden dan transacties in hun blokken gestopt kunnen hebben,
  die al uitgegeven werden in een vorig blok waar ze nog niet van op de
  hoogte waren.
\end{enumerate}

Als minder mensen minen dan hebben we het tegenovergestelde:

\begin{enumerate}
\def\labelenumi{\arabic{enumi}.}
\item
  Nieuwe bitcoins worden te traag uitgegeven waardoor het uitgifteschema
  niet klopt.
\item
  Het systeem wordt onbruikbaar omdat je soms uren, dagen of langer zal
  moeten wachten om je transacties in het grootboek te krijgen.
\end{enumerate}

Het totale aantal hashes per seconde van alle miners op het netwerk
noemen we de \emph{hash-rate}.

{De tijd tussen blokken varieert afhankelijk van de hash-rate die komt
en gaat, evenals willekeurige kans.}

\hypertarget{moeilijkheidsaanpassing-consensus-over-het-doelnummer}{%
\section{Moeilijkheidsaanpassing: consensus over het
doelnummer}\label{moeilijkheidsaanpassing-consensus-over-het-doelnummer}}

Aangezien bitcoin een vrijwillig en permissieloos systeem is waar mensen
deel aan kunnen nemen zoals zij willen, zonder dat iemand de leiding
heeft, zal het aantal miners op het netwerk sterk schommelen. We hebben
een manier nodig om de productie van blokken stabiel te houden in plaats
van telkens een versnelling of vertraging in productie te zien wanneer
miners zich aansluiten of verdwijnen.

Hoe kunnen we het moeilijker maken om geldige hashes te vinden als meer
spelers meedoen aan de loterij en makkelijker wanneer spelers weggaan?
Op die manier zouden we de uitgifte en blokintervallen stabiel kunnen
houden.

Herinner je dat het minen van bitcoin een loterij is waarin je een
willekeurig getal moet vinden dat kleiner is dan het doelnummer.

{We proberen deze kleine ruimte te raken. Het aantal mogelijke
uitkomsten is extreem groot, dus het zal heel lang duren om daar te
komen door willekeurige worpen van de dobbelsteen.}

Bitcoin lost dit probleem op door middel van een \emph{aanpassing van de
moeilijkheidgraads}. Omdat iedereen dezelfde code draait die dezelfde
regels afdwingt, én iedereen een volledige kopie heeft van de
transactiegeschiedenis, kan iedereen onafhankelijk berekenen hoe snel
blokken op dat moment geproduceerd worden.

Telkens wanneer we 2016 blokken geproduceerd hebben (ongeveer 2
weken)\footnote{De aanpassingsperiode van 2016 blokken is op basis van
  het beoogde blokinterval van 10 minuten; 10 minuten x 2016 blokken =
  twee weken. Het blokinterval werd arbitrair vastgesteld door Satoshi
  om groot genoeg te zijn zodat de meeste nodes kunnen op de hoogte zijn
  van de laatste blok. De periode van twee weken is ook ietwat arbitrair
  gekozen, maar is ontworpen om te voorkomen dat het spel belazerd wordt
  door te snelle achtereenvolgende aanpassingen in de hash-rate.},
analyseren we de snelheid van uitgifte over de afgelopen periode. We
kijken hoeveel tijd nodig was om die 2016 blokken te produceren en
passen vervolgens passen het doelnummer aan om de productie van blokken
te versnellen of te vertragen.

Iedereen neemt de laatste 2016 blokken en deelt ze door de tijd die
gemiddeld nodig was om te ze creëren. Was het gemiddelde meer dan 10
minuten? Dan gaan we te traag. Lag het gemiddelde onder de 10 minuten?
Dan gaan we te snel.

We kunnen nu het doelnummer aanpassen zodat het groter of kleiner is,
proportioneel met hoe traag of snel we gaan, om te zorgen dat we in lijn
blijven met het afgesproken interval van 10 minuten.

Het doelnummer kan een groter getal worden, waardoor een groter bereik
van hashes een geldige oplossing wordt. Hierdoor wordt de kans om een
winnende hash te vinden voor de miners groter en hoeven ze minder
energie te gebruiken om een blok te vinden. Dit heet \emph{de
moeilijkheidsgraad verlagen}.

{Het verhogen van het doel vergroot de geldige ruimte, waardoor het
waarschijnlijker wordt om in minder pogingen een juiste oplossing te
vinden. Daardoor wordt het goedkoper in verbrande energie.}

Anderzijds, kunnen we het doelnummer kleiner maken zodat er minder
hashes geldig zijn en miners meer energie moeten spenderen om een blok
te vinden. Dit heet \emph{de moeilijkheidsgraad verhogen}.

Dit betekent dat we voor elke periode van 2016 blokken precies weten wat
het doelnummer is. Het geeft ons de magische grens waarbinnen een hash
van de proof-of-work moet vallen om een winnend lot te bemachtigen,
binnen die periode.

De aanpassingen van de moeilijkheidsgraad en het doelnummer is misschien
wel de belangrijkste innovatie van bitcoin. Het stelt iedereen in staat
om onafhankelijk de lotnummers te verifiëren, die op hun beurt weer
gebaseerd zijn op een doelnummer dat ook onafhankelijk te verifiëren is.
Hierdoor kunnen we een loterij spelen zonder dat iemand ons de winnende
combinatie hoeft te vertellen.

In Figuur \protect\hyperlink{fig10}{6.5} zie je de \emph{hash-rate}
(lijn) en de moeilijkheidsgraad (staven) doorheen de tijd. De
moeilijkheid wordt iedere 2016 blokken aangepast en lijkt een beetje op
een trap. Telkens wanneer de hash-rate boven de moeilijkheid uitstijgt,
zie je dat de moeilijkheidsgraad stijgt om bij te benen. Wanneer de
hash-rate zakt, zoals tussen oktober en december 2018, zakt ook de
moeilijkheidsgraad. De aanpassing van de moeilijkheidsgraad volgt altijd
wat de hash-rate doet met een vertraging van 2016 blokken (twee weken).

{Hash-snelheid versus moeilijkheid.}

Omdat er een vertraging van 2016 blokken op de aanpassing zit, is het
mogelijk dat grote pieken omhoog of omlaag in hash-rate leiden tot een
over- of onderproductie van bitcoins tijdens dat interval en licht
afgeweken wordt van het uitgifteschema.

Het verhogen van de hash-rate gaat gepaard met de productie van een
grote hoeveelheid nieuwe hardware, waardoor pieken relatief
ongebruikelijk zijn en niet al te veel impact hebben. Het effect van
iedere piek, omhoog of omlaag, zal beperkt blijven tot dat interval van
2016 blokken. Na de volgende aanpassing komen we opnieuw bij een
gemiddelde van 10 minuten per blok.

\hypertarget{hash-rate-en-de-dollarwaarde-van-bitcoin}{%
\section{Hash-rate en de dollarwaarde van
bitcoin}\label{hash-rate-en-de-dollarwaarde-van-bitcoin}}

Bitcoin herberekent de moeilijkheidsgraad automatisch op basis van alle
rekenkracht van de loterijspelers. Dit zijn de miners die energie
verbruiken om te hashen. Op die manier raakt de echte wereld vervlochten
met onze digitale wereld. De prijs van bitcoin, de prijs van hardware en
energie en de moeilijkheid van het doelnummer creëren
feedback-koppelingen:

\begin{enumerate}
\def\labelenumi{\arabic{enumi}.}
\item
  Speculanten kopen bitcoin omdat ze denken dat de prijs hoger gaat,
  waardoor ze de prijs opdrijven tot \$X.
\item
  Miners verbruiken tot \$X aan energie en hardware om bitcoins te
  verdienen.
\item
  Een grote vraag van kopers duwt de prijs omhoog en leidt tot meer
  miners die aardig verdienen aan hun activiteit.
\item
  Meer miners betekent meer hash-rate en meer verbruikte energie om
  bitcoin te produceren. Het netwerk wordt nog sterker beveiligd. De
  kopers worden gerustgesteld door de veiligheid van het netwerk en
  drijven de prijs soms nog verder omhoog.
\item
  Na 2016 blokken gaat de moeilijkheidsgraad omhoog door de aanwezigheid
  van meer hash-rate.
\item
  Een hogere moeilijkheid betekent een lager doelnummer. De miners
  vinden nu minder vaak blokken waardoor sommigen meer dan \$X spenderen
  in kosten om een bitcoin te minen.
\item
  Voor sommige miners blijkt het niet meer interessant om te minen,
  omdat ze meer energie verbruiken dan de winst die hun activiteit
  oplevert wanneer ze hun bitcoin verkopen. Ze leggen hun machines af en
  de hash-rate op het netwerk gaat terug omlaag.
\item
  De volgende 2016 blokken passeren. De moeilijkheid wordt opnieuw
  berekend en aangezien sommige miners offline gingen, wordt het deze
  keer makkelijker. Het doelnummer gaat omhoog.
\item
  De lagere moeilijkheid betekent dat miners, die voorheen onrendabel
  waren, hun machines terug aanzetten of dat nieuwe miners op het toneel
  verschijnen.
\item
  Ga terug naar 1.
\end{enumerate}

In een neerwaartse markt kan de cyclus in de andere richting gaan.
Gebruikers van het netwerk dumpen dan hun munten, de prijs gaat omlaag
en miners worden onrendabel.

Het algoritme voor de aanpassing van de moeilijkheidsgraad zorgt dat er
altijd een evenwicht wordt gevonden tussen de prijs en de hoeveelheid
hash-rate die aanwezig is op het netwerk. Zelfs als de prijs drastisch
zou zakken en de helft van de huidige hash-rate uit het netwerk zou
duwen, dan zou de volgende aanpassing het opnieuw rendabel maken op een
nieuw evenwichtsniveau.

De aard van de moeilijkheidsaanpassing duwt inefficiënte miners eruit
ten voordele van degenen die werken met de goedkoopst mogelijke energie
en met de laagste totale operationele kosten. Na verloop van tijd dwingt
het bitcoin miners richting afgelegen hoeken op de planeet. Ze gebruiken
energiebronnen die onderbenut of volledig onontgonnen zijn. Een rapport
van CoinShares uit 2019 schat dat ongeveer 75\% van de bitcoin miners
hernieuwbare energie gebruikt.\footnote{Lees meer over de laatste stand
  van zaken betreffende mining op
  \url{https://coinshares.com/research/bitcoin-mining-network-june-2019}}

De laatste jaren ging de prijs pijlsnel omhoog, net als de totale
hash-rate. Hoe hoger de hash-rate, hoe moeilijker het is om het netwerk
aan te vallen. Om te bepalen wat in het grootboek komt, moet je ten
minste evenveel energie en hardware beheren als de helft van het hele
netwerk. Op vandaag wordt geschat dat de elektriciteit die gebruikt
wordt door bitcoin miners equivalent is met het verbruik van een land
van gemiddelde grootte.

\hypertarget{vergoedingen-en-het-einde-van-blokbeloning}{%
\section{Vergoedingen en het einde van
blokbeloning}\label{vergoedingen-en-het-einde-van-blokbeloning}}

Hoe zorgen we dat miners nog altijd een drijfveer hebben om energie te
spenderen om het netwerk te beveiligen wanneer de blokbeloning
uiteindelijk afloopt? Het antwoord zijn de vergoedingen voor
transacties. Met tijd vervangen de vergoedingen de beloning en ze geven
de miners ook de drijfveer om transacties daadwerkelijk in een blok te
plaatsen. Anders zouden ze perfect lege blokken kunnen produceren en
enkel de blokbeloning opstrijken.

De vergoedingen worden bepaald op een vrije markt waar gebruikers bieden
voor de schaarse ruimte binnen een blok. Gebruikers die transacties
versturen geven aan hoeveel vergoeding ze willen betalen aan miners. Op
hun beurt kunnen de miners de transactie al dan niet insluiten in het
volgende blok, afhankelijk van de vergoedingen. Wanneer slechts weinig
transacties in het volgende blok willen, zijn de vergoedingen laag en is
er weinig competitie. Wanneer de ruimte in de blokken opgevuld raakt,
gaan gebruikers meer bereid zijn om een hogere vergoeding te betalen om
hun transactie sneller bevestigd te zien (in de volgende blok). Wie niet
wil betalen kan zijn transactievergoeding laag zetten, maar moet dan
rekening houden met een langere wachttijd.

In traditionele financiële systemen zijn vergoedingen doorgaans
gebaseerd op een percentage van het bedrag van de transactie. In bitcoin
is de waarde van de transfer niet relevant voor de vergoeding. In plaats
daarvan, zijn vergoedingen proportioneel met de schaarse grondstof die
ze consumeren: \emph{blokruimte}. De vergoedingen worden gemeten in
satoshi's per byte opgebruikte ruimte in een blok.\footnote{Een byte is
  8 bits.} Op die manier kan het zijn dat een transactie van een miljoen
bitcoin tussen twee personen goedkoper is qua vergoeding dan een
transactie die 1 BTC splits in 10 ontvangers aangezien die laatste meer
ruimte zal innemen in de transactielijst.

Tijdens periodes waarin bitcoin enorm in trek was, zoals de grote
stierenmarkt van 2017, gingen transactievergoedingen door het dak.
Sindsdien zijn een aantal nieuwe functies geïmplementeerd die het
probleem van hoge transactiekosten op het netwerk verlichten.

Een van die aanpassingen is Segregated Witness. Deze aanpassing zorgt
voor een andere voorstelling van de data in blokken. Transacties die
gebruik maken van die upgrade kunnen meer dan de originele 1MB aan
blokruimte gebruiken (via een aantal slimme trucs die niet verder
toegelicht worden in dit boek).

Nog iets dat bijdraagt aan het verlagen van de hoge transactiekosten is
\emph{batching}. Grote handelsbeurzen en -platformen in het ecosysteem
begonnen met het bundelen van transacties voor verschillende gebruikers
in één: in tegenstelling tot traditionele betalingen via de bank of
PayPal, die altijd geld van één persoon naar een andere sturen, kan een
bitcointransactie een groot aantal inputs combineren en een groot aantal
outputs produceren. Een beurs die bitcoin wil versturen om uit te
betalen aan 100 verschillende gebruikers kan dat dus perfect doen in een
enkele transactie. Op die manier wordt schaarse blokruimte veel
efficiënter gebruikt. Wat ogenschijnlijk slechts enkele transacties per
seconde zijn, kunnen in realiteit perfect duizenden transacties
inhouden.

Met \emph{SegWit} en \emph{batching} komen we al een heel eind om de
vraag naar blokruimte te verlagen. Verdere verbeteringen zullen het
gebruik van blokruimte nog efficiënter maken. Niettemin zal er opnieuw
een dag komen dat transactievergoedingen terug oplopen en blokken door
de grote vraag naar blokruimte weer voller en voller raken.

We hebben nu bijna alle aspecten van bitcoin onder de loep genomen:

\begin{enumerate}
\def\labelenumi{\arabic{enumi}.}
\item
  Centrale bank vervangen door een gedistribueerd grootboek.
\item
  Een loterij opgezet om te bepalen wie in het grootboek mag schrijven.
\item
  Spelers van de loterij worden verplicht om energie te gebruiken om
  lotjes te bemachtigen (door te hashen). Het is voor iedereen eenvoudig
  om een winnend lot te verifiëren door de hash naast ons eigen,
  onafhankelijk bepaald, doelnummer te leggen.
\item
  Duidelijke regels voor de deelnemers: wie de regels niet volgt, zal
  niet geaccepteerd worden. Hun blok met beloning in de zogenaamde
  \emph{coinbase}-transactie wordt dan verworpen en ze ontvangen geen
  bitcoin. Op die manier ontmoedigen we valsspelen en bieden we een
  economische stimulans om de regels te volgen.
\item
  Controle over de timing en selectie van het doelnummer voor de loterij
  door iedereen te laten berekenen wat het doelnummer moet zijn op basis
  van vastgelegde regels in de software en de geschiedenis van de
  laatste 2016 blokken.
\item
  Handhaven van het uitgifteschema door moeilijkheidsgraad aan te passen
  als gevolg van een hogere of lagere hash-rate.
\item
  Gebruik van \emph{open source} code om ervoor te zorgen dat iedereen
  voor zichzelf kan verifiëren dat zij dezelfde regels hanteerden
  betreffende geldigheid van transacties, blokbeloning en
  moeilijkheidsaanpassing.
\end{enumerate}

Geen centrale autoriteit meer. We hebben nu een volledig gedistribueerd
en decentraal systeem. Dat is bijna het volledige plaatje. Eén probleem
rest ons nog. Wanneer een nieuwe deelnemer aansluit bij het netwerk en
een kopie van het grootboek opvraagt, kunnen ze van verschillende
\emph{nodes} afwijkende geschiedenissen ontvangen. Hoe zorgen we voor
een enkele, lineaire geschiedenis en hoe voorkomen we dat miners het
verleden herschrijven?

\hypertarget{het-grootboek-beveiligen}{%
\chapter{Het grootboek beveiligen}\label{het-grootboek-beveiligen}}

We hebben besproken hoe we erin slagen om kopieën bij te houden van, en
te schrijven naar een gedistribueerd grootboek dat zonder dwang of
corruptie werkt. We doen dit met behulp van een loterij en op basis van
validatie door consensus.

Maar wat gebeurt er wanneer een loterijwinnaar besluit om zich
kwaadwillig op te stellen? Kan een miner historische boekingen in het
grootboek aanpassen? Kunnen onze kwaadwillige actoren Eva, Davy en Femke
samenspannen en de geschiedenis herschrijven of rekeningsaldi wijzigen
en zichzelf meer munten toekennen?

In dit deel gaan we de \emph{blockchain} bespreken. Eigenlijk is het
louter een marketingterm die de technologiesector binnengedrongen is. In
bitcoin worden blokken aan elkaar gehecht om een duidelijk link van de
ene transactie naar de volgende te behouden. Een ketting van blokken,
een \emph{blockchain}, dus. Hierdoor ontstaat een lineaire geschiedenis
van creatie en uitgaven sinds Satoshi's \emph{genesis block} in 2009 tot
op vandaag.

In het vorige deel hebben we een klein beetje gelogen om het eenvoudig
te houden. Wanneer je meespeelt met de loterij (om te minen), is het
niet enkel de wachtende transacties en een lukrake \emph{nonce} die je
hasht. Je voegt daar ook nog een hash van het blok net daarvoer toe. Op
die manier ontstaat een duidelijke link tussen jouw blok en het vorige.

Herinner je dat de output van een hash-functie onvoorspelbaar is en
afhankelijk is van alle inputdata die je gebruikt. De hashes van ons
blok bevat nu drie verschillende inputs:

\begin{enumerate}
\def\labelenumi{\arabic{enumi}.}
\item
  De transacties die we naar het grootboek willen schrijven
\item
  Een lukrake \emph{nonce}
\item
  Een hash van de vorige blok die we gebruiken als basis voor de
  geschiedenis van ons grootboek.
\end{enumerate}

\begin{figure}

{\centering \includegraphics{./images/fig11.png}

}

\caption{\label{fig-figuur11}De drie inputs die zijn gebruikt om een
hash voor de loterij te bouwen, bevatten nu ook de vorige winnende hash,
waardoor een link ontstaat van het ene blok naar het andere.}

\end{figure}

Dit stelt ons in staat om een historisch overzicht van elk blok te
bouwen tot en met het eerste blok, ontgonnen door Satoshi. Wanneer we
een nieuw blok aan de ketting toevoegen, moeten we valideren dat het
geen transacties bevat die bitcoins uitgeven die in het verleden al een
gespendeerd zijn.

Wanneer ook maar iets verandert in de inputs van de hash, zal de output
van de hash drastisch en onvoorspelbaar anders zijn. Als je probeert de
data uit een oud blok te manipuleren, zal je ook de resulterende hash
veranderen. Aangezien die hash ook werd gebruikt in de input voor de
blokken die daar net na kwamen, zal je ook de hashes van die blokken
veranderen. De hash van de laatste blok in de ketting, die gelinkt is
met alle voorgaande, doet dienst als een vingerafdruk voor de volledige
geschiedenis van het grootboek tot op dat moment.

Het is onmogelijk om vals te spelen bij proof-of-work omdat iederen weet
hoeveel energie gebruikt moet worden per blok om het vereiste doelnummer
te vinden. Mocht iemand willen proberen om een oudere blok in de ketting
aan te passen, zouden ze de proof-of-work hash moeten aanpassen van de
blok waar ze mee knoeien én die van alle andere blokken die daarna
komen. De \emph{blockchain} is niet alleen fraudebestendig, het is ook
enorm duur om het te proberen.

Iedere nieuwe blok die gevonden wordt draagt effectief bij aan de
veiligheid van alle blokken die ervoor kwamen omdat het de hoeveelheid
elektriciteit die nodig is om de proof-of-work hashes voor de ketting
tot op dat punt te herschrijven, verhoogt. Een transactie in een blok,
begraven onder 6 opeenvolgende blokken wordt aanzien als finaal door de
meeste handelaars. Het zou namelijk een aanzienlijke hoeveelheid energie
kosten om de laatste 6 blokken opnieuw te hashen met de hash-rate van
vandaag. Een transactie die 100 blokken diep zit? Vergeet het maar.

Wanneer je een kopie van de \emph{blockchain} downloadt, is elke
transactie in elke blok volledig transparant. Je kan de proof-of-work
hashes zelf controleren om zeker te zijn dat niets aangepast werd door
de persoon die jouw het grootboek bezorgde.

\hypertarget{wanneer-blokken-botsen}{%
\section{Wanneer blokken botsen}\label{wanneer-blokken-botsen}}

Er ontbreekt nog een element in het consensus-systeem: hoe zorgen we dat
iedereen met dezelfde lineaire geschiedenis van transacties werkt
wanneer miners tegelijk twee blokken vinden en ze naar iedereen
uitsturen?

Stel je voor dat we nu een wereldwijd netwerk draaien. Mensen van over
heel de wereld, van de VS tot China, zijn allemaal aangesloten bij dit
globale netwerk en ze spelen allemaal mee met de proof-of-work mining
loterij.

Iemand in Chicago vindt een geldig blok. Ze deelt het resultaat met het
netwerk en alle computers in de VS accepteren het. Tegelijk vindt iemand
in Shanghai een paar seconden later ook een blok. De buren van die
vinden hebben nog niets vernomen van het blok uit Chicago. Ze accepteren
het Chinese voorstel.

Beide blokken bevatten een transactie van 1 bitcoin van Alice naar Bob.
Onmiddellijk na ontvangst stuurt Bob het bedrag opnieuw door naar
Charles. Wegens het verschil in timing reflecteert het blok uit de VS
deze situatie en Bob saldo van nul. Echter, de Chinese speler vond een
oplossing en publiceerde een blok voordat Bob's transactie naar Charles
bekend was. Het blok uit China toont een saldo voor Bob van 1 bitcoin.

Het netwerk is nu verdeeld. Het is onduidelijk welke versie van het
grootboek juist is, aangezien beide versie geldige transacties bevatten
die correct gelinkt zijn met alle voorgaande transacties. De twee versie
bevatten een geldige hoeveelheid proof-of-work. Dit noemen we een
\emph{chain split} (splitsing van de ketting). Je kan geen centrale
autoriteit raadplegen om uit te maken welke versie wint. Hoe pakken we
dit aan?

Bitcoin biedt een eenvoudige oplossing: gewoon afwachten. Het staat de
miners vrij om te kiezen welk blok ze willen kiezen als basis om verder
op te werken. De Amerikanen zullen minen om verder te bouwen op het
eerste blok waar zij van hoorden en de Chinezen bouwen verder op hun
versie.

In de volgende tien minuten wordt opnieuw een blok gevonden. De code van
bitcoin stipuleert dat diegene die de meeste energie gebruikt heeft voor
alle blokken in hun ketting wint. Deze cruciale spelregel vraagt ons om
het totale verrichte werk in een ketting te sommeren en de voorkeur te
geven aan de ``zwaarste'', cumulatieve proof-of-work ketting. Dit
principe noemen we Nakamoto Consensus, ter ere van Satoshi.

Stel dat een Chinese miner opnieuw een volgend blok wint. Hun ketting is
nu één blok verder dan de Amerikaanse en bevat meer totale
proof-of-work. Wanneer ze deze bevinding meedelen aan de rest van het
netwerk, realiseren de spelers in de VS zich dat de Chinese \emph{nodes}
een ketting geproduceerd hebben waar harder aan gewerkt is. Ze zien hun
``foutje'' onmiddellijk in en herorganiseren zich. Dit betekent dat zij
hun laatste blok alsnog verwerpen en de twee blokken uit China opnemen
in hun grootboek.

\begin{figure}

{\centering \includegraphics{./images/fig12.png}

}

\caption{\label{fig-figuur12}Een kettingsplitsing is een natuurlijk
proces dat optreedt wanneer miners tegelijkertijd een blok vinden tijd.
De ketting die zwaarder wordt door totaal bewijs van werk is geldig, en
de andere blok wordt wees.}

\end{figure}

Het blok uit de VS wordt nu een zogenaamd wees-blok (\emph{orphan
block}). Aangezien het alsnog werd verworpen, gaat de beloning voor de
vinden verloren en de transacties uit dat blok worden niet in het
grootboek geregistreerd. De verworpen transacties zijn niet verloren.
Sommige werden misschien ook opgenomen in het blok uit China en de rest
kan uiteindelijk in een toekomstig blok alsnog in het grootboek
geschreven worden.

Alle onbevestigde transacties worden door miners lokaal bijgehouden op
hun computer in een \emph{mempool}. Elke transactie uit een verworpen
blok komt terug in de mempool terecht. Ze worden vervolgens door iemand
anders in een blok geplaatst, zolang er geen conflict is met de nieuwe
geschiedenis van het grootboek vastgelegd in het laatste blok.

Hoewel we in dit voorbeeld naar de \emph{nodes} refereren als zijnde
Amerikaans of Chinese, weten nodes in realiteit niets over elkaars
identiteit of geografische locatie. Het enige bewijs van validiteit dat
ze nodig hebben, is dat iemand de zwaarste, cumulatieve proof-of-work
ketting heeft en dat de transacties in die ketting zelf allemaal geldig
zijn (geen dubbele-uitgaven).

Dit soort splitsingen van de ketting zijn vrij normaal en gebeuren af en
toe in bitcoin. Doorgaans wordt opnieuw consensus bereikt in het volgend
blok. Verbeteringen in technologie van bekendmaking van blokken en
verhoogde netwerk-connectiviteit maken dit probleem mettertijd minder
groot. Op vandaag, en hoogstwaarschijnlijk voor de nabije toekomst,
heeft bitcoin een harde limiet op de hoeveelheid data dat toegelaten
wordt in een blok. Een deel van de reden dat bitcoin, iedere tien
minuten, relatief kleine blokken produceert, is om te verzekeren dat
\emph{orphans} erg zeldzaam zijn.

Minen is een spel van kansen. Soms liggen blokken tien minuten uiteen,
maar andere keren slechts enkele seconden. Indien we om de paar seconden
blokken zouden produceren, of indien we erg grootte blokken zouden
hebben, zou de kans groter zijn dat Amerikaanse en Chinese blokken
conflicteren. Ze liggen ver uiteen en het duurt langer om elkaar te
bereiken. Als \emph{orphans} te veel voorkomen, zou de ketting
ontrafelen. We zouden \emph{orphan na orphan} zien verschijnen en
\emph{nodes} zouden de tijd niet hebben om uit te maken welke nu het
laatste, juiste blok is.

Het is belangrijk om blokken klein te houden om de kans te vergroten dat
het hele netwerk de laatste blok kan ontvangen vooraleer te beginnen aan
een volgende loterij. De andere, wellicht belangrijkste reden, is dat
kleine blokken ook de vereisten qua hardware voor het draaien van een
\emph{node} relatief laag houden. Zo blijft het de moeite om aan te
sluiten bij het netwerk en blijft mining meer verspreidt over tijd.
Grote blokken zouden miners aanzetten om zich te vestigen in
\emph{datacenters} in gekende geografische locaties om splitsingen, die
slecht zijn voor hun rendabiliteit, te vermijden

\hypertarget{de-enige-echte-ketting}{%
\section{De enige, echte ketting}\label{de-enige-echte-ketting}}

Laten we terugkeren naar ons voorbeeld uit hoofdstuk 3 waarin Henri voor
het eerst aansluit bij het bitcoin netwerk.

De \emph{node} van Henri zal een connectie maken met enkele andere
\emph{nodes} op het netwerk. Vervolgens vraagt hij die om nog andere
\emph{nodes} die zij kennen en maakt ook daar verbinding mee. Dit heet
\emph{node discovery}.

Sommige van die \emph{nodes} zullen slechte bedoelingen hebben en valse
kopieën van het grootboek bezorgen. Ze kunnen bijvoorbeeld ongeldige
handtekeningen voor transacties bevatten, of vervalste en oneerlijk
ontgonnen bitcoin die geen geldige proof-of-work hashes hebben. Al die
versies zullen onmiddellijk verworpen worden en de afzenders worden door
de node van Henri verbannen.

Andere \emph{nodes} zullen wel eerlijk zijn, maar conflicterende versies
van de waarheid hebben. Misschien ging een net offline en loopt hij nog
enkele blokken achter. Wanneer hij verschillende kopieën van de
blockchain download die allemaal geldig zijn, zal de software gebruik
maken van Nakamoto Consensus. Door te meten wat het totale, cumulative
proof-of-work is, weet Henri onmiddellijk welke de zwaarste ketting is
die wordt aanzien als de enige echte.

\emph{Nodes} spreken voortdurend met elkaar om zeker te zijn dat ze het
meeste recente blok hebben. Aangezien alle nodes de regel van de
zwaarste ketting volgen, is er consensus over wat de ware staat van het
grootboek is. Henri moet niet vertrouwen op een meerderheid van stemmen.
Dit systeem zou makkelijk te bedriegen vallen door een grote hoeveelheid
\emph{nodes} te draaien die kwaadaardig zijn.

Zelfs als Henri met verschillende verouderde of kwaadaardige nodes en
slechts 1 correcte node connecteert, dan zou de bitcoin-software weten
welke de juiste versie is. Die versie bevat de grootste hoeveelheid
proof-of-work en bevat geldige transacties helemaal tot aan de
\emph{genesis block}. Het belang hiervan kan niet genoeg benadrukt
worden. Henri moet niemand vertrouwen; de software op zijn computer zal
alle validaties uitvoeren die nodig zijn om zeker te zijn dat de
\emph{blockchain} waar hij mee werkt de enige juiste is.

Het is daarom uiterst moeilijk voor hackers om een \emph{node} een valse
kopie van de ketting te bezorgen. Om dat te doen zou je alle eerlijke
connecties moeten kunnen uitsluiten en het doelwit enkel laten
verbinding maken met je eigen, kwaadaardige \emph{nodes}.

\hypertarget{omkeerbaarheid-van-transacties}{%
\section{Omkeerbaarheid van
transacties}\label{omkeerbaarheid-van-transacties}}

Doorgaans ontstaan concurrerende versies van het grootboek per toeval en
wordt snel uitgeklaard welke de juiste versie is. Maar iemand die het
netwerk wil aanvallen kan gebruik maken van Nakamoto Consensus door meer
dan 50\% van de totale \emph{hash-rate} te beheren. Op die manier kunnen
ze de zwaarste, cumulatieve proof-of-work ketting produceren. Die versie
van het grootboek zal transacties bevatten die de aanvaller kiest,
zolang ze genoeg energie willen verbruiken om de aanval door te zetten.
Wanneer ze deze ketting bekendmaken aan het netwerk, zouden andere
\emph{nodes} hem accepteren als zijnde de echte. Die heet een
51\%-aanval, omdat je meer dan de helft van alle rekenkracht op het
netwerk nodig hebt om de aanval succesvol uit te voeren.

Het is belangrijk om te begrijpen dat er geen echte finaliteit van
transacties is in bitcoin, aangezien 51\%-aanvallen of \emph{orphan
blocks} altijd tot de mogelijkheid behoren. Ontvangers van transacties
wachten typisch tot enkele blokken boven op hun transactie werd gemined.
Wanneer de hoeveelheid energie die nodig zou zijn om de ketting te
herschrijven hoog genoeg is, wordt het erg onwaarschijnlijk dat de
transactie ongedaan wordt gemaakt en beschouwen de deze transfer als
finaal.

Blokken die ontgonnen worden boven op een blok waar onze transactie in
zit, noemen we doorgaans \emph{confirmaties}. Dus wanneer iemand zegt
dat een bitcointransactie 6 confirmaties heeft, wordt bedoeld dat er 6
blokken gepasseerd zijn sinds de transactie in het grootboek zit.
Wanneer je als handelaar een digitaal product verkoopt met geringe
marginale kosten, kan het voldoende zijn met slechts 1 confirmatie of
zelfs zonder confirmaties. Je stuurt de download link zodra de
transactie aangekondigd werd op het netwerk. Wanneer je een huis
verkoop, kan het misschien meer aangewezen zijn om te wachten op 12
confirmaties. Dat kost gemiddeld zo'n twee uur aan mining. Hoe langer je
wacht, hoe meer proof-of-work boven op het blok met jouw transactie
komt. Het wordt veel duurder om de transactie om te keren. Vandaag
accepteren de meeste mensen een transactie met 6 confirmaties.

Moest de \emph{hash-rate} van bitcoin in belangrijke mate dalen, wat
betekent dat minder energie ieder blok beveiligd, dan kan je altijd het
aantal confirmaties verhogen vereist voor een ``finale afwikkeling''.
Hoewel de niet-finaliteit van transacties op het eerste zicht
verontrustend lijkt, is het belangrijk om in het achterhoofd te houden
dat transacties met kredietkaarten tot wel 120 dagen later kunnen worden
teruggedraaid.

Aan de andere kant, zijn bitcointransacties na een paar blokken zo goed
als onomkeerbaar. Vanuit dit oogpunt, is de omkeerbaarheid en finaliteit
van bitcoin een ontzettende verbetering met de meeste traditionele
betalingsnetwerken, althans voor de handelaar.

Op vandaag wordt geschat dat iemand met de energie van het volledige
bitcoin netwerk ter beschikking -- een serieuze uitdaging, aangezien je
toegang tot de energie van een klein land én alle gespecialiseerde
hardware moet hebben -- nog altijd meer dan een jaar zou nodig hebben om
de volledige geschiedenis van de \emph{chain} te herschrijven. Je kan
deze data bekijken op \url{https://bitcoin.sipa.be/}

Footnote: dit interessante artikel gaat dieper in op ongeldige blokken
in bitcoin:
\url{https://hackernoon.com/bitcoin-miners-beware-invalid-blocks-need-not-apply-51c293ee278b}

\hypertarget{forks-en-51-aanvallen}{%
\chapter{Forks en 51\%-aanvallen}\label{forks-en-51-aanvallen}}

In de begindagen minede Satoshi zijn eerste bitcoins door gebruik te
maken van de CPU in zijn computer. Aangezien de moeilijkheid
oorspronkelijk erg laag was, bleek het relatief goedkoop om die munten
te produceren.

Met tijd, begonnen mensen te sleutelen aan de software om te minen en
werd het efficiënter. Uiteindelijk werd software geschreven die gebruik
maakte van gespecialiseerde grafische apparatuur (GPU's), gewoonlijk
gebruikt door \emph{gamers}, om te minen.

Met GPU's werd minen een heel stuk efficiënter. De moeilijkheidsgraad
ging snel omhoog om de grote hoeveelheden \emph{hash-rate} van de GPU's
bij te benen. Minen met een CPU werd onrendabel.

Na de komst van GPU mining werd de efficientie van miners nog verhoogd
door de product van \emph{Application Specific Integrated Circuits}
(ASIC's). Deze computerchips doen maar 1 ding: de bitcoin sha256
hash-functie uitvoeren. Ze kunnen enkel dit algoritme en bleken opnieuw
een orde van grootte meer efficiënt dan GPU's. De grafische kaarten voor
gamers werden snel onrendabel om te minen, net zoals gebeurde met CPU's.
Iedere paar jaar komt een nieuwe generatie ASIC's naar buiten die beter
werken dan de vorige versies.

De eerste miners op het netwerk spendeerden slechts enkele centen aan
elektriciteit om hun bitcoin te produceren. Terwijl de prijs omhoogging,
en meer miners zich aansluiten bij het netwerk, ging de
moeilijkheidsgraad van het minen omhoog en werd het duurder en duurder
om bitcoins te produceren. Vandaag hangt de prijs rond de \$8.000 per
BTC en spenderen miners duizenden dollars per geproduceerde bitcoin.

\hypertarget{mining-pools}{%
\section{Mining pools}\label{mining-pools}}

Een probleem met bitcoin mining is dat het niet-deterministisch is,
zoals het rollen met een dobbelsteen. Dit betekent dat je een grote
hoeveelheid geld kunt uitgeven aan elektriciteit en nooit een geldig
blok vinden.

In 2010 werden zogenaamde \emph{mining pools} al populair om het
probleem van variabiliteit te verbeteren. Een \emph{mining pool} is een
manier om het risico te delen, vergelijkbaar met een verzekering.

Alle miners dragen bij aan een \emph{pool}. Het lijkt alsof ze 1 grote
miner zijn op het netwerk. Wanneer iemand in die \emph{pool} een geldig
blok vindt, wordt de beloning gedeeld met iedereen die bijdroeg,
proportioneel verdeelt volgens de \emph{hash-rate} die ze bijdroegen.
Dit stelt ook kleine mining-operaties in staat om beloningen te
verdienen voor de kleine hoeveelheid rekenkracht die zij bijdragen. In
ruil voor de dienst die zij aanbieden, nemen \emph{pool} een stukje van
de beloning.

Het concept van \emph{mining pool} introduceerde enige centralisatie.
Gebruikers sloten zich aan bij de grotere en bekendere diensten. Maar de
gebruikers van de pools zijn wel eigenaar van hun \emph{hash-rate}. Ze
kunnen op elk moment veranderen naar een nieuwe \emph{mining pool}.

Er bestaat historisch precedent voor individuele miners die een
\emph{pool} verlaten omdat die te groot is geworden. In 2014 zat bijna
de helft van de rekenkracht op het netwerk bij Ghash.io. De miners zagen
in hoe gecentraliseerd dit werd en gingen vrijwillig over op andere
diensten.

Hoewel de relatief gecentraliseerde mining pools vandaag een realiteit
zijn, worden constant verbeteringen voorgesteld om de mining-technologie
beter te maken, zoals bijvoorbeeld BetterHash. Met deze verbetering
zouden individuele miners meer controle hebben over welke transacties ze
precies willen minen, zonder hiervoor afhankelijk te zijn van
coördinatie van hun \emph{mining pool}.

\hypertarget{aanvallen}{%
\section{51\%-aanvallen}\label{aanvallen}}

Centralisatie in \emph{mining pools} leidt tot bezorgdheid dat enkele
van de grote pools zouden kunnen samenspannen om een 51\%-aanval uit te
voeren op het netwerk. Vandaag hebben de vijf grootste identificeerbare
pools meer dan 50\% van de totale mining \emph{hash-rate}. Laat ons
onderzoeken hoe zo'n aanval wordt uitgevoerd en welke gevaren eraan
verbonden zijn.

Wanneer je net meer dan 50\% van de rekenkracht in handen hebt, kan je
de meerderheid van de toevoegingen aan het grootboek domineren. Je kan
immers de zwaarste ketting produceren, mits je de aanval even kan
volhouden. Herinner je dat Nakamoto Consensus stelt dat \emph{nodes}
altijd de zwaarste cumulative proof-of-work \emph{chain} moeten
accepteren die ze kennen.

Hier is een voorbeeld van een simpele 51\%-aanval:

\begin{enumerate}
\def\labelenumi{\arabic{enumi}.}
\item
  Stel dat heel het netwerk bitcoin hasht aan 1000 hashes per seconde.
\item
  Je koopt een heleboel mining hardware en elektriciteit om 2000 hashes
  per seconde te produceren. We hebben nu 66\% van de totale
  \emph{hash-rate} (2000 van de 3000 hashes per seconde).
\item
  Je begint een ketting te minen met enkel lege blokken.
\item
  Binnen twee weken publiceer je de ketting van lege blokken. Omdat je
  veel sneller was dan de eerlijke miners, zal jouw versie twee keer zo
  veel proof-of-work bevatten. De aankondiging van jouw versie op het
  netwerk zal alle andere nodes doen overgaan op een \emph{reorg} waarin
  alle transacties uit de laatste 2 weken ongedaan worden gemaakt.
\end{enumerate}

Naast het minen van lege blokken, waardoor de \emph{chain} onbruikbaar
wordt, is het ook mogelijk om een dubbele-uitgaven aanval uit te voeren:

\begin{enumerate}
\def\labelenumi{\arabic{enumi}.}
\item
  Stuur bitcoin naar een handelsbeurs
\item
  Ruil om naar USD en haal de dollars af.
\item
  Op een later tijdstip kan je de \emph{chain} publiceren zonder de
  transactie naar de handelsbeurs uit stap 1 hierboven.
\item
  Je herschreef de geschiedenis van het grootboek en bezit nu zowel de
  originele bitcoins als de dollars.
\end{enumerate}

De energieconsumptie van bitcoin's \emph{hash-rate} is ongeveer
equivalent met dat van een gemiddeld land. De benodigde hardware en
elektriciteit verwerven om zo'n aanval uit te voeren is enorm duur.
Schattingen tonen dat het ongeveer \$700.000 dollar per uur kost om een
51\%-aanval uit te voeren. Die kost blijft snel stijgen. Deze schatting
neemt de reactie van eerlijke miners, wat de kost van de aanval nog zou
opdrijven, niet in rekening.

Daarnaast is ook nog eens zeer moeilijk om een dubbele-uitgaven aanval
succesvol af te ronden zonder vingerafdrukken achter te laten die
duidelijk maken wie je bent. Tenslotte, je zou een hoeveelheid energie
van een gemiddeld land aan het opgebruiken zijn, na miljoenen dollars
aan hardware aan te kopen, en je moet een handelsbeurs vinden waar je
miljoenen dollars kan uitcashen om de aanval te doen slagen.

Maar stel dat een kwaadaardige partij, met ongelimiteerde budgetten,
zoals een overheid, zou beslissen om dit te doen en erin slaagt om de
aanval vol te houden zodat het effect op het netwerk meer dan gewoon wat
overlast is. Dan nog zou het netwerk in principe kunnen reageren door de
proof-of-work functie te veranderen (naar iets anders dan sha256).
Hierdoor zou alle bestaande bitcoin mining hardware, die de aanvaller
ook gebruikt, nutteloos worden. Die machines dienen specifiek om sha256
hashes te doen. Maar het aanpassen van het proof-of-work algoritme moet
je zien als een laatste redmiddel en alle eerlijke miners zouden
eveneens in de klappen delen. Toch zou het netwerk overleven.

Naast de onwaarschijnlijkheid van de aanval, geeft het hebben van de
meerderheid van de \emph{hash-rate} je nog altijd het recht niet om
dingen te doen die nog belangrijker zijn:

\begin{enumerate}
\def\labelenumi{\arabic{enumi}.}
\item
  Je kan geen munten uit het niets creëren die de regels van
  uitgifteschema schenden. Aangezien dit de consensusregels van de
  blokbeloning zou overtreden, worden de blokken gewoon verworpen, zelfs
  al hadden ze genoeg proof-of-work.
\item
  Je kan geen munten uitgeven die niet van jou zijn. Het is onmogelijk
  om een geldige digitale handtekening hiervoor te leveren.
\end{enumerate}

De nodes die bitcoin accepteren voor betalingen zouden het netwerk
eerlijk houden in het aanzicht van een oneerlijke meerderheid van miners
door simpelweg de regels van bitcoin toe te passen. Aldus is een
51\%-aanval eerder hinderlijk dan een veiligheidsrisico. Waarschijnlijk
is het \emph{worst-case scenario} hier dat een overheid met een goed
gevulde portemonnee, bitcoin probeert onbruikbaar te maken. Maar zo'n
aanval kan niet eeuwig duren. Wanneer bitcoin herstelt van een aanval
zoals beschreven, zou het enkel haar veerkracht benadrukken en een nog
groter probleem worden voor zij die het willen aanvallen.

Hoewel bitcoin tot dusver nog nooit met succes aangevallen is met een
51\%-aanval, hebben we wel voorbeelden van andere \emph{blockchain} die
veel minder hash-rate hebben die hen beveiligd. In deze gevallen waren
de beurzen het slachtoffer van de aanval en verloren ze geld aan munten
met lage hash-rate die ze in de eerste plaats nooit hadden moeten
aanbieden op hun platform.

\hypertarget{accounts-zonder-identiteit}{%
\chapter{Accounts zonder identiteit}\label{accounts-zonder-identiteit}}

We hebben nu een gedistribueerd grootboek gebouwd zonder centrale
autoriteit, een loterij om te bepalen wie erin mag schrijven, een
systeem om eerlijke miners te belonen en valsspelers te straffen, een
manier om de moeilijkheidsgraad aan te passen om een consistent
uitgifteschema te garanderen en conflicten te verminderen en een systeem
om de geldigheid van de keten te controleren door te kijken naar het
cumulatieve \emph{proof-of-work} en de transactiegeschiedenis.

Laten we nu eens kijken naar identiteit. Om in een traditioneel
banksysteem geld te versturen moet je eerst jezelf identificeren bij de
bank. Je geeft je identiteit (je bankpas) en pincode in bij de
geldautomaat, of typt je gebruikersnaam en wachtwoord in een mobiele
app. De bank zorgt ervoor dat entiteiten niet dezelfde identiteit delen.

Hoe kunnen we in ons nieuwe op bitcoin gebaseerde financiële systeem een
rekening openen zonder een centrale partij die de identiteiten bijhoudt?
Hoe kunnen we onze identiteit loskoppelen van financiële transacties,
zodat identiteitsdiefstal voorkomen wordt en we niemand hoeven te
vertrouwen met onze informatie? Hoe zorgen we ervoor dat wanneer Alice
aankondigt Bob te willen betalen, we kunnen garanderen dat zij het echt
is en dat ze de bevoegdheid heeft om dat geld te verplaatsen?

\hypertarget{een-bitcoinrekening-aanmaken}{%
\section{Een bitcoinrekening
aanmaken}\label{een-bitcoinrekening-aanmaken}}

In ons systeem bestaat geen centrale tussenpersoon (zoals een bank) om
alle rekeninghouders te registreren. Wat als we iedereen zijn eigen
gebruikersnaam en wachtwoord laten kiezen? Doorgaans controleert een
bank of een gebruikersnaam al bestaat, maar dat is in ons geval
onmogelijk aangezien er geen centrale partij is om nieuwe identiteiten
uit te geven. Een gebruikersnaam en wachtwoord zijn in ons geval niet
voldoende. We zullen opnieuw een techniek moeten gebruiken die we al
kennen uit eerdere hoofdstukken, namelijk gigantische willekeurige
getallen.

Door grote willekeurige getallen te genereren, kan iedereen loten kopen
om mee te spelen met de loterij. We kunnen hetzelfde doen om rekeningen
aan te maken. Om een bitcoinrekening, of \emph{adres}, aan te maken,
zullen we eerst twee 256-bits getallen genereren die wiskundig aan
elkaar gekoppeld zijn --- een \emph{publiek/privé-sleutelpaar}. Herinner
dat een 256-bits getal ongeveer even groot is als het aantal atomen in
het heelal, dus twee mensen die per ongeluk hetzelfde sleutelpaar
genereren is haast onmogelijk. We geven ons adres aan iedereen die ons
geld wil sturen. We gebruiken de privésleutel om het geld weer uit te
geven. Dit is hoe het werkt.

Versleuteling is een methode om leesbare tekst om te zetten naar
geheimschrift, zodat alleen iemand die de sleutel heeft het originele
bericht kan lezen door de versleuteling weer ongedaan te maken. Als
kinderen speelden sommigen van ons al met speelgoed waar een enkele
sleutel nodig was om een bericht in wartaal te veranderen, om het even
later met dezelfde sleutel weer te ontsleutelen. Dit soort codering
wordt symmetrisch genoemd. Een systeem met een publiek/privé-sleutelpaar
is \emph{asymmetrisch}, omdat je met de ene sleutel kunt versleutelen en
de andere moet gebruiken om weer te ontsleutelen.

De publieke sleutel kan je naar believen met de hele wereld delen.
Mensen die je berichten willen sturen kunnen ze versleutelen met je
publieke sleutel, en omdat alleen jij de privésleutel bezit, ben jij de
enige die de berichten kan decoderen.

Laten we eens kijken hoe Alice munten naar Bob stuurt. Om een transactie
te ontvangen genereert Bob een sleutelpaar en houdt hij zijn
privésleutel geheim. Hij genereert een \emph{adres}; een lange reeks van
cijfers en letters op basis van een hash van zijn publieke sleutel. Bob
deelt dit adres vervolgens met Alice.

Vergelijk dit adres met een brievenbus. In plaats van brieven kan Alice
munten in deze brievenbus laten vallen. Maar alleen Bob heeft de
privésleutel die nodig is om de brievenbus te openen en de munten te
besteden.

Wanneer je geld op de bank zet, geef je ze je gebruikersnaam en
wachtwoord. Wanneer je cheques uitschrijft, onderteken je met je naam om
te verifiëren dat jij het bent die de cheque uitschrijft. Wanneer je
bitcoins wilt verplaatsen, toon je bewijs dat je de sleutel bezit van
het adres waar de munten bij horen.

Alice moet bewijzen dat ze de privésleutel van haar publieke sleutel
heeft. Ze wil haar privésleutel echter niet blootstellen aan hackers
omdat ze dan haar munten uit haar brievenbus kunnen stelen.

Alice's bewijs van sleutelbezit wordt een \emph{digitale handtekening}
genoemd. Alice construeert een transactie die er ongeveer zo uit ziet:

\begin{verbatim}
Adres 12345 met 2,5 bitcoins 
verstuurt 2 bitcoins naar adres 56789 en 
stuurt 0.5 bitcoins terug naar adres 12345
\end{verbatim}

In werkelijkheid zijn de adressen gigantische 160-bits getallen. Zij
versleutelt vervolgens dezelfde transactie met haar privésleutel en
genereert daarmee een \emph{digitale handtekening}.

Wanneer ze haar transactie publiceert naar de andere nodes op het
netwerk, onthult ze de publieke sleutel van de brievenbus van waaruit ze
verzendt, en de digitale handtekening die ze met haar privésleutel heeft
gegenereert. Alice kondigt het volgende aan:

\begin{itemize}
\item
  Ik stuur munten vanaf adres 12345
\item
  Hier is de publieke sleutel voor adres 12345. Je zult zien dat als je
  de publieke sleutel hasht, je hetzelfde adres zult verkrijgen.
\item
  Hier is een handtekening die ik heb versleuteld met de privésleutel
  die correspondeert met dit adres. Je kunt de publieke sleutel
  gebruiken om deze te ontsleutelen en controleren of het overeenkomt
  met mijn transactiedata.
\end{itemize}

\begin{figure}

{\centering \includegraphics{./images/fig13.png}

}

\caption{\label{fig-figuur13}Er wordt een digitale handtekening
gegenereert door de transactie te versleutelen met de privésleutel. Dit
kan worden ontsleuteld met behulp van de openbare sleutel, die bij
iedereen bekend is.}

\end{figure}

Aangezien iedereen nu de publieke sleutel van de brievenbus van Alice
heeft, kan de digitale handtekening eenvoudig worden ontsleuteld. Als de
handtekening kan worden ontsleuteld met de publieke sleutel van het
adres, weet iedereen dat Alice in het bezit moet zijn van de
bijbehorende privésleutel. Ontsleutelen was anders niet mogelijk
geweest, omdat de publieke sleutel alleen berichten kan ontsleutelen die
zijn versleuteld met de bijbehorende privésleutel. Het is belangrijk om
hierbij op te merken dat niemand haar privésleutel heeft gezien, maar
wél het bewijs dat ze over de juiste privésleutel beschikt om haar
handtekening te zetten.

In tegenstelling tot je pincode of een handtekening op een cheque, is
een digitale handtekening specifiek voor de unieke transactiegegevens
die je ondertekend. De handtekening kan dus niet worden gestolen en
opnieuw worden gebruikt voor een andere transactie. Elke transactie
krijgt een andere handtekening, zelfs als deze wordt verzonden vanaf de
hetzelfde publieke adres, met dezelfde privésleutel, aangezien alle
nieuwe gegevens de handtekening wijzigen.

\hypertarget{kan-je-een-privuxe9sleutel-raden}{%
\section{Kan je een privésleutel
raden?}\label{kan-je-een-privuxe9sleutel-raden}}

Laten we eens kijken hoe groot de kans is om een privésleutel te raden,
wat je de mogelijkheid zou geven om munten te verplaatsen van het
bijbehorende publieke adres. Onthoud dat een sleutel uit 256-bits
bestaat. Elke bit heeft slechts twee waarden (een of nul). Dat betekent
dat je elk bit kunt visualiseren als het opgooien van een muntje.

Een privésleutel van 1 bit is alsof we een munt opgooien. Kop of munt,
één of nul? Je hebt een kans van één op twee om het goed te raden.

Basis statistiek: de kans op verschillende gebeurtenissen wordt berekend
door de individuele kans van elke gebeurtenis met elkaar te
vermenigvuldigen. Als een munt \(1/2\) kans heeft om kop te landen, dan
is de kans om twee keer op rij kop te gooien \(1/2 \times 1/2 = 1/4\) of
1 op 4.

Als je de uitkomst van 8 munten op een rij zou moeten raden is dat
\(2^{8}\); een kans van één op 256.

Een kentekenplaat heeft 6 letters en cijfers. Er zijn 26 letters en 10
cijfers, dus in totaal 36 tekens. Het aantal mogelijke kentekenplaten is
dus \(36^{6}\), en je kans om de mijne te raden is dus een op de twee
miljard.\footnote{De inspiratie voor dit gedeelte kwam van een
  uitstekende Medium post die de waarschijnlijkheid van een aantal
  gebeurtenissen in detail beschrijft. Ik raad aan de volledige post te
  lezen voor de context:
  \url{https://medium.com/@kerbleski/a-dance-with-infinity-980bd8e9a781}}

Een kredietkaart bestaat uit zestien cijfers. Ieder cijfer heeft 10
verschillende mogelijkheden, dus je kans om mijn kredietkaart te raden
is \(10^{16}\), dat is één op 10.000.000.000.000.000 of ongeveer één op
tien quadriljoen.

Er zijn ongeveer \(10^{50}\) atomen op aarde. Als ik aan een willekeurig
atoom denk, is jouw kans om die te raden ongeveer

\begin{verbatim}
    Één op 1.000.000.000.000.000.000.000.000.
    000.000.000.000.000.000.000.000.000.
\end{verbatim}

Een privésleutel heeft 256 bits, wat gelijk is aan \(2^{256}\) of
ongeveer \(10^{77}\). En daarmee is de kans om mijn volledige
privésleutel correct te raden kleiner dan de kans op het raden van een
specifiek atoom in het universum, of de kans om 10 keer achter elkaar de
jackpot van de Staatsloterij te winnen.

Maar wat nou als een super krachtige computer al het gokwerk voor je zou
kunnen doen? Dit wordt het best uiteengezet in de Redditpost op
\url{https://bit.ly/2Dbw9Qd} en ik raad het je aan om hem in zijn
volledigheid te lezen. Hij is wellicht wat technisch, maar de laatste
paragraaf geeft je een goed beeld wat er voor nodig is om alle mogelijke
256-bits sleutels te noteren:

\begin{quote}
Dus, als je de volledige planeet als harde schijf zou kunnen gebruiken,
1 byte per atoom zou kunnen opslaan, de sterren gebruikt als brandstof
om je langs 1000 miljard sleutels te fietsen, dan heb je 37 octiljoen
(\(10^{48}\)) keer de aarde nodig voor opslag, en 237 miljard keer de
zon om je apparaat van brandstof te voorzien, waar je alles te samen
3.6717 octodeciljoen (\(10^{57}\)) over zult doen

--- {u/PSBlake, R/Bitcoin}
\end{quote}

Het is dus praktisch onmogelijk om iemands privésleutel te raden. Dat
niet alleen; het aantal mogelijke bitcoinadressen is zo groot, dat het
aan te raden is om voor iedere transactie een nieuw adres met een nieuwe
privésleutel te genereren. Dus in plaats van een enkele bankrekening,
zou je zo maar eens over duizenden of zelfs miljoenen bitcoin accounts
kunnen beschikken. Namelijk 1 voor iedere transactie die je ooit hebt
ontvangen.

Het is misschien verontrustend dat je bitcoin account slechts beveiligd
is door toeval, maar hopelijk helpt de verklaring hierboven je om te
beseffen dat dit vele malen veiliger is dan de pincode tot je
bankrekening, opgeslagen op een centrale server, en een eenvoudig doel
voor hackers.

\hypertarget{het-saldo-monitoren}{%
\section{Het saldo monitoren}\label{het-saldo-monitoren}}

Het wordt tijd om een laatste leugentje van de voorgaande hoofdstukken
te corrigeren. Er worden namelijk geen saldi bijgehouden op de
blockchain. In plaats daarvan maakt bitcoin gebruik van zogeheten UTXO:
Unspent Transaction Outputs. De transactie output is simpelweg een woord
voor de muten die je in een vorige transactie hebt ontvangen, ongeacht
of je ze van iemand ontvangen hebt of door het minen hebt verkregen in
de \emph{coinbase transactie}.

Bitcoins zijn deelbaar in 100 miljoen eenheden die we satoshis noemen.
Dit in tegenstelling tot de metalen munten die vaak alleen in vooraf
gespecificeerde eenheden voorkomen, zoals we bijvoorbeeld de 10 cent,
twintig cent en euro muntstukken kennen. Daarom zul je soms munten van
verschillende adressen met elkaar moeten combineren, of juist een
grotere UTXO in tweeën moeten splitsen, om ze naar iemand anders te
sturen. Zie het als het sturen van een stapel munten naar een machine
die ze omsmelten en nieuwe munten maken van elke waarde die u wilt.
Portefeuilles, die later in dit hoofdstuk worden besproken, beheren dit
over het algemeen allemaal achter de schermen, zodat u alleen het bedrag
hoeft op te geven dat u wilt verzenden.

Laten we zeggen dat Alice een adres heeft dat 1 bitcoin bevat. Ze wil
0,3 bitcoins naar Bob sturen. Ze genereert een transactie die haar adres
toont met een 1 bitcoin UTXO als input en twee outputs: een nieuwe
bitcoin UTXO ter waarde van 0.3 naar Bob's adres, en een nieuwe bitcoin
UTXO ter waarde van 0.7 terug naar haar eigen adres als wisselgeld. Het
wisselgeld kan naar haar oorspronkelijke verzendadres gaan, of voor een
betere privacy kan ze het naar een nieuw adres sturen dat ze ter plekke
genereert.

\begin{figure}

{\centering \includegraphics{./images/fig14.png}

}

\caption{\label{fig-figuur14}Als je geen UTXO hebt in het exacte bedrag
dat je wilt sturen, dan zal er een worden gesplitst om wisselgeld te
maken. Je kunt ook verschillende UTXO's combineren om een nieuwe grotere
te maken.}

\end{figure}

Er is geen manier in de keten om te vertellen wie welk adres
controleert. Daarvoor zou je de corresponderende private sleutels moeten
kennen en ze moeten koppelen aan echte identiteiten. Het UTXO-model
moedigt een zeer mooi privacymechanisme aan door bij elke
muntverplaatsing het wisselgeld naar een nieuw adres te sturen. Zo kan
een persoon honderden of duizenden adressen bezitten als hij vele malen
munten heeft verzonden of ontvangen. De software van de portemonnee
beheert dit alles voor ons, zodat we ons geen zorgen hoeven te maken
over de details.

Dus, om het saldo van een bepaald adres te controleren, moeten we
eigenlijk alle UTXO's optellen die dit adres als uitgang hebben. De
totale set van huidige UTXO's in bitcoin groeit wanneer mensen van één
adres naar veel adressen sturen, en krimpt wanneer mensen
consolidatietransacties uitvoeren waarbij munten van veel adressen aan
één adres worden uitgegeven.

Het UTXO-model maakt een eenvoudige en efficiënte validatie van dubbele
uitgaven mogelijk, omdat een bepaalde UTXO maar één keer kan worden
uitgegeven. Wij hoeven niet de hele geschiedenis van uitgaven van een
bepaalde rekening te kennen.

We kunnen ook een willekeurig aantal UTXO's tegelijk creëren en
vernietigen, waardoor complexe transacties ontstaan die verschillende
inputs en outputs mixen. Dit maakt het idee van CoinJoin\footnote{\href{https://en.bitcoin.it/wiki/CoinJoin}{Zie
  https://en.bitcoin.it/wiki/CoinJoin}} mogelijk, waarbij verschillende
partijen deelnemen aan een enkele bitcoin-transactie die een willekeurig
aantal inputs mengt om een willekeurig aantal outputs te produceren, en
zo de geschiedenis van de UTXO's versluiert. De populariteit van
dergelijke technieken neemt toe en is belangrijk voor de privacy en
fungibiliteit, een term die zegt dat elke bitcoin gelijkwaardig is aan
elke andere bitcoin. Op die manier, als sommige bitcoins in de handen
van onfrisse partijen terechtkomen, zijn ze niet voor eeuwig bezoedeld
alleen maar omdat ze een keer voor iets snode zijn gebruikt.

\hypertarget{wallets}{%
\section{Wallets}\label{wallets}}

Een account aanmaken is niets meer dan een willekeurig 256 bit
sleutelpaar aanmaken. We kunnen duizenden of miljoenen accounts
aanmaken, dus hebben we een manier nodig om ze te traceren. In bitcoin
wordt het woord portemonnee gebruikt om te verwijzen naar elk soort
apparaat dat uw sleutels bijhoudt. Dat kan zo simpel zijn als een stuk
papier of zo complex als een stuk hardware.

De originele bitcoin-code die door Satoshi werd gepubliceerd, werd
geleverd met een software-portemonnee. Deze portemonnee genereerde uw
adressen voor u, sloeg uw sleutels op en selecteerde UTXO's voor u om
uit te geven, zodat u gemakkelijk bitcoins van elke waarde kon
versturen.

In tegenstelling tot de portemonnee van uw bank, die meestal de vorm
heeft van een mobiele of webapplicatie die door uw bank is gemaakt, is
bitcoin een volledig open systeem. Daarom zijn er tientallen
portemonnees, waarvan de meeste gratis zijn, en veel ook open source,
evenals een half dozijn implementaties van hardware-portemonnees en er
komen er nog meer. Iedereen met kennis van computerprogrammering kan
zijn eigen portemonnee bouwen of de code van een open source portemonnee
lezen om er zeker van te zijn dat er niets vals aan de hand is.

Omdat je private key het enige is dat je nodig hebt om je munten uit te
geven, moet je die goed bewaken. Als iemand uw kredietkaart steelt, kunt
u het bedrijf bellen en een fraudeklacht indienen en proberen uw geld
terug te krijgen. Bij bitcoin is er geen tussenpersoon. Als iemand uw
privésleutel heeft, heeft hij uw munten in handen, en er is niemand die
u kunt bellen.

Prive-sleutels zijn ook gevoelig voor verlies. Als u uw portemonnee op
uw computer bewaart en de computer wordt gestolen of vliegt in brand,
dan heeft u een probleem. Als u de beste bitcoin-praktijken volgt en
elke keer dat u betalingen ontvangt een nieuw adres genereert, wordt het
veilig opslaan en back-uppen van deze privésleutels al snel een lastige
zaak.

In de loop der tijd heeft het bitcoin-ecosysteem een aantal oplossingen
voor dit probleem ontwikkeld. In 2012 werd BIP32 (Bitcoin Improvement
Proposal, een mechanisme voor mensen om ideeën te verspreiden over hoe
bitcoin verbeterd kan worden) voorgesteld om Hierarchical Deterministic
Wallets te maken. Het idee hierachter is dat we met slechts één
willekeurig getal, een se d genaamd, continu vele sleutelparen kunnen
genereren die bitcoinadressen en privésleutels voor hen
vertegenwoordigen.

Als u tegenwoordig een van de algemeen beschikbare software- of
hardware-portemonnees gebruikt, genereert deze automatisch nieuwe
sleutels voor u voor elke transactie, zodat u slechts één hoofdsleutel
hoeft te back-uppen.

In 2013 kwam BIP39 om het back-uppen van sleutels nog eenvoudiger te
maken. In plaats van een willekeurig getal te gebruiken, zouden sleutels
worden gegenereerd uit een willekeurige set van door mensen leesbare
woorden. Hier is een voorbeeld van een seed:

\begin{verbatim}
    witch   collapse    practice    feed
    shame   open        despair     creek
    road    again       ice         least
\end{verbatim}

Met deze methode werd het back-uppen van sleutels heel eenvoudig: je kon
het zaad op een stuk papier schrijven en het in een kluisje stoppen. Je
zou de zin zelfs uit je hoofd kunnen leren en uit een falend economisch
regime als Venezuela weg kunnen lopen zonder iets bij je te hebben,
zonder dat iemand er iets van merkt dat je je rijkdom in je hoofd
meedraagt.

Bovendien kan een bitcoin-adres meer dan één privé-sleutel vereisen om
toegang te krijgen. Adressen met verschillende handtekeningen of
multisig-adressen kunnen een grote verscheidenheid aan
beveiligingssystemen gebruiken. Twee mensen kunnen bijvoorbeeld een
rekening delen met 1-of-2 multisig, waarbij elke partij kan tekenen voor
transacties en munten kan uitgeven.

Een 2-of-2 multisig die vereist dat beide partijen sleutels leveren om
uit te geven kan worden gebruikt om te voorkomen dat een enkele persoon
controle krijgt over een rekening, bijvoorbeeld tussen zakenpartners.

Je kunt een eenvoudig escrow-systeem maken met een 2-van-3 multisig. De
koper krijgt een sleutel, de verkoper krijgt een andere sleutel, en een
derde sleutel wordt aan een arbiter gegeven. Als koper en verkoper het
eens zijn, kunnen ze samen de fondsen deblokkeren. In het geval van een
geschil kan de arbiter samen met een van de partijen optreden om de
fondsen te deblokkeren.

U kunt een 3-of-5 multisig schema gebruiken om uzelf te beschermen tegen
het verlies van sleutels door uzelf toe te staan maximaal 2 van de 5
sleutels te verliezen en nog steeds in staat te zijn de rekening te
deblokkeren. U kunt twee van de sleutels op verschillende plaatsen
bewaren, twee bij verschillende betrouwbare vrienden die elkaar niet
kennen, en één bij een gespecialiseerde bewaardienst zoals BitGo die uw
transacties mede ondertekent, waardoor uw bitcoin zeer moeilijk te
stelen is terwijl u uzelf beschermt tegen het verlies van sleutels.

U kunt zelfs nog verder gaan en adressen maken die ontgrendeld worden
door vrij complexe voorwaarden met behulp van programmeerconstructies
zoals voorwaardelijke verklaringen ("als dit, dan dat"). Je zou zelfs
munten kunnen opsluiten in een adres dat pas over 10 jaar toegankelijk
is, en zelfs jij als maker van zo'n adres kunt niet van gedachten
veranderen en de code veranderen om die munten voor die tijd uit te
geven.

Er komen steeds meer semi-custodiale oplossingen van bedrijven zoals
Casa en Unchained Capital, die u helpen om sleutels op een veilige
manier op te slaan. In tegenstelling tot een bank, die je rekening kan
bevriezen, fungeren deze oplossingen voor gedeeltelijke bewaring als een
back-up of vertrouwde medeondertekenaar, maar kunnen ze zelf je fondsen
niet meenemen zonder je sleutels. Portefeuillesoftware evolueert
voortdurend omdat daarvoor niemands toestemming nodig is, in
tegenstelling tot de app van je bank. Daarom zien we steeds meer nieuwe
toetreders en meer innovatie.

Dit is ingrijpend en wereldveranderend. Nooit eerder was het mogelijk om
je vermogen bij je te dragen op een manier die volledig veilig is tegen
inbeslagname of diefstal.

\hypertarget{wie-bepaalt-de-regels}{%
\chapter{Wie bepaalt de regels?}\label{wie-bepaalt-de-regels}}

We hebben nu een functioneel en gedistribueerd systeem om waarde te
volgen en te verplaatsen. Laten we eens kijken wat we zover hebben
gebouwd:

\begin{enumerate}
\def\labelenumi{\arabic{enumi}.}
\item
  Een gedistribueerd grootboek, waarvan iedere deelnemer een kopie
  bewaard.
\item
  Een loterij-systeem gebaseerd op proof-of-work en aanpassingen in
  moeilijkheidsgraad om het netwerk te beveiligen tegen gesjoemel en het
  uitgifteschema consistent te houden.
\item
  Een consensussysteem waarmee iedere deelnemer zelfstandig in staat is
  om de volledige geschiedenis van de blockchain te valideren door
  gebruik te maken van open source software genaamd de bitcoin client.
\item
  Een identificatiesysteem met digitale handtekeningen waarmee naar
  willekeur een account-achtige mailbox kan worden gecreëerd voor de
  ontvangst van munten, zonder tussenkomst van een centrale autoriteit.
\end{enumerate}

Nu is het tijd om een van de meest interessante en contra-intuïtieve
zaken van bitcoin te tackelen. Waar komen de regels vandaag, hoe worden
ze afgedwongen en hoe kunnen ze over tijd veranderen.

\hypertarget{bitcoin-software}{%
\section{Bitcoin-software}\label{bitcoin-software}}

In de vorige hoofdstukken gingen we ervan uit dat iedereen op het
netwerk dezelfde regels valideert: ze verwerpen dubbele betalingen,
controleren ieder blok op proof-of-work, of ieder blok verwijst naar het
vorige blok, en of iedere transactie in ieder blok correct is
ondertekend door de eigenaar van het adres, naast een veelvoud aan
andere zaken waar men het in de loop der tijd over eens is geworden.

We hebben ook gezegd dat bitcoin open source software is. Open source
betekent dat iedereen de code kan lezen, maar ook dat iedereen zijn
eigen kopie kan wijzigen. Hoe komen dergelijke wijzigingen in bitcoin?

Bitcoin is een \emph{protocol}. In computersoftware verwijst deze term
naar een set regels die door de software worden gehanteerd. Zolang je
binnen de regels blijft, staat het je vrij om de software naar wens te
wijzigen. Als we het hebben over mensen die "bitcoin nodes runnen,"
bedoelen we in feite dat ze software draaien die zich aan het bitcoin
protocol houden. Deze software kan met andere bitcoin nodes
communiceren, transacties en blokken doorgeven, andere nodes ontdekken
om mee te verbinden, enzovoorts.

De daadwerkelijke implementatie van het bitcoinprotocol is aan de
gebruiker. Er zijn vele verschillende implementaties van het
bitcoinprotocol. De populairste is Bitcoin Core, een uitbreiding van het
werk dat door Satoshi Nakamoto werd vrijgegeven.

Er zijn ook andere implementaties, in andere computertalen en
onderhouden door andere mensen. Consensus is een essentieel onderdeel
van bitcoin. Om ervoor te zorgen dat de consensus in bitcoin behouden
blijft, draait het grootste gedeelte van de nodes dezelfde Bitcoin Core
software. Zo worden incidentele bugs voorkomen die er anders voor zouden
kunnen zorgen dat nodes het niet langer eens zijn welke blokken geldig
of juist ongeldig zijn. In feite is er geen enkele volledige
specificatie van het bitcoinprotocol, dus de beste manier om nieuwe
bitcoin client software te ontwikkelen is om de originele code te lezen
en te zorgen dat je er niet te veel vanaf wijkt, zelfs als het een
aantal fouten bevat.

\hypertarget{wie-maakt-de-regels}{%
\section{Wie maakt de regels?}\label{wie-maakt-de-regels}}

De regels die bitcoin vormgeven staan gecodeerd in de Bitcoin Core
client. Maar wie bepaalt deze regels? Waarom zeggen we dat bitcoin
schaars is als iemand zomaar het limiet kan wijzigen van 21 naar 42
miljoen?

In een gedistribueerd systeem, moeten alle nodes overeenstemmen over de
regels. Als jij als miner besluit om de software zo te wijzigen dat je
jezelf twee keer zoveel mag belonen bij het minen van een blok dan
volgens het huidige beloningsschema is toegestaan, dan zal iedere andere
node in het netwerk je blok weigeren. Het wijzigen van de regels is
extreem lastig omdat je de duizenden nodes wereldwijd moet overtuigen om
de nieuwe regels te hanteren.

Bitcoins bestuursmodel is contra-intuïtief, met name voor mensen uit
onze westerse democratie. We zijn gewend om op basis van stemmen te
besturen -- de meerderheid van mensen kan bepalen, een wet doorvoeren,
en hun wil opleggen aan de minderheid. Bitcoins bestuursmodel ligt veel
dichter bij anarchie dan bij democratie.

Iedereen die bitcoin-betalingen accepteert, bepaalt voor zichzelf wat
zij als bitcoin beschouwen. Als iemand software draait die zegt dat er
21 miljoen bitcoins zijn, en u probeert hen bitcoins te sturen
geproduceerd door uw malafide software die deze limiet negeert, zullen
uw munten als vals worden gezien en worden geweigerd.

Laten we eens kijken naar de verschillende deelnemers in de
bitcoinwereld en hoe ze elkaar in evenwicht houden:

\textbf{Nodes:} Iedere deelnemer in het bitcoinnetwerk runt een node en
bepaalt zelf welke software hierop draait. De meeste mensen gebruiken
Bitcoin Core. Dit is de implementatie van het bitcoinprotocol die door
Satoshi in het leven werd geroepen en verder is doorontwikkeld door
honderden onafhankelijke ontwikkelaars en bedrijven van over de hele
wereld. Mocht deze software implementatie kwaadaardig blijken,
bijvoorbeeld door inflatie te introduceren, dan zal geen enkele
node-operator het nog langer draaien. Nodes worden onder andere gedraaid
door iedereen die bitcoin accepteert: Handelaren, Exchanges,
Wallet-aanbieders, en mensen zoals jij en ik die bitcoin gebruiken voor
welke reden dan ook.

\textbf{Miners:} Sommige van deze nodes minen ook; ze brengen nieuwe
bitcoin in omloop, nemen transacties op en zorgen ervoor dat het erg
kostbaar wordt om het grootboek te manipuleren. Je zou de miners kunnen
beschouwen als de enige regelgevers omdat zij de enige zijn die in het
grootboek schrijven, maar dat zijn ze niet. Ze volgen de regels die
worden bepaald door de nodes die bitcoin accepteren. Zodra miners
blokken produceren waarin bijvoorbeeld een extra beloning is opgenomen,
dan worden deze verworpen door de andere nodes, met als gevolg dat de
extra beloning geen waarde meer vertegenwoordigd. Dus kan je stellen dat
iedere gebruiker met een node onderdeel is van een anarchisch
overheidsmodel - zij bepalen aan welke regels de munten die zij als
bitcoin beschouwen moeten voldoen, en iedere overtreding van deze regels
wordt direct afgewezen.

\textbf{Gebruikers / Investeerders:} Gebruikers zijn de mensen die (de
valuta) bitcoin kopen en verkopen. Lang niet alle gebruikers runnen hun
eigen node, maar vertrouwen op de node van een derde partij,
bijvoorbeeld de aanbieder van hun portemonnee, waar deze aanbieder
fungeert als soort van proxy voor de wensen en verlangens van hun
gebruikers. Gebruikers bepalen de waarde van de munt op de open markt
door vraag en aanbod. Als de miners en handelsbeurzen zouden
samenspannen om zoiets radicaals als inflatie te implementeren, dan zou
de markt de munt die deze nieuwe regels zou volgen waarschijnlijk
verwerpen, de prijs zou dalen en de bedrijven al snel zonder werk komen
te zitten. Een intolerante minderheid kan de originele munt dus altijd
zelfstandig in leven houden.

\textbf{Ontwikkelaars:} De Bitcoin Core software is het meest populaire
bitcoin client project. Het heeft een rijk ecosysteem van honderden 's
werelds beste crypto ontwikkelaars en bedrijven aan weten te trekken.
Het Bitcoin Core project is uitermate conservatief omdat de software een
netwerk draaiende houdt dat ondertussen meer dan \$1000 miljard aan
waarde vertegenwoordigd. Ieder idee voor grootschalige verandering
doorloopt een proces genaamd Bitcoin Improvement Proposal (BIP)
\footnote{Lees meer over Bitcoin Core's ontwikkelingsprocess in
  \href{https://blog.lopp.net/who-controls-bitcoin-core-/}{blog.lopp.net/who-controls-bitcoin-core-/}}
en iedere verandering in de code wordt nauwkeurig beoordeeld door
collega-ontwikkelaars. Het verbeterproces en de code review is volledig
transparant en publiek. Iedereen mag meedoen, commentaar geven of
wijzigingen in de code voorstellen. Als bepaalde ontwikkelaars corrupt
worden en software ontwikkelen die niemand wil draaien, dan kan een
gebruiker simpelweg zelf bepalen om andere software te draaien, een
oudere versie te hanteren of besluiten om zelf iets nieuws te beginnen.
Hierdoor zijn core developers min of meer gedwongen om veranderingen
door te voeren die naar wens is van de gebruikers, anders lopen ze het
risico om de status van referentie-implementatie te verliezen.

\hypertarget{een-splitsing-van-de-regels-forks}{%
\section{Een splitsing van de regels
(Forks)}\label{een-splitsing-van-de-regels-forks}}

Als het goed is begrijp je nu hoe bitcoin software de afgesproken regels
handhaaft en dat mensen zelf mogen bepalen welke software ze willen
draaien om de regels af te dwingen waar ze in geloven.

Miners bepalen zelf welke regels ze hanteren als ze blokken produceren,
maar zijn gedwongen om hierbij te luisteren naar de wens van de
gebruikers. Doen ze dit niet, dan worden de blokken niet geaccepteerd en
riskeren ze de blokbeloning.

We weten inmiddels ook dat bitcoin software de blockchain met het meeste
proof-of-work accepteert als de enige ware keten en dat er af en toe een
splitsing (een zogeheten fork) ontstaat omdat er bij toeval gelijktijdig
een blok wordt geproduceerd.

Door de grote verscheidenheid aan deelnemers in het netwerk zijn de
regels vanaf het begin zo goed als in steen gebeiteld. Alle protocol
upgrades die tot nu toe zijn gedaan zijn achterwaarts compatibel,
waardoor de basis consensusregels ook voor niet geupgrade nodes
gewaarborgd zijn gebleven.

Laten we nu dan bespreken hoe regels wel kunnen veranderen. Een bewuste
fork vindt plaats als een aantal gebruikers en/of miners er voor kiezen
om niet langer de bestaande regels van bitcoin hanteren. Er zijn twee
verschillende typen forks die zich al hebben voorgedaan: soft-forks,
deze zijn wél achterwaarts compatibel, en hard-forks, deze zijn niet
achterwaarts compatibel. Laten we eens kijken hoe dat werkt.\footnote{\href{https://blog.bitmex.com/bitcoins-consensus-forks/}{Zie
  blog.bitmex.com/bitcoins-consensus-forks/}}

Een \emph{soft-fork} is een achterwaarts compatibele wijziging in de
consensus regels van bitcoin waarbij de regels strenger worden. Dit
betekent dat als je een oude node draait, en deze niet upgrade met de
nieuwe regels, je node de nieuwe blokken desondanks als geldig zal
verklaren. Laten we eens naar een voorbeeld kijken om dat duidelijk te
maken.

Op 12 september 2010 werd er een nieuwe regel geïntroduceerd: blokken
mogen maximaal 1 MB groot zijn. Deze regel werd geïntroduceerd om spam
op de blockchain te voorkomen. Voor deze regel waren blokken van alle
groottes toegestaan, maar volgens de nieuwe regel waren alleen kleinere
blokken geldig. De regels werden dus strenger, maar merk op dat de
nieuwe kleinere blokken ook geldig zijn voor gebruikers die bewust of
onbewust geen upgrade doorvoerden.

Een soft-fork is een niet-disruptieve manier om een systeemverbetering
door te voeren omdat het de node-operators alle tijd geeft om de
software vrijwillig te upgraden. Ook zonder upgrade blijft het mogelijk
om (zoals voorheen) de blokken te controleren. De miners zijn
daarentegen wel gebonden aan de nieuwe regels. Vanaf het moment dat de
miners de upgrade naar 1 MB hadden gemaakt, waren alle blokken max 1 MB
in grootte. Users running old versions of the software were none the
wiser.

In het geval van een hard-fork wordt een niet achterwaarts compatibele
wijziging doorgevoerd. Een \emph{hard-fork} is een verruiming van de
regels waardoor voorheen invalide blokken vanaf dan als valide worden
beschouwd. Bestaande nodes die niet upgraden zullen de blokken die
volgens de nieuwe regels geproduceerd worden afwijzen omdat ze deze
nieuwe blokken als niet valide beschouwen.

Een hard-fork waar bijna alle nodes van het netwerk akkoord mee zijn,
zal weinig problemen geven. Iedere node zal de wijziging haast
onmiddellijk doorvoeren. De enkele achterblijver wordt later alsnog
geforceerd om de software te upgraden, omdat zijn `oude' software geen
blokken meer aan de blockchain toevoegt. Ze voldoen immers niet aan de
oude regels.

In de praktijk gaan hard-forks nooit zonder slag of stoot. In een
volledig gedecentraliseerd en anarchish systeem kan je mensen niet
dwingen om de nieuwe regels te hanteren. In Augustus 2017 waren er een
aantal mensen die niet zo blij waren met de hoge transactiekosten op het
bitcoin netwerk. bitcoin had als gevolg van de soft-fork in 2010 een
regel dat blokken kleiner moesten zijn dan 1 MB. Ze besloten om een
blockchain te creëren met grotere blokken. Deze fork werd bekend als
Bitcoin Cash.

Bij een hard fork waar geen consensus over is (zoals Bitcoin Cash) en
die niet wordt gevolgd door alle miners en nodes, onstaat een nieuwe
blockchain. Deze blockchain deelt de geschiedenis met de orginele chain
inclusief de bestaande UTXO set (account balances) tot het moment waarop
de fork plaatsvindt. Echter, vanaf de splitsing volgt iedere blockchain
zijn eigen weg.

Rondom de fork met Bitcoin Cash ontstonden felle discussies wat juist
\emph{wel} en juist \emph{niet} als de echte bitcoin door mocht gaan.
Sommige Bitcoin Cash voorstanders waren van mening dat bitcoin
gedefinieerd zou moeten worden door wat 10 jaar geleden door Satoshi in
de whitepaper is geschreven. Ze kozen specifieke woorden uit de bitcoin
whitepaper om hun punt te bewijzen. Maar op consensus gebaseerde
systemen geven geen gehoor aan een beroep op autoriteit. Ze werken door
collectieve acties van een groot aantal individuen, zoals de keuze welke
software te draaien en welke munt te kopen of verkopen op de open markt.

In het geval van deze fork besloot het merendeel van de nodes - zoals
portemonnees, handelsbeurzen en merchants - de software niet te wijzigen
omdat de wijziging niet werd gedragen door het overgrote deel van de
ontwikkelaars, en een veel kleinere hash-rate om het netwerk te
beveiligen.

Het probleem van hard forks is dat ze alleen kunnen slagen als iedereen
de overgang maakt Als er te veel achterblijvers zijn, ontstaan er 2
verschillende munten. Dus, bitcoin bleef bitcoin, en Bitcoin Cash werd
een andere munt. Aangezien iedereen die bitcoin had voorafgaand aan de
fork voor niets Bitcoin Cash kreeg toegekend, besloten velen dit "gratis
geld" te verkopen, waardoor de prijs verder omlaag kelderde.

Vandaag bestaan er nog vele andere forks naast bitcoin, zoals Bitcoin SV
(een fork van Bticoin Cash), Bitcoin Gold, Bitcoin Diamond en Bitcoin
Private. Al deze munten hebben slechts een klein deel van bitcoins
hash-rate, weinig ontwikkelaars, en nauwelijks on-chain activiteit en
marktvolume. Hun gebrek aan liquiditeit maakt ze het ideale slachtoffer
voor pump en dumps; een snelle prijsstijging gevolgd door een
spectaculaire en pijnlijke daling. Velen zijn getroffen door portemonnee
hacks, 51\% attacks en andere rampen. Sommige zijn complete oplichterij,
of simpelweg voor gokkers en bijna allemaal een hoge mate van
centralisatie. De website \href{https://forkdrop.io/}{forkdrop.io} telt
op dit moment 74 bitcoin-wannabes.

\begin{figure}

{\centering \includegraphics{./images/fig15.png}

}

\caption{\label{fig-figuur15}Munten van een soft-fork kunnen naar oudere
nodes gestuurd worden. Een hard-fork produceert nieuwe
terugwaarts-incompatibele UTXO's die niet geaccepteerd zullen worden
door oude nodes.}

\end{figure}

Vele andere munten gebruiken vergelijkbare code, maar zijn begonnen met
een lege blockchain zonder bitcoins transactiegeschiedenis. Voorbeelden
hiervan zijn Litcoin en Dogecoin. Dergelijke munten zijn geen bitcoin
forks omdat ze niet dezelfde transactiegeschiedenis delen, ookal zouden
ze in code vergelijkbaar kunnen zijn.

Een Bitcoinfork heeft geen enkel effect met betrekking tot de limiet van
21 miljoen Bitcoins. Vergelijk het met een wereld waarin de wereldwijde
goudvoorraad ligt opgeslagen in het zwaarbewaakte en streng ontworpen
Fort Knox. Nu bouw jij een klein en slecht ontworpen hutje, zet er een
enkele bewaker voor en noemt het Fort Knox Lite. Je verft wat stenen met
bladgoud en stopt ze in het hutje. Daaropvolgend breng je een
nieuwsbericht uit om de wereld te vertellen dat je het goud hebt
opgesplitst en iedere goudbezitter een equivalente hoeveelheid aan
gratis stenen zult toekennen die opgeslagen liggen in je hutje.

Bitcoin wordt inmiddels beveiligd door een groot aantal miners, waardoor
een 51\% aanval vrijwel onbetaalbaar is geworden. Een fork van Bitcoin
met slechts een handjevol miners is daarentegen ontzettend kwetsbaar. De
code is waarschijnlijk gebrekkig en gebouwd door een onervaren team van
ontwikkelaars, netzoals je hutje. De Bitcoin forks worden dan ook niet
geaccepteerd door de bestaande nodes omdat ze breken met de regels van
Bitcoin, netzoals iemand een steen met bladgoud ook niet zou accepteren.
De productiekosten van de gesplitse munten en stenen zijn
verwaarloosbaar, aangezien je ze gratis aan iedere bezitter hebt
overhandigd. Daarmee wordt ook de marktinteresse al snel minder.

Nu je nadenkt over de duizenden Bitcoinklonen, allen zonder significante
waarde, beschouw de volgende paradox: Bitcoin opsplitsen is eenvoudig en
kost je niets. De regels van Bitcoin wijzigen of nieuwe bitcoin creëren
is dat allesbehalve. Mocht iemand met beperkte Bitcoinkennis je vragen
waarom Bitcoin speciaal is, dan weet je nu wat je kunt antwoorden.

De decentrale aard van het Bitcoin ecosysteem geeft grote voorkeur aan
de status quo. Het implementeren van grote veranderingen neemt maanden
of jaren in beslag aan het verkrijgen van consensus, discussie en peer
review. Dit is van groot belang en wenselijk voor een systeem met als
doel om als wereldwijd geld te fungeren. Bitcoin is een delicate dans
tussen duizenden deelnemers die allemaal vanuit eigenbelang handelen, en
vaak met tegengestelde belangen. Het is een volledig anarchisch
vrije-markt systeem waar niemand de baas is.

\hypertarget{whats-next}{%
\chapter{What's next?}\label{whats-next}}

\hypertarget{is-bitcoin-de-myspace-van-crypto}{%
\section{Is Bitcoin de MySpace van
crypto?}\label{is-bitcoin-de-myspace-van-crypto}}

Waarom schrijf ik een boek over Bitcoin en niet over het grotere crypto
ecosysteem? Zijn er geen duizenden andere munten? Wat maakt bitcoin zo
speciaal, anders dan dat het de eerste decentrale cryptovaluta is? Is
het niet trager dan andere munten en met minder mogelijkheden dan de
nieuwe concurrenten?

Dit is een veelgestelde vraag van nieuwkomers. Na een eerste introductie
tot bitcoin en haar werking, is de volgende logische vraag vaak: "Zo'n
blockchain is interessant, maar hoe weten we dat er geen betere versie
opdaagt die van bitcoin de MySpace maakt van crypto?"

Bitcoins concurrentievoordeel is veel, veel groter dan het voordeel
destijds van MySpace. Laten we eens kijken wat een concurrent nodig
heeft om bitcoin te vervangen.

\hypertarget{gemakkelijker-verkoopbaar-en-grotere-liquiditeit}{%
\section{Gemakkelijker verkoopbaar en grotere
liquiditeit}\label{gemakkelijker-verkoopbaar-en-grotere-liquiditeit}}

Het eerste om te begrijpen is dat een vergelijking met MySpace en
Facebook zwak is, omdat het je niets kost om tegelijkertijd op beide
platforms een account te onderhouden. Dit is precies wat er gebeurde
tijdens de transitie van de één naar de ander. Zodra genoeg kritieke
massa was overgestapt naar Facebook, stopte men met MySpace.

Maar dit is niet hoe geld werkt. Als jij een euro aan bitcoin hebt, is
dat een euro die je niet in een andere munt hebt. Je moet er bewust voor
kiezen om de één voor de ander te verhandelen. Het is onmogelijk om
tegelijkertijd hetzelfde vermogen op te slaan in verschillende munten.
Vraag jezelf eens af: waarom zou je geld bewaren in iets anders dan de
meest liquide en best geaccepteerde valuta. Het enige antwoord is
speculatie. Als je niet in staat bent om de gehele economie te
overtuigen om de andere munt te accepteren, dan is er geen enkele manier
dat het dominant wordt.

De liquiditeit van bitcoin is vele malen groter dan alle concurrenten.
Op moment van schrijven is de marktkapitalisatie van bitcoin ongeveer
\$1 biljard (1000 miljard).\footnote{\href{https://messari.io/onchainfx}{messari.io/onchainfx}}
De eerstvolgende concurrent, Ethereum, heeft een kapitalisatie van
slechts \$500 miljard. Dit zegt nog niets over de ware liquiditeit, de
mate waarin je een aanzienlijk bedrag kunt verkopen zonder dat de prijs
drastisch zal dalen.

Liquiditeit heeft een sneeuwbaleffect. Het meeste liquide geld
aanhouden, betekent dat andere mensen het willen, waardoor de
liquiditeit nog verder toeneemt. Door vast te houden aan alles behalve
het meest liquide geld, straf je jezelf in de hoop dat anderen hetzelfde
gaan doen. Er is simpelweg geen enkele economisch motief om van de één
op de andere dag massaal op een concurrent over te stappen.

\hypertarget{aantoonbaar-tien-jaar-1000-miljard-beveiligd}{%
\section{Aantoonbaar tien jaar \$1000 miljard
beveiligd}\label{aantoonbaar-tien-jaar-1000-miljard-beveiligd}}

Bitcoin begon in 2009 als internet experiment voor computernerds.
Slechts 1 jaar later volgde de eerste transactie (een pizza voor 10000
bitcoins) en inmiddels is 1 bitcoin al meer dan \$60.000 waard. Dit
gebeurde in relatieve rust, zonder al te veel ophef. Door de jarenlange
aanvallen is immuunsysteem van bitcoin inmiddels van wereldklasse, met
het grootste netwerk aan rekenkracht ter wereld. Al tien jaar lang
blijkt het onmogelijk om te hacken en het beveiligt inmiddels meer dan
1000 miljard dollar.

Het is haast onmogelijk om vandaag de dag in stilte een nieuwe
cryptovaluta te lanceren. Laten we eens kijken naar een alternatieve
blockchain, EOS, met een waarde van ongeveer ~\$10 miljard bij de
lancering van het netwerk (en vandaag nog slechts de helft waard). Het
netwerk moest slechts 2 dagen na aanvang al op slot wegens fouten in de
code. Deze fouten werden zonder al te veel toezicht of review verholpen.
Durf jij hier \$1000 miljard in bewaren? Misschien bestaat EOS over 10
jaar nog, wie het weet mag het zeggen. Maar tegen die tijd is bitcoin 20
jaar oud en beveiligt het biljarden aan waarde.

\hypertarget{aanvallen-met-bestaande-rekenkracht-afslaan}{%
\section{Aanvallen met bestaande rekenkracht
afslaan}\label{aanvallen-met-bestaande-rekenkracht-afslaan}}

Met duizenden crypto's en allerlei hashing-algoritmes, staat iedere
nieuwe munt onder bedreiging van een 51\%-aanval met bestaande
rekenkracht. Bitcoin Gold en vele andere munten zijn hier al slachtoffer
van geweest.

Een nieuwe concurrent moet deze aanvallen van bestaande rekenkracht dus
kunnen overleven, of moet op zoek naar een algoritme zonder
gespecialiseerde ASIC's. Maar zonder ASIC's is het systeem juist weer
kwetsbaar voor een aanval met standaard GPU's. Een ander probleem is dat
een nieuwe concurrent niet zomaar vanaf dag 1 veel waarde kan
beveiligen, zoals EOS probeerde. Dit is roekeloos en voorbode voor een
gecentraliseerd systeem. Dat betekent dus ook dat een nieuwe concurrent
niet zomaar geld op kan halen, maar net zoals bitcoin langzaam zal
moeten groeien om de beveiliging proportioneel te laten groeien. Echter,
met langzame groei wordt het haast onmogelijk om bitcoin nog in te
halen.

\hypertarget{decentralisatie}{%
\section{Decentralisatie}\label{decentralisatie}}

Een groot gedeelte van bitcoin's beveiligingsmodel berust op een hoge
graad van decentralisatie. Dit betekent dat het moeilijk is om het
protocol te wijzigen en men erop kan vertrouwen dat het de eigenschappen
zoals beloofd in de broncode zal waarborgen (gelimiteerd aanbod, etc).
Bitcoin bewees deze eigenschap te bezitten toen een groot aantal
bedrijven en miners de blockgrootte wilden wijzigen, om het protocol een
bepaalde richting op te sturen.\footnote{Lees hier meer over de
  achterkamertjespolitiek die plaatsvond bij de zogeheten Segwit2X fork:
  \href{https://bitcoinmagazine.com/articles/now-segwit2x-hard-fork-has-really-failed-activate}{bitcoinmagazine.com/articles/now-segwit2x-hard-fork-has-really-failed-activate}}.
De voorgestelde aanpassingen faalden spectaculair nadat ze werden
afgewezen door de gebruikers van bitcoin.

Een concurrent die decentralisatie nastreeft zal bedrijven en teams met
bekende mensen moeten uitsluiten, om de onderdrukking en single points
of failure te voorkomen. Ook munten met het motto move fast and break
things zijn uitgesloten, want dat kan alleen bij voldoende
centralisatie. Al met al is een concurrent dus snel en kwetsbaar voor
centralisatie, of traag en niet in staat om bitcoin in te halen.

\hypertarget{trek-de-beste-ontwikkelaars-aan}{%
\section{Trek de beste ontwikkelaars
aan}\label{trek-de-beste-ontwikkelaars-aan}}

Zoals de wervelwind van activiteit bij Linux, andere UNIX-achtige
besturingssystemen weerhield van concurrentie, zo kan het ook gebeuren
bij bitcoin. Elke dag wordt de gemeenschap groter en worden nieuwe
bedrijven, met nieuwe diensten, gebouwd bovenop bitcoin. Een concurrent
moet ontwikkelaars zien te stelen van een exponentieel groeiend netwerk,
met inmiddels honderden bedrijven en tal van educatieve programma's en
conferenties.

\hypertarget{een-wereldwijd-financieel-netwerk}{%
\section{Een wereldwijd financieel
netwerk}\label{een-wereldwijd-financieel-netwerk}}

Bitcoin wordt inmiddels gedragen door een wereldwijd netwerk van
handelsbeurzen, futures en andere financiële derivaten bij grote spelers
zoals de Chicago Mercantile Exchange (CME), honderden hefboomfonsen en
handelskantoren, en een netwerk van mensen die bitcoin al gebruiken als
alternatief voor gebroken valuta zoals de Venezulaanse bolivar. Bitcoin
laat zich dus niet zo makkelijk vervangen.

Instellingen zoals de CME nemen een nieuwe concurrent pas op als er al
voldoende handelsvolume is. Je zult bedrijven er dus van moeten
overtuigen om de nieuwe concurrent te accepteren in plaat van bitcoin.
Een concurrent die waarschijnlijk minder veilig en minder liquide is,
minder competente ontwikkelaars heeft en per definitie over minder
wereldwijde adoptie beschikt.

\hypertarget{solide-geld}{%
\section{Solide geld}\label{solide-geld}}

Bitcoin is nooit bedoeld als snelle en goedkope manier van betalen. Dat
is een groot misverstand. De fundamentele eigenschappen waarbij het
grootboek wereldwijd wordt gerepliceerd, staan dat simpelweg niet toe.
Daarentegen groeit bitcoin's primaire en reeds bewezen toepassing als
censuur-resistent, solide geld.

Alles wat daarbij komt, zoals het goedkoper maken van internationale
overboekingen, zijn kersen op de taart. De meeste concurrenten richten
zich nog steeds op snelle betalingen, terwijl dat probleem al redelijk
goed is opgelost door vele gecentraliseerde bedrijven. Daarnaast heeft
bitcoin daar ook inmiddels een oplossing voor gevonden met het snel
groeiende Lightning Netwerk.

Om te kunnen concurreren op het front van solide geld moeten
onveranderbare eigenschappen en decentralisatie ten alle tijde voorop
staan bij de ontwikkeling. Bitcoin's ecosysteem is gebouwd door
\emph{cypherpunks} en heeft de kans gehad om langzaam te groeien, maar
de meeste munten worden ontwikkeld door winstgedreven, gecentraliseerde
teams en maken dus weinig kans op slagen.

\hypertarget{toekomstige-ontwikkelingen-in-bitcoin}{%
\section{Toekomstige ontwikkelingen in
Bitcoin}\label{toekomstige-ontwikkelingen-in-bitcoin}}

We hebben inmiddels het volledige protocol onder de loep genomen. Laten
we eens kijken naar de toekomst en sommige van de aankomende
verbeteringen bespreken.

Bitcoin is programmeerbaar geld waar we allerlei diensten bovenop kunnen
bouwen. Dit is een volledig nieuw concept en we staan pas aan het begin
van wat mogelijk is.

\hypertarget{lightning-netwerk}{%
\section{Lightning Netwerk}\label{lightning-netwerk}}

Bitcoin heeft verschillende periodes van hoge transactievergoedingen
gekend op momenten dat er hoge vraag was naar de blokruimte. Bitcoin is
op dit moment slechts in staat om ongeveer 3 tot 7 transacties per
seconde te verwerken, op basis van de hoeveelheid transacties die in een
blok passen. Ook al kan een batch-transactie honderden mensen betalen in
een transactie, dat is nog steeds te weinig capaciteit voor een
wereldwijd betalingsnetwerk.

Het vergroten van de blokruimte is een naïve oplossing en desondanks
door veel concurrenten, waaronder Bitcoin Cash, toegepast. Bitcoin heeft
gekozen voor een andere route omdat het vergroten van de blokken de
decentrale eigenschappen zoals het aantal nodes en de geografische
verspreiding van nodes negatief beïnvloedt.

Een verhoging van de blokgrootte kan er hoe dan ook niet voor zorgen dat
bitcoin geschikt wordt als wereldwijd betalingsnetwerk --- het schaalt
simpelweg niet genoeg. Welkom bij het Lightning Netwerk: een protocol en
verschillende software implementaties om \emph{off-chain}
bitcointransacties te verwerken die periodiek kunnen verrekend worden op
de blockchain. Over het Lightning Netwerk zouden we een heel boek kunnen
schrijven, maar we beperken ons hier tot de hoofdlijnen.

De basisgedacht bij Lightning is dat we niet iedere transactie op de
blockchain hoeven te registreren. Vergelijk het met een tijdelijke
rekening bij de kroeg, waar drankjes worden aangestreept en pas aan het
eind van de avond wordt afgerekend. Iedere drankje afzonderlijk betalen
is tijdrovend en omslachtig. Voor bitcoin geldt min of meer hetzelfde:
iedere koffie of ieder biertje registreren op de blockchain en die data
verspreiden over duizenden computers over de wereld is niet schaalbaar,
noch goed voor jouw privacy.

Het Lighting Netwerk heeft het potentieel om bitcoin op vele fronten te
verbeteren:

\begin{itemize}
\item
  Vrijwel ongelimiteerd transactievolume: honderdduizenden
  micro-transacties uitvoeren en eenmalig verrekenen op de blockchain.
\item
  Directe betaling: Wachten op confirmatie op de blokchain is niet
  langer nodig.
\item
  Extreem lage transactievergoedingen: geschikt voor micro-betalingen
  zoals de betaling van enkele centen om een blog te lezen.
\item
  Betere privacy: slechts de deelnemende partijen aan de transactie
  hebben er kennis van, in tegenstelling tot on-chain betalingen die met
  de hele wereld gedeeld worden.
\end{itemize}

Lightning maakt gebruik van betalingskanalen. Dit zijn on-chain
bitcointransacties waarbij een hoeveelheid bitcoin wordt vastgezet en
beschikbaar gemaakt binnen het Lightning Netwerk, voor directe, bijna
kosteloze transacties. Het Lightning Netwerk is in de beginfase maar nu
al veelbelovend. Zie als voorbeeld de website
\href{https://yalls.org}{yalls.org} waar je via Lightning kunt betalen
om artikelen te lezen.

\hypertarget{bitcoin-in-de-ruimte}{%
\section{Bitcoin in de ruimte}\label{bitcoin-in-de-ruimte}}

Bitcoin is uitstekend bestand tegen censuur omdat het bestand is tegen
aanvallen (je kan je eigendom in je hoofd bewaren), en bestand tegen
censuur omdat er maar één eerlijke miner op het netwerk nodig is om jouw
transacties uit te voeren (en je kunt zelf minen).

Desalniettemin, aangezien bitcoin via internet wordt verstuurd is het
vatbaar voor censuur op netwerkniveau. Bijvoorbeeld doordat autoritaire
regimes die bitcoin willen onderdrukken, kunnen proberen om te
verhinderen dat bitcoindata hun land binnenkomt en verlaat via het
internet.

Het Blockstream Satellite-netwerk is een eerste poging om het netwerk
uit te breiden en te beveiligen tegen censuur op staatsniveau, evenals
een poging om afgelegen gebieden te bereiken die mogelijk geen
verbinding met het internet hebben. Dit satellietsysteem maakt het voor
iedereen met een schotel en relatief goedkope apparatuur, mogelijk om
verbinding te maken en de bitcoin-blockchain te downloaden. Binnenkort
wordt bidirectionele communicatie ook mogelijk. Er zijn nu ook
inspanningen zoals TxTenna, die bouwen aan \emph{off-the-grid
mesh-netwerken}. Als zo'n setup gecombineerd zou worden met een
satellietverbinding, zou het bijna niet te stoppen zijn.

\hypertarget{verder-lezen}{%
\chapter*{Verder lezen}\label{verder-lezen}}
\addcontentsline{toc}{chapter}{Verder lezen}

\markboth{Verder lezen}{Verder lezen}

Dit was het. Je hebt bitcoin onder de loep genomen, en hopelijk ben je
nu klaar om de wondere wereld van bitcoin verder te verkennen. Waar ga
je heen vanaf hier? Hier zijn een paar bronnen om je verder te helpen:

\begin{itemize}
\item
  \emph{De Bitcoin Standaard} (Saifedean Ammous). Verkrijgbaar bij
  \href{https://konsensus.network/product/de-bitcoin-standaard}{Konsensus
  Network}
\item
  \emph{Bitcoin Investment Theses} (Pierre Rochard).
  \href{https://pierre-rochard.medium.com/bitcoin-investment-theses-part-1-e97670b5389b}{Online
  lezen}
\item
  \emph{Dank God voor Bitcoin/} (Lyle Pratt, George Mekhail, Jimmy Song,
  Gabe Higgins, Julia Tourianski, Derek Waltchack, Robert Breedlove en
  J.M. Bush). Verkrijgbaar bij
  \href{https://konsensus.network/product/dank-god-voor-bitcoin/}{Konsensus
  Network}
\item
  \emph{Gelaagd Geld} (Nik Bahtia). Verkrijgbaar bij
  \href{https://konsensus.network/product/gelaagd-geld/}{Konsensus
  Network}
\end{itemize}

Om meer te lezen over de technische aspecten:

\begin{itemize}
\item
  De whitepaper; \emph{Bitcoin: Een Peer-to-Peer Electronisch geld
  Systeem} (Satoshi Nakamoto).
  \href{https://bitcoin.org/files/bitcoin-paper/bitcoin_nl.pdf}{Online
  lezen}
\item
  \emph{Mastering Bitcoin} (Andreas Antonopoulos)
\item
  \emph{Programming Bitcoin} (Jimmy Song)
\item
  Jimmy Songs seminar is beschikbaar op:
  \href{https://programmingblockchain.com}{programmingblockchain.com}
\end{itemize}

Achtergrondinformatie over de geschiedenis en filosofie van Bitcoin:

\begin{itemize}
\item
  \emph{Planting Bitcoin} (Dan Held). Zie
  \href{https://medium.com/@danhedl/planting-bitcoin-sound-money-72e80e40ff62}{medium.com/@danhedl/planting-bitcoin-sound-money-72e80e40ff62}
\item
  \emph{Bitcoin Governance} (Pierre Rochard). Zie
  \href{https://medium.com/@pierre_rochard/bitcoin-governance-37e86299470f}{medium.com/@pierre\_rochard/bitcoin-governance-37e86299470f}
\item
  \emph{Bitcoin Past and Future} (Murad Mahmudov). Zie
  \href{https://blog.usejournal.com/bitcoin-past-and-future-45d92b3180f1}{blog.usejournal.com/bitcoin-past-and-future-45d92b3180f1}
\item
  De video's van Andreas Antonopoulos video's, en in het bijzonder
  \emph{Currency Wars} en \emph{The Monument of Immutability}. Zie
  \href{https://www.youtube.com/user/aantonop}{youtube.com/user/aantonop}
\end{itemize}

Een groot deel van het bitcoin-ecosysteem leeft op Twitter. Hier is een
lijst van mensen in willekeurige volgorde, die de moeite waard zijn om
te volgen. Begin hier en ga dieper het konijnenhol in:

\href{https://twitter.com/lopp}{@lopp}\\
\href{https://twitter.com/pwuille}{@pwuille}\\
\href{https://twitter.com/adam3us}{@adam3us}\\
\href{https://twitter.com/danheld}{@danheld}\\
\href{https://twitter.com/pierre_rochard}{@pierre\_rochard}\\
\href{https://twitter.com/bitstein}{@bitstein}\\
\href{https://twitter.com/theonevortex}{@theonevortex}\\
\href{https://twitter.com/AlenaSatoshi}{@AlenaSatoshi}\\
\href{https://twitter.com/WhatBitcoinDid}{@WhatBitcoinDid}\\
\href{https://twitter.com/stephanlivera}{@stephanlivera}\\
\href{https://twitter.com/TheBlock__}{@TheBlock\_\_}\\
\href{https://twitter.com/TheLTBNetwork}{@TheLTBNetwork}\\
\href{https://twitter.com/real_vijay}{@real\_vijay}\\
\href{https://twitter.com/jimmysong}{@jimmysong}\\
\href{https://twitter.com/Excellion}{@Excellion}\\
\href{https://twitter.com/starkness}{@starkness}\\
\href{https://twitter.com/dickerson_des}{@dickerson\_des}\\
\href{https://twitter.com/roasbeef}{@roasbeef}\\
\href{https://twitter.com/saifedean}{@saifedean}\\
\href{https://twitter.com/Melt_Dem}{@Melt\_Dem}\\
\href{https://twitter.com/_jillruth}{@\_jillruth}\\
\href{https://twitter.com/giacomozucco}{@giacomozucco}\\
\href{https://twitter.com/Snyke}{@Snyke}\\
\href{https://twitter.com/aantonop}{@aantonop}\\
\href{https://twitter.com/MustStopMurad}{@MustStopMurad}\\
\href{https://twitter.com/danheld}{@danheld}\\
\href{https://twitter.com/peterktodd}{@peterktodd}\\
\href{https://twitter.com/dergigi}{@dergigi}\\
\href{https://twitter.com/skwp}{@skwp}\\
\href{https://twitter.com/konsensusn}{@KonsensusN}

Je kunt meer van mijn geschriften vinden op
\href{https://yanpritzker.com}{yanpritzker.com}.

Tot ziens aan de andere kant.

\hypertarget{dankwoord}{%
\chapter*{Dankwoord}\label{dankwoord}}
\addcontentsline{toc}{chapter}{Dankwoord}

\markboth{Dankwoord}{Dankwoord}

Ik wil graag mijn dank uitspreken richting de vele mensen die me
feedback hebben gegeven tijdens de vroege concepten van dit boek. In het
bijzonder: Joe Levering, Phil Geiger, Yury Pritzker, Jonathan Wheeler,
Walter Rosenberg, Michael Santosuosso en David Harding. Ook wil ik Jimmy
Song bedanken voor het Programming Blockchain-seminar, waaardoor ik de
schop onder mijn kont kreeg die ik nodig had om dit boek te schrijven.

\hypertarget{over-de-auteur}{%
\chapter*{Over de auteur}\label{over-de-auteur}}
\addcontentsline{toc}{chapter}{Over de auteur}

\markboth{Over de auteur}{Over de auteur}

Yan Pritzker is al 20 jaar ontwikkelaar en ondernemer, met veel ervaring
in het opstarten van bedrijven. Yan is mede-oprichter en CTO van Swan
Bitcoin, een bedrijf dat zich richt op educatie en begeleiding van de
volgende tien miljoen (nieuwe) bitcoiners. Yan schrijft over bitcoin en
aanverwante onderwerpen op
\href{https://yanpritzker.com}{yanpritzker.com}. Je kunt hem ook volgen
op Twitter: \href{https://twitter.com/skwp}{@skwp}.

\part{Extra}

\hypertarget{references}{%
\chapter*{References}\label{references}}
\addcontentsline{toc}{chapter}{References}

\markboth{References}{References}

\hypertarget{refs}{}
\begin{CSLReferences}{0}{0}
\end{CSLReferences}


\backmatter

\end{document}
